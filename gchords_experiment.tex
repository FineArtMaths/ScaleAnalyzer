\documentclass[english]{./gbook}
\usepackage{babel}
\usepackage{verbatim}
\usepackage{amssymb}
\usepackage{amsmath}
\usepackage{mathrsfs}
\usepackage{geometry}
\usepackage{fancyhdr}
\usepackage[all]{xy}
\xyoption{arc}

\newcommand{\helv}{\fontfamily{phv}\fontseries{b}\fontsize{9}{11}\selectfont}
\newcommand{\titlebreak}{}

\pagestyle{fancy}
	\renewcommand{\chaptermark}[1]{\markboth{\helv \chaptername\ \thechapter:\ #1}{}}
	\fancyhead[RO,LE]{\helv \thepage}
	\fancyfoot[RO,LE,RE,LO]{}
	\fancyfoot[CE,CO]{}
	\renewcommand{\headrulewidth}{0.4pt}
	\renewcommand{\footrulewidth}{0pt}
	
\linespread{1.1}

\makeatother

\begin{document}
	\marginparpush=0pt
	\marginparsep=0pt
	\oddsidemargin=18pt
	\evensidemargin=18pt
	\marginparwidth=54pt
	\headheight=0pt
	\headsep=45pt
	\textwidth=424pt
	\textheight=565pt



\frontmatter
\pagestyle{empty}
\vspace*{120pt}
\begin{center}
\begin{tabular}{r}\hline
\\
{\large Rich Cochrane
} \\
\textbf{{\Huge
Arpeggio and Scale Resources}} \\ 
{\large A Guitar Encyclopedia
} \\
\\
\hline
\end{tabular}
\end{center}

\newpage
\pagestyle{empty}
\vspace*{250pt}
\begin{center}
\begin{tabular}{r}
B I G \ N O I S E \ P U B L I S H I N G\\
Published by Big Noise Publishing, London, UK\\
Published under the following Creative Commons license:\\
Attribution-NonCommercial-NoDerivs 2.0 UK: England and Wales (CC BY-NC-ND 2.0)\\
http://creativecommons.org/licenses/by-nc-nd/2.0/uk/\\
\\
http://cochranemusic.com/arpeggios-and-scales-guitar-encyclopedia\\
\\
Some rights reserved. Unaltered electronic copies of this publication may be circulated freely. \\
You are not permitted to sell this work in any form.\\
\\
A paper copy of this book is available from Lulu.com. If you like it, and especially if you teach \\
from it, please consider supporting the author by purchasing a copy from the link above.\\
\\
Typeset in \LaTeX\\
\end{tabular}
\end{center}
\pagestyle{empty}

\pagestyle{headings}

\tableofcontents
\mainmatter

\begin{large}

\chapter*{About This Book}

This book is divided into three parts. The first part, covering chapters 1 to 4, is a detailed study of arpeggio patterns and an introduction to the study of scales. It covers a lot more than most guitarists know about scales, but it's really just the beginning. The second and third parts present a huge number of scales, most of which have never had their fingering diagrams published. This collection is unique in the literature of the guitar or, as far as the author knows, among music books in general.

If you're a beginner, take the material in the first part slowly and supplement it with intensive listening to the kind of music you want to play. You may also want to add a book of `licks' or transcribed solos to help you apply the theoretical material in this book to the musical style that interests you. Scale knowledge may be more or less important depending on the kind of music you want to play, but it is rarely interesting to hear a musician `playing scales' as such. This book provides the raw material; it's up to you to figure out how to turn it into music, and that's been done in a variety of ways down the centuries.

Advanced players will mostly be interested in the second and third parts. Use them as a source of inspiration: dip in more or less at random to find new and unusual sounds. This part of the book isn't intended to be read as a course but to act as a reference book that will hopefully inspire you to get out of a rut whenever you find yourself in one.

\part{Fundamentals}

\chapter{\mbox{Basic} \mbox{Scale} \mbox{Theory}}

This chapter introduces the basic theoretical ideas behind the study of scales. The reader may already have met most of the terms introduced in this chapter, although perhaps without clearly seeing how they all fit together. Some of this chapter is quite theoretical; it's not essential to understand everything here in order to use the rest of the book, but it will certainly help.

\section*{Pitches, Notes, Pitch Classes and Intervals}

We all know that pitches can be very low or very high; the pitches we can hear range from the deep rumble of a lorry driving past to the most piercing whistle. We can imagine pitch as being a continuous, smooth slope from low to high. Pitches are measured in hertz (Hz), a number that describes where on the slope they are; an average person can hear any pitch that lies between about 16Hz and about 16,000Hz.

Musicians, though, almost never think about the pitches they're playing in these terms. The reason for this is that we don't use the full range of possibilities in ordinary music-making. Instead, we've chosen certain specific pitches and called them `notes'. Think of the way a piano keyboard is divided up into distinct keys, rather than being a smooth surface. The keys are notes, which are those particular pitches we've chosen to form our musical system. Other cultures use other notes and some use none at all, but we'll only be concerned with the notes on the piano in this book.

The guitar fretboard is more or less the same as a piano keyboard: the frets prevent us, usually, from playing any pitches except the ones we call `notes'. Guitarists can get around this by bending the string or using a slide, but we still use the notes as reference-points in most music. A violin's fingerboard is smooth, but that just means that violinists have to practice very hard to `play in tune' -- that is, to be able to play the pitches that our music recognises as `proper notes'.

Now, it so happens that there's an acoustic phenomenon that causes one pitch that's exactly half or double another one to sound oddly similar to it. For example, if we play the pitch 400Hz, the pitches 800Hz (2$\times$400), 1600Hz (2$\times$800) and so on sound strongly `in agreement' with it. So much so, in fact, that we give them all the same name. So the note on your open sixth string is called `E', and so is the note at the second fret of the fourth string, even though it's obviously not the same pitch. The same goes for the open first string; its pitch is four times the pitch of the bottom string, and so it's called `E' as well.

We can't say these three notes have the same `pitch', because that would be wrong; clearly they don't. So instead we say they belong to the same `pitch class'. There are twelve pitch classes in Western music that, for historical reasons, are named like this:
\begin{quote}
\begin{tabular}{llllllllllll}
	A & A$\sharp$ & B & C & C$\sharp$ & D & D$\sharp$ & E & F & F$\sharp$ & G & G$\sharp$ \\
	  & B$\flat$  &   &   & D$\flat$  &   & E$\flat$  &   &   & G$\flat$  &   & A$\flat$
\end{tabular}
\end{quote}
Where two names are given, one underneath the other, these are just two names for the same pitch class\footnote{There are other `enharmonic' names, too, such as B$\sharp$ and D$\flat\flat$, but these needn't concern us here.}. The term `pitch class' is unfamiliar to most people, and in fact we'll generally use the term `note' to mean `pitch class'. Whether you play a high `E' or a low one is up to you; it's the fact that it's an `E' that we're interested in.

So, playing A, A$\sharp$ and so on creates a series of notes that get higher and higher in pitch. When we reach G$\sharp$ and go up one more step to the next note we find it's also called `A' just like the note we started on, but this one is much higher in pitch. This A is said to be `one octave higher' than the first one. They're the same \emph{pitch class} but different \emph{pitches}.

If you choose any two notes, we call the difference between them an `interval'. Two notes that are very different in pitch (say, a very high note and a much lower one) are said to be separated by a large interval. Two notes that are closer in pitch are separated by a smaller interval.

The smallest interval is called a `semitone', which is the distance between one fret and the next one up or down. We can describe any interval by the number of semitones it contains. The number of semitones in an interval between two notes tells you how many frets would lie between them if you were to play both notes on the same string (naturally you might choose not to do that if the interval is larger than about four semitones). Unfortunately the name of the interval, although it's usually a number, isn't necessarily the number of semitones it contains. Such is the whimsical nature of musical terminology, although in this case it might make more sense by the end of this chapter. Most of this book requires only the following intervals, which it will often be convenient to abbreviate as follows:
\begin{quote}
\begin{tabular}{lll}
	Interval & Abbreviation & Number of frets \\
	\hline Semitone & s & 1 fret \\
	Tone	& t & 2 frets \\
	Minor third & mT & 3 frets \\
	Major third & MT & 4 frets \\
	Fourth	& Fo & 5 frets \\
	Fifth	& Fi & 7 frets
\end{tabular}
\end{quote}
Ideally you should know the name and sound of all the intervals, including but not limited to those above, and be able to find them on the guitar. This facility comes with time and familiarity and will be greatly aided by the study of arpeggios that this book covers in detail. Fingerings for all the intervals in the octave can be found in the last chapter of this part of the book.

\section*{Gamuts and Scales}

Taking all of the twelve notes in our musical system and playing around with them is a curiously unrewarding exercise, at least for the beginner. Although you can play almost any music -- including the solos of John Coltrane, the operas of Wagner and the complete works of Frank Zappa -- using just those twelve notes, if you try you will find it tricky to do anything very interesting or, just as importantly, to understand what you did enough to enable you to develop it. The problem is that the undifferentiated string of semitones doesn't give you any structure to work with; it's a kind of unrefined raw material. For this reason, most musical traditions recognise the idea of a `gamut'. A gamut is nothing more complicated than a selection of some of the available notes. Say we've chosen just A C D E and G. We'll only play those five notes; then A C D E G is a gamut drawn from the twelve notes we have available.

The nice thing about a gamut is that it generally has some structure embedded in it. Those notes A C D E G do not, for example, have the same intervals between all of them; some are `further apart' than others. Try playing these notes on the guitar and you may find it possible to produce something like music with them; much more so, anyway, than with the `total chromatic' of twelve notes\footnote{Superb music has been created using the total chromatic; the compositions of the Second Viennese School and their followers provide many examples. Yet these achievements required other methods of organisation and a deliberate rejection of the tonal system on which this book is based. This book teaches just one way of organising musical materials; it has been an extremely productive approach, but it is by no means the only valid one.}. Don't spend too much time on this, though, because gamuts are not the main focus of our attention.

A scale is not just a gamut, but a gamut with a `root note' (usually just called the `root'). The root note isn't something extra added to the gamut, it's just one of the notes chosen to be the most important one. In the gamut A C D E G, you could choose to make E the root note. Think of the root as the most important note in the scale; it's just as if one note in the gamut were put in charge of the others. 

One way to make the E note stand out is to have it playing in the background, so try playing these notes while letting your bass E string ring over the top. Improvise like this for a while. The E in the gamut, because it's the same note as the bass string, will sound more stable, and you may find your improvised phrases often seem to `want' to begin or end on it. This is what root notes do, but they do more than this: you should find that the C sounds particularly \emph{un}stable, for example. Try playing a phrase and ending on that C, with the E string ringing out (you'll have to hit it from time to time to sustain it); don't you now want to hear another phrase that ends somewhere else? Doesn't the C sound like a note of tension that needs a resolution? The root note determines the way each note sounds, and it's the presence of a root that turns a mere gamut into a scale. Put another way, \emph{fixing the root gives musical meaning to every note}.

\section*{Key Centres}

Another way of looking at a scale is as a sequence of intervals, which we will call an `interval map'. The interval map of a scale is simply a list of the distances (intervals) between its adjacent notes, ordered in sequence starting with the root. Intervals maps will be absolutely central to most of the theoretical content of this book, including the rest of this chapter, so take this part slowly.

Here is the interval map of the scale A C D E G with C chosen to be the root, using the abbreviations given in the table above. It's written starting with the C (that's conventional: start with the root) and going up steadily in pitch in the smallest intervals possible:
\begin{quote}
\begin{tabular}{rrrrrrrrrrrrrrr}
	C &&D &&E &&G &&A &&(C)\\
	 &t && t&&mT &&t &&mT &
\end{tabular}
\end{quote}
This interval map contains two different kinds of interval -- some tones (two-fret intervals) and some minor thirds (three-fret intervals). The sequence of an interval map is important. Here is a scale that contains three tones and two minor thirds just like the C D E G A scale does:
\begin{quote}
\begin{tabular}{rrrrrrrrrrrrrrr}
	C && D$\sharp$ &&F$\sharp$ &&G$\sharp$ && A$\sharp$ &&(C)\\
	 & mT &&mT && t && t&& t &
\end{tabular}
\end{quote}
but it sounds quite different (try it). We'd like to make this definition: two scales that have the same interval map are in fact the same scale. 

The trouble is that this doesn't agree with the definition given earlier. There we stated that a scale is a gamut with a root note, but consider these two gamuts with root notes:
\begin{quote}
\begin{tabular}{rrrrrrrrrrrrrrr}
	C &&D &&E &&G &&B$\flat$ &&(C)\\
	 &t && t&&mT &&mT &&t &
\end{tabular}
\end{quote}
\begin{quote}
\begin{tabular}{rrrrrrrrrrrrrrr}
	D &&F &&F$\sharp$ &&A &&C &&(D)\\
	 &t && t&&mT &&mT &&t &
\end{tabular}
\end{quote}
They are clearly different gamuts -- the first contains a B$\flat$, for example, that the second doesn't -- and they have different root notes (C and D). By the original definition, then, they're definitely different scales. Yet their interval maps are the same. So are they the same scale or not? Well, we say they're the same scale but \emph{in different keys}. When we say that a particular scale is played `in the key of C', all we mean in this book is that the pitch-name of the root note is C. Based on that and the interval map, you can work out the other notes in the gamut underlying the scale if you want to do so.

Scale patterns are really no different from movable chord shapes (bar chords, for example). A scale shape in a fretboard box looks like this -- don't worry what the different types of dot mean, we'll get to that:
%~C|5|2,b7|%
This isn't a fixed pattern of pitches, because you can play the pattern in different positions on the fingerboard, and as you move it up or down the pitches will change. Instead, it's a fixed pattern of intervals. If you know how to play some particular scale pattern with the root as C, you can always move it up two frets to play it `in the key of' D instead. The intervals stay the same, fixed by the pattern, so you don't have to think about them, and they guarantee that the other notes will automatically be correct.

Imagine we've called the scale C D E G A the `Common Pentatonic scale'. We say that this is the Common Pentatonic `in C', or sometimes `in the key of C'. A scale with the same interval map but a different set of pitches is the same scale in a different key; looking at the root note will tell you which key.

A warning note is in order here: the key system is a lot more than just fixing root notes. Just because a song is `in the key of C' does \emph{not} mean that all of the scales you play on it will have C as their roots. This is a more or less completely different meaning of the term `in the key of C'. All that we mean \emph{in this book} when we say that a scale is `in C' is that the pitch of its root note is C. When people talk about a whole piece of music being in a particular key, or about it changing key, they mean something that we can't get into here. Just think of the idea of a scale being `in the key of C' and a song being `in the key of C' as two separate ideas and you'll be fine.

To sum up all our definitions, we say that two scales are exactly the same if (and only if) they have the same gamuts and the same root notes, but that two scales are the same apart from their keys if they have the same interval maps but different root notes. 

We will mostly be interested in the second of these precisely because changing the key of a scale is as simple as moving a pattern on the fingerboard. We will not, therefore, be much concerned with which specific pitches are in the scale (the gamut), but we'll be extremely interested in the interval map because that, in the form of the fingerboard pattern, is something we can easily visualise and learn. We will therefore often say, as is normal practice, that two things are the same scale regardless of the key they're being played in as long as their interval maps are the same. After all, it's natural to say that the Common Pentatonic scale in C and the Common Pentatonic scale in D are examples of the same scale, that is, the Common Pentatonic scale\footnote{There is a philosophical dimension to this to do with instantiations of universals that will probably be of interest only to those who understand this footnote; they'll be able to figure the rest out on their own.}.

Now, based on the definition just given, for any scale to be a Common Pentatonic scale it must have the same interval map as this one, which we might write as 
\begin{quote}
	t t mT t mT
\end{quote}
So how do you play a Common Pentatonic scale in D? Certainly not by playing D E G A C, which contains the same gamut as the C Major but has D as its root, because the interval map of that scale is
\begin{quote}
	t mT t mT t
\end{quote}
which is not at all the same. Leaving aside the fact that we know these must be related \emph{somehow} because they contain all the same pitches, we can see from the definition of a scale that these two cannot be the same because their interval maps don't match. 

Instead, fix the root as D and follow the interval map, which becomes a kind of recipe for building the scale:
\begin{quote}
\begin{tabular}{rrrrrrrrrrrrrrr}
	D &&E &&F$\sharp$ &&A &&B &&(D)\\
	 &t && t&&mT &&t &&mT &
\end{tabular}
\end{quote}
It's worth taking a moment to write down the pitches in the G Major Pentatonic by following the interval map provided, to be sure you understand what's going on (the answer is in this footnote\footnote{The G Major Pentatonic contains the notes G A B D E.}).

\section*{Modes}

Modes are very, very simple. Guitarists get confused by them because they seem to be different from scales. They are not; modes are scales. The two words `scale' and `mode' simply have different functions: a scale is just a scale, but a mode is always a mode \emph{of} something. This is the way we use the word `mode' in other areas of life: cars and trains are both `modes of transport', for example, but you don't generally find `modes' floating around on their own.

Take the example we've been working with, the gamut A C D E G. Make one note the root -- say E, as we did before -- and we have a scale. Now, if we make C the root instead, we have a totally different scale. If you can\footnote{This is a bit awkward. If you find it too difficult, either drop your low E string down to a C or record yourself playing the note C for a while and then play over the top of it. Alternatively, you could ask a patient friend to play the note for you instead, should you be fortunate enough to have one.}, play a C on one of the bass strings while improvising with the gamut, as you did when sustaining the E. You should notice that, even if you're playing the same phrases as before, the note C now sounds like the point of stability on which it's nice to end a decisive phrase, while the D, which before was fairly indifferent-sounding, is now quite a strong note of tension. Now a phrase can `resolve' by moving from D to C, when this increased the tension when the root was E. Yet these two scales, which are clearly very different not only in theory (because they have different interval maps) but also in sound are obviously connected. They are, in fact, modes of each other.

Most people stop there when they're explaining modes to guitarists, assuming they even get that far. It can leave you feeling a bit confused, though, because it doesn't really explain what's going on. Here are the interval maps of the two scales, written out in a way that should make their relationship clearer:
\begin{quote}
\begin{tabular}{rrrrrrrrrrrrrrrrr}
	C &&D &&E &&G &&A &&(C)&& \\
		&t && t&&mT &&t &&mT & &&\\
	&&D &$\Updownarrow$&E &$\Updownarrow$&G &$\Updownarrow$&A &$\Updownarrow$&C && (D) \\
		&  && t &&mT &&t &&mT && t 
\end{tabular}
\end{quote}
It's obvious that their interval maps are closely related. In fact, if you consider the interval maps to `wrap around' in a circle, you can imagine the second interval map to be just the same as the first, but starting on a different interval. This starting-point is, of course, the result of choosing a different root note. This relationship is precisely what's meant when we say that one scale is a mode of another\footnote{There is a branch of mathematics called `group theory' that formalises this characterisation of modality. With patience, anyone can learn the small amount of group theory required to understand the behaviour of the twelve-note pitch class system, but it is quite unnecessary unless your ambition is to be a music theorist.}. We care about which scales are modes of each other only because their fingerings are going to be more or less the same; this makes them easier to learn. That will become clear when you study your first group of modes and see how the fingerings are shared.

\section*{The Numbering System}

Because of the way a scale's notes change when we change the root note, it's not helpful to write down a scale in letters as we have thus far with, say, the scale `C D E G A'. Also, because guitarists don't think in intervals most of the time, it's no good for us to just write down the interval map either. We need some other way to write out a scale that is really practical.

To do this it helps to have some kind of baseline in comparison with which other musical material can be described precisely and helpfully. Any scale could have been chosen, but the Major scale is important throughout Western tonal music and has a very familiar sound, so it would be the obvious one to choose even if it wasn't already the standard\footnote{The numbering system described here evolved along with the harmonic and melodic system that is underpinned by the Major scale, so nobody ever really `chose' the Major scale for this role.}. In C, the Major scale contains the following notes:
\begin{quote}
\begin{tabular}{rrrrrrrr}
C	&D	&E	&F	&G	&A	&B	&(C)
\end{tabular}
\end{quote}
We begin by numbering each note in the Major scale in C, like this:
\begin{quote}
\begin{tabular}{rrrrrrrr}
C	&D	&E	&F	&G	&A	&B	&(C)\\
1	&2	&3	&4	&5	&6	&7	&(1)
\end{tabular}
\end{quote}
Now we state that the numbers will stay the same whichever key you're in; here, for example, is the Major scale in D:
\begin{quote}
\begin{tabular}{rrrrrrrr}
D	&E	&F$\sharp$	&G	&A	&B	&C$\sharp$	&(D) \\
1	&2	&3	&4	&5	&6	&7	&(1)
\end{tabular}
\end{quote}
You could think of it this way: 1, the root, is the one you start on when you're building the scale. Then, if you follow the interval map for the Major scale, 2 is the next note you hit. If 1 is D, 2 will be E. If 1 is C, 2 will be D. You may want to pause at this point to write out the notes of the Major scales in G and in A$\flat$, following the interval map (which you will have to work out), and then check them in this footnote\footnote{G Major contains G A B C D E F$\sharp$ and A$\flat$ Major contains A$\flat$ B$\flat$ C D$\flat$ E$\flat$ F G. Don't worry if your versions contained enharmonic names for the same notes; so it's fine if you wrote A$\flat$ Major as A$\flat$ A$\sharp$ C D D$\sharp$ F G or something similar, as long as the pitch classes are equivalent.}.

Now, consider a scale called the Dorian scale, which in D contains the notes D E F G A B C. Comparing this to the D Major scale, we see that the third and seventh notes have been lowered by a semitone. There's a simple and obvious way of notating this:
\begin{quote}
\begin{tabular}{rrrrrrrr}
D	&E	&F	&G	&A	&B	&C	&(D) \\
1	&2	&$\flat$3	&4	&5	&6	&$\flat$7	&(1)
\end{tabular}
\end{quote}
Here the `$\flat$' (the flat sign) means `lower the note by a semitone'. It doesn't indicate that the pitch class played will be a flat note; the 7 of D Major is C$\sharp$, and C$\sharp$ lowered by a semitone is C, not C$\flat$, so the $\flat$7 of D is C, not C$\flat$. Make sure you understand this distinction. 

You can clearly see from the numbered version that a Dorian scale is the same as a Major scale with its third and seventh notes lowered by one fret (semitone). If you work out its interval map, you will see that the Dorian scale is a mode of the Major scale; this is a worthwhile way to check your understanding and should only take a few minutes. Note that it is just as correct to say that the Major scale is a mode of the Dorian, and that we don't refer to `the Dorian mode' but to `the Dorian scale' and `the Major scale', two scales that are modes of one another. You will hear terminology like `the Dorian mode' used all the time; it is one of the reasons why guitarists find modes confusing. It's best to think of it as shorthand for `the Dorian scale, which is a mode of the Major scale', but put that way it sounds like a foolish kind of shorthand because it hides important and useful information.

In many scales, a note will need to be raised instead of lowered by a semitone; in this case, a `$\sharp$' (the sharp sign) is used. Many of the more exotic scales have numbers that are $\flat\flat$ (double-flat) or $\sharp\sharp$ (double-sharp). The meaning is pretty obvious: lower or raise the note by two semitones rather than just one. There are even a few that are triple-sharp or triple-flat, again with the  obvious meaning: raise or lower by three semitones. From the numbered form of a scale you can, if you wish, work out which notes are in it in any particular key, but the fact is that you won't need to; as long as you know the fingering pattern, you can play the scale in any key automatically. The next chapter provides the practical framework you will need to enable you do to that.

\section*{Hypermodes}

This final concept is more advanced (and obscure) than the others in this chapter and if it's in any way confusing you should just skip it for now. The idea is that you can extend the concept of `mode' to include gamuts without root notes, sometimes with useful or interesting results. We begin by thinking of the modes in a different way, and we have to be \emph{very} careful here. Why? Because this is precisely the way most introductions to modes explain things, and it's extremely unhelpful most of the time. So please be sure you understand everything in this chapter before reading on.

We can think of the D Dorian scale as a C Major scale with D chosen to be the root instead of C, and in general we can think of the Dorian scale in any key as the Major scale `built on' the $\flat$7 (i.e. the Major scale whose `root' is the $\flat$7 in relation to our actual root note). This is not the way we want to think about it in practice because it's tremendously confusing, but in theory it's not wrong to do so. Well, we can then find all the other notes that give us modes of the Major scale, and there will be seven in total, one for each note in the Major scale itself. But we know there are twelve notes in music altogether -- what about the other five?

These form what I call `hypermodes'. They are not scales, because they have no root notes, but a root note will probably be implied by the harmony or other aspects of context. They are rarely heard, but there are some occasions when it's useful to be able to talk about them without the worry that they are `not really scales' and so are somehow `not legitimate'. One example is when jazz musicians superimpose arpeggios on chords; another is when they `side-slip' by playing a scale in different key from the one they would normally use, often a semitone up or down, in order to get a simple `outside' sound.

In this book we only look at hypermodes of scales with up to four notes because the hypermode concept seems to become less useful the more notes there are, and because considering hypermodes of all scales would considerbly bulk it out with material only a few readers would get anything out of. We will look at hypermodal applications of the triad and seventh arpeggios shortly, mostly to provide interesting material for the intermediate student who wants to learn these thoroughly before moving on to larger structures.




\chapter{\mbox{The Triad} \mbox{Arpeggios}:\titlebreak \mbox{A Framework} \mbox{for Scales}}

The idea behind this chapter is that the four standard triad arpeggios give you a way to `map out' the whole fingerboard, providing a framework or outline that can be filled in by the notes of a scale. If you know this framework well then learning the additional notes that form scales is made a lot easier. Arpeggios themselves have many interesting musical applications, so committing this framework to memory needn't be a chore; it's quite possible to play triad and seventh\footnote{Seventh arpeggios are covered in the next chapter.} arpeggios for years without getting bored, and to make exciting music using only this `framework'. All students should know the principles explained in this chapter very thoroughly, and the next chapter provides some applications that will help to make learning the triads more interesting. The major and minor triads are essential for all musicians, and it's strongly recommended that the diminished and augmented triads be learned as well, although not necessarily right away.

An arpeggio is really just a collection of notes with a root note or, in other words, a scale. All arpeggios are scales, but not all scales are arpeggios. We say that a scale is an arpeggio if its notes, played together, form a chord, and we name it after that chord. For example, the Major Pentatonic scale, 1 2 3 5 6, is referred to just as correctly as a `major 6/9 arpeggio', because playing those five notes together gives a chord called the major 6/9. 

Of course, the notes of any scale can be played together to form a chord, especially if you have a taste for dissonance. Even the twelve-note `chromatic scale' has been used chordally to form `clusters' by composers like Penderecki and Cowell\footnote{Every musician should check out Penderecki's \emph{Threnody to the Victims of Hiroshima} and Henry Cowell's solo piano music, the first for its raw power and the second for Cowell's inventive and irreverent approach to the instrument. As both composers knew, theory should be a tool, not straitjacket.}. In this book, we only consider a scale to be an arpeggio if it has a maximum of four notes. The Major Pentatonic mentioned in the last chapter isn't an arpeggio because it contains five notes. This is a bit of an arbitrary rule, and it's not really how the word `arpeggio' is used by real people out there in the world, but it makes sense for our purposes.

\section{The CAGED Arpeggio System}

Any scale -- including any arpeggio -- appears on the guitar as a bunch of notes scattered all over the whole fingerboard. How should you go about learning something like this? Perhaps one string at a time? Or one fret at a time? Or maybe you should learn the locations of all the root notes, then all the seconds, thirds, fourths and so on? How do you cope with the fact that the notes all move around when the pitch of the root changes?

For almost\footnote{Positions only really make sense with instruments that (a) have more than one string and (b) are stopped by the fingers (as opposed to harp-like instruments).} as long as they've existed, stringed instruments have been learned in what are called `positions', each position consisting of all of the notes that can be reached without moving the index finger up or down the neck, or at least without moving it more than a semitone (one fret) either way from where it starts. Because each position requires only a minimum of hand movement, it helps promotes efficiency.

It also turns out that position-based fingerings provide an easier way to learn a scale or arpeggio than, say, learning one whole string at a time. The reason for this is that a position makes a shape that you can visualise, and that in turn makes it easier to remember. The importance of this can't be overemphasised; if you want to learn a large number of chords, arpeggios, scales or anything else on the guitar, you must be prepared to treat them as pictures rather than as a set of numbers or -- worse still -- of pitch-names. 

Readers who have the opportunity to use a pen and paper at times when they can't play the guitar (while using public transport, say) are even encouraged to practice drawing these fingerboard diagrams from memory; it might appear to be a pointless task, but the fixing of these images in the mind is one of the greatest difficulties the beginning player faces, and some of the work of memorisation can certainly be done away from the instrument. You should then get into the habit of visualising these pictures mentally, without the aid of pen and paper. It sounds like nonsense but it's a very effective way to learn.

Here's an example of one of these diagrams, which happens to represent something called a `major triad arpeggio':
%~C|5|%
Vertical lines represent strings and horizontal lines are frets. The black dots represent the notes you play, and in particular the circled dots indicate root notes. If you want to use this pattern to play a major triad arpeggio in the key of E you need to position it so that one of those circled notes is the pitch `E'. Once you've positioned one root note, of course, all of the other notes will fall into their correct places. That's what the diagram is designed to do: position one root and the rest of the notes take care of themselves. This particular pattern is known colloquially as the `C'-shape pattern because, if you play it in open position (that is, with the notes on the lowest frets being open strings) you will pick out the notes of the well-known open C major chord that you learned very early on. The difference is that now you're picking single notes, so it makes sense to have more than one note on each string. 

The pattern just given covers only four frets and, as you know, there are twelve frets on the guitar before patterns start to repeat\footnote{Not counting symmetrical patterns; details on those follow.}. If you only knew this shape you'd only be able to play the arpeggio in one part of the fingerboard for a given root note. Imagine you wanted to play an E major triad arpeggio, but whatever you're currently playing is up around the $10^{\text{th}}$ fret. You'd have to move your hand all the way down to the fourth fret to play the E major triad using this pattern. That kind of shifting around takes time and breaks up your playing. 

To achieve real fluency with any arpeggio or scale you need to be able to find the pattern for it in any position for any root note. That sounds like a tall order, but it can be achieved with just four more patterns like the one above. Here's the second such pattern, colloquially known as the `A'-shape:
%~A|5|%
If you can find an `A'-shape barre chord then you can find the arpeggio above just as easily. The same goes for the other three major triad arpeggios:
\[
\begin{array}{ccc}
	%~G,NoDisplay|5|%   &    %~E,NoDisplay|5|%   &    %~D,NoDisplay|5|%   \\
	\text{`G'-shape} & \text{`E'-shape} & \text{`D'-shape}
\end{array}
\]
and these five patterns form the basis of the so-called `CAGED' system. Each one provides a position and when you put them together they cover the whole fingerboard:\\
\begin{tabular}{rr}
%~C,NoDisplay|16|% 
& 
\parbox[b]{10cm}{
This diagram covers sixteen frets. It starts at the top with the `C'-shape, whose bottom notes are the top notes of the `A'-shape, which is immediately below it. Next come the `G', `E' and `D' shapes, and finally the `C' shape again, an octave higher. Note that the `D'-shape is really redundant; it contains no notes except for those in the `E'- and `C'-shapes it overlaps. It is, though, useful to have this extra coverage. It's recommended that scales should be learned using the first four shapes first, with the `D'-shape just being used to consolidate.
\rule{0pt}{6ex} 
}
\end{tabular}

You might not yet know movable chords that fit all of these shapes; that isn't important, although if you do then it will certainly help you to memorise them. To learn them, start with one shape -- it doesn't matter which -- and learn that. Just play around with it and try to make some music. Then learn the pattern immediately above or below it and start sliding your hand between the positions as you play. Again, play around like this for a while. By repeating this once more (both up and down), you'll have learned all five shapes and so will have covered the whole fingerboard. It takes some time -- don't rush it, you're building the foundation of everything else in this book.

The major triad's formula is 1 3 5; in the key of C major, for example, its notes are C (the root), E (the third) and G (the fifth). As you will know, the major triad is not the only one; there are also minor triads and -- though less often-encountered -- diminished and augmented ones as well.

Every musician needs to know the CAGED shapes for the minor triad; minor chords are as fundamental a part of our music as major ones. The minor triad's formula is 1 $\flat 3$ 5, and hence only differs in the third being a semitone lower. In the key of C, the minor arpeggio is C (the root), E$\flat$ (the flat third) and G (the fifth). It's very instructive to compare the arpeggio shapes for the minor triad with those of the major because it's important to learn which of the black notes is the third and which the fifth.
\[
\begin{array}{ccccc}
	%~Cm,NoDisplay|5|%   &%~Am,NoDisplay|5|%   &%~Gm,NoDisplay|5|%   &    %~Em,NoDisplay|5|%   &    %~Dm,NoDisplay|5|%   \\
	\text{`C'-shape} & \text{`A'-shape} & \text{`G'-shape} & \text{`E'-shape} & \text{`D'-shape}
\end{array}
\]
\begin{tabular}{rr}
%~Cm,NoDisplay|16|% 
& 
\parbox[b]{10cm}{
One very fortunate thing about this is that the tension between the minor and major third, when the harmony is a major chord, is a typical sound in both jazz and blues. It's therefore a pleasure to learn to ornament major arpeggios by adding the contrasting minor third. This makes learning the minor arpeggio shapes much less of a chore than it might be otherwise. Just by knowing this trick, and of course by knowing the arpeggio shapes very well, you can get along pretty nicely playing 12-bar blues, a bit of rock and even simple jazz, as long as not too much variety is called for. 
}
\end{tabular}

The diminished and augmented triads are, at first, more troublesome and many musicians don't know them, or at least not well. This is a shame, since both are very useful. You've just seen that playing the minor triad over a major chord gives a `blue note', the flat third, which clashes in a good way with the natural third in the underlying harmony. The diminished triad extends this idea by providing a flat fifth, another important `blue note'. Its formula is 1 $\flat 3$ $\flat 5$, so in C major it would be C (the root), E$\flat$ (the flat third) and G$\flat$ (the flat fifth).
\[
\begin{array}{ccccc}
	%~Cd,NoDisplay|5|%   &%~Ad,NoDisplay|5|%   &%~Gd,NoDisplay|5|%   &    %~Ed,NoDisplay|5|%   &    %~Dd,NoDisplay|5|%   \\
	\text{`C'-shape} & \text{`A'-shape} & \text{`G'-shape} & \text{`E'-shape} & \text{`D'-shape}
\end{array}
\]

\begin{tabular}{rr}
%~Cd,NoDisplay|16|% 
& 
\parbox[b]{10cm}{
Playing with the tension between flat and natural fifth expands your ability to play blues and jazz lines, and also provides a more interesting way to play over minor chords. If you have the patience it's worth learning this arpeggio, although it isn't essential to do so immediately.

\rule{0pt}{15ex} 
}
\end{tabular}

The augmented triad is the least-loved of all; it has a weird, lumpy sound when you first hear it. Its formula is 1 3 $\sharp 5$, so in C the pitches are C (the root), E (the third) and G$\sharp$ (the sharp fifth).
\[
\begin{array}{ccccc}
	%~Ca,NoDisplay|5|%   &%~Aa,NoDisplay|5|%   &%~Ga,NoDisplay|5|%   &    %~Ea,NoDisplay|5|%   &    %~Da,NoDisplay|5|%   \\
	\text{`C'-shape} & \text{`A'-shape} & \text{`G'-shape} & \text{`E'-shape} & \text{`D'-shape}
\end{array}
\]
The augmented triad is a common sound in modern jazz, and you'll see some applications of and variations on it over the following pages.

There's no harm in waiting a while before learning the augmented triad shapes, but when you do you might be surprised by how similar they are, which is made even more striking when you look at them joined up along the whole fingerboard:\\
\begin{tabular}{rr}
%~Ca,NoDisplay|16|% 
& 
\parbox[b]{10cm}{
It's worth considering why this happens. The interval map of the arpeggio is MT, MT, MT; the distance between the root and third, between the third and sharp fifth and between the sharp fifth and the root are all equal. For example, the arpeggio of E augmented contains E (the root), G$\sharp$ (the third) and B$\sharp$ (the sharp fifth). Now, B$\sharp$ is just C, so C augmented and E augmented contain exactly the same pitches. If you work out the pitches for G$\sharp$ augmented you'll see that the same is true of that, too.
}
\end{tabular}\\
This arpeggio is our first example of a `symmetrical' structure; we'll see all the other symmetrical scales in a later chapter where you'll also find a fuller explanation of what's going on. We also discuss some musical applications in the next chapter, so don't worry if it sounds weird to you at the moment.


\section{Playing Through Changes}

Many guitarists always play their solos over just one static harmony, or over a group of chords that allows them to vary the scales or arpeggios they use quite freely but doesn't force them to change at particular times. The 12-bar blues and many other progressions popular in rock and folk songs are examples of this type of music. There is nothing wrong with it; many a fine musician has never played through changes, and would not have much idea how to.

In other forms of music, however, playing a solo that follows the harmony of a song is necessary because sticking to a single scale, although fine in some parts, will produce `wrong notes' in others\footnote{There really is no such thing as a `wrong note', but we have all had the unpleasant experience of playing a note that sounds nothing like we expected. The author had a friend in college who claimed that, when this happened, the thing to do was to play the note again, as if you meant it. There's something in this, but not a great deal. More convincingly, it's said that when the great jazz pianist Thelonious Monk was rehearsing and a band-member played a wrong note he would stop the band and ask, `What were the implications of that?' -- that is, he'd explore what it would mean if the `wrong' note had actually been deliberate. Anyone who has listened to his music will very likely believe the story.}. Some guitarists, impressed by the existence of modes, waste time trying to find a single scale that `fits' over a whole section of a song of this sort, but the results are usually pretty lifeless; the point of this kind of music is to let the harmony shape your solo and to move with it, not to try to ignore it for as long as possible.

Instead, guitarists who want to play through changes are encouraged to first learn to play the chords using movable chord shapes in a single position (or as close to that as the changes allow), and then to progress to playing the arpeggios in the same positions. Having learned the progression this way, one can move quite smoothly from strumming chords to playing triad arpeggios in a solo. It might not be a very interesting solo, but at least the notes won't be wrong, and you will be clearly following the harmony of the tune. You can then play scales that you'll have learned within the framework of these arpeggios; you'll see how this works shortly. Learning to play the song in this way in all five positions enables you to improvise while moving all over the fingerboard. It takes time but isn't difficult, or at least it's only as difficult as the complexity and speed of the tune. Making great music over complicated harmonies is very hard indeed, but so is making great music over simple ones. 





\chapter{\mbox{Applications} \mbox{of} \mbox{the} \titlebreak \mbox{Triad} \mbox{and} \mbox{Seventh} \titlebreak \mbox{Arpeggios}}

The triad arpeggios described in the previous chapter form the foundation of most single-note lead guitar vocabulary. As a result, it's really worth not just looking at them but developing a deep familiarity with these fingerings. The problem is that playing triad arpeggios soon gets tiresome because on their own they don't sound very interesting.

A simple and quick way to make the study of arpeggios pay off has already been mentioned -- the  use of minor or diminished sounds over major chords, providing the famous `blue notes' of the $\flat$3 and $\flat$5. This chapter provides some additional resources that guitarists usually learn only at a more advanced stage, but that can be introduced earlier in order to make the complete absorption of arpeggios more palatable and hence to delay the study of scales a little. The better the arpeggios are known, the easier it will be to learn scales without them becoming a straitjacket. An added advantage is that many of them use the augmented and diminished triads, with which most guitarists are less familiar. 

It's not necessary to work beyond the first section of this chapter (covering seventh arpeggios), and you can move on to the next chapter immediately after that section if desired. Some of the subjects covered in this chapter are normally considered `advanced' and are presented here as optional material that will help to consolidate knowledge of the triad and seventh arpeggios.

This chapter just contains a few examples: players who enjoy them will find many more in dedicated chord books, especially those designed for jazz guitarists, and the ideas in them are applicable to other musical styles if you use your imagination and your ears. Good books for this are Ted Greene's \emph{Chord Chemtistry} and Steve Khan's \emph{Contemporary Chord Khancepts}, both of which are designed for guitarists: general (theory-based) books about harmony and books for pianists can be great sources of inspiration too.

\section{Adding the Seventh}

As you probably know, most chords are based on one of the four triads, but some have notes added to them to make them more interesting, especially the seventh. Not only that; the kind of seventh is determined, to some extent, by the function of the chord in the music. Playing B$\Delta$ on the ninth bar of a 12-bar blues in E gives an immediate and rather unpleasant\footnote{As with `wrong' notes, there is no such thing as an `unpleasant' sound as such, but many sounds will be unpleasant \emph{to you}, and this may very well be one of them.} indication of that: try it once you know how to do this, which you will shortly, and hear it for yourself.

There are three kinds of seventh that can be added to a triad, but only two are `real sevenths'. One is the natural or `major' seventh, which is one semitone below the root. Adding it to the major triad creates the major seventh chord (or arpeggio), which for reasons passing understanding is generally abbreviated using the symbol `$\Delta$':
\[
\begin{array}{ccccc}
	%~C,NoDisplay|5|7%   
	&%~A,NoDisplay|5|7%   
	&%~G,NoDisplay|5|7%
	&%~E,NoDisplay|5|7%
	&%~D,NoDisplay|5|7%
\end{array}
\]
This arpeggio works over any major triad that has a `tonic' or `subdominant' function; if you don't know what that means, let your ears be the guide. Major sevenths are not usually added to the other triads, although they can be added to the minor triad to create the `m$\Delta$' chord; that's uncommon (though hardly unknown) in most forms of music although it does appear in jazz tunes, but as an arpeggio it has plenty of applications.

The second type of seventh is the $\flat$7, which is a whole tone (two semitones) below the root:
\[
\begin{array}{ccccc}
	%~C,NoDisplay|5|b7%   
	&%~A,NoDisplay|5|b7%   
	&%~G,NoDisplay|5|b7%
	&%~E,NoDisplay|5|b7%
	&%~D,NoDisplay|5|b7%
\end{array}
\]
The $\flat7$ is commonly added to major triads where the chord has a `dominant function'\footnote{If you don't know what this means, ignore it for now -- it's to do with how the chord fits in with the rest of the chords in the song.}, or to almost any major triad in some blues and rock contexts. It can also be added to the minor triad in just about any situation:
\[
\begin{array}{ccccc}
	%~Cm,NoDisplay|5|b7%   
	&%~Am,NoDisplay|5|b7%   
	&%~Gm,NoDisplay|5|b7%
	&%~Em,NoDisplay|5|b7%
	&%~Dm,NoDisplay|5|b7%
\end{array}
\]
You should certainly know the major, dominant and minor seventh arpeggios; since you already know the triads, learning them requires only that you learn the position of the seventh in relation to the root, which doesn't take much effort.

If you play jazz then you'll also benefit from knowing the `half-diminished' arpeggio, which is formed by adding the $\flat 7$ to a diminished triad:
\[
\begin{array}{ccccc}
	%~Cd,NoDisplay|5|b7%   
	&%~Ad,NoDisplay|5|b7%   
	&%~Gd,NoDisplay|5|b7%
	&%~Ed,NoDisplay|5|b7%
	&%~Dd,NoDisplay|5|b7%
\end{array}
\]
Half-diminished chords come up frequently in jazz tunes and the arpeggio has some nice applications, some of which are touched on in this chapter. This arpeggio can also, of course, be used whenever you would use a diminished triad over a dominant seventh chord.

The half-diminished arpeggio contrasts with the diminished arpeggio itself, which is formed by adding a $\flat\flat 7$ (which is the same as the 6) to the diminished triad:
\[
\begin{array}{ccccc}
	%~Cd,NoDisplay|5|6%   
	&%~Ad,NoDisplay|5|6%   
	&%~Gd,NoDisplay|5|6%
	&%~Ed,NoDisplay|5|6%
	&%~Dd,NoDisplay|5|6%
\end{array}
\]
The diminished arpeggio is, like the augmented triad, symmetrical; its interval map consists of four minor thirds. This is why the patterns for this arpeggio are more regular than the rest, and as with the augmented arpeggio you'll probably find it quite easy to learn. This, too, can be used wherever you might use a diminished triad over a dominant seventh chord, although it is more dissonant than the half-diminished.

As a stepping-stone to the study of scales, learning these five arpeggios is an achievable goal that will immediately enrich your playing. It also provides a solid framework for learning and applying scalar ideas. With the applications given in the rest of this chapter, this material alone could provide a year's practice and study and will pay immediate dividends, but beginners might want to move onto the next chapter and look at the Common Pentatonic and Major scale groups first.

\section{A `Scale' of Dissonance}

Before we look at superimpositions, the subject of the rest of this chapter, it will help to have a general idea of what their results mean. Say a particular musical idea contains the notes $\flat$2, 3, $\sharp$5 and 6 and the underlying harmony is a dominant seventh chord. Will this sound consonant or dissonant? In other words, is it a good pattern to use to resolve a phrase, or will it introduce tension?

There is no scientific way to describe dissonance, and it is entirely defined by your style of music, but the following table gives a general idea of a likely ordering of the dissonance of various notes over various harmonies. The style referred to here is jazz of a reasonably mainstream kind, but even then different personal styles would cause variations in the ordering and the context of a particular piece of music can have a huge impact as well. If you play in another musical style or tradition then you'll most likely want to evolve your own version, which might look very different.

Those notes near the top of the list are consonant and are safe notes to hold, or to end a phrase on. As you go further down the list, the notes become at the same time more interesting and also more risky.
\begin{quote}
	\begin{tabular}{llllll}
		$\Delta$	&	Dominant 7	&	Minor 7	&	Half-diminished	&	Diminished	&	Augmented \\
		1  &	 1  &	 1  &	 1  &	 1  &	 1  \\
		3, 5&3, 5 &$\flat$3, 5 & $\flat$3 $\flat$5 & $\flat$3 $\flat$5 & 3, $\sharp$5 \\
		7 & $\flat$7 & $\flat$7 & $\flat$7  & 6 & $\flat$7, 7\\
		2, 6, $\sharp$4 & 2, 6, $\sharp$4, $\flat 3$ & 2, 4, 6 & 2, 6, 4 & 2, 4 & 2, 6, 4 \\
		4 & 4 & $\flat$2, $\sharp$4 & $\flat$2 & $\flat$2 & $\sharp$4, $\flat$2, $\flat$3 \\
		$\flat$6,$\flat$3  & $\flat$6, $\flat$2 &$\flat$6 &  $\flat$6 & $\flat$6, 7 & \\
		$\flat$7, $\flat$2 & 7 & 7, 3 & 7, 3, 5 & $\flat$7, 3, 5 & 7, 5
	\end{tabular}
\end{quote}
Eventually you should aim to develop your own version of this table, but do it organically, over time, rather than writing something out and memorising it.

Going back to the question with which we began, $\flat$2, 3, $\sharp$5 and 6 over a dominant seventh will be quite dissonant because of the $\flat$2 and $\sharp$5 (=$\flat$6). When playing a phrase based on these notes, you might want to begin and end on either the 3 or 6, which are more consonant, using the other two notes to create tension. This way of thinking becomes internalised and hence `natural' after a while, but it helps to have a sketchy roadmap like this from the beginning.

\section{Simple Superimpositions}

The applications described in the remainder of this chapter are all based on one simple concept: the idea of playing an arpeggio with a different root note to the chord you're playing over. So, if you're playing on the first bar of a blues in E, your underlying chord is E7. You could attack this by playing a dominant 7 arpeggio whose root is on the root of the chord -- that is, E dominant 7 itself -- but that wouldn't be very interesting. On the other hand you could, as we shall see, play the dominant 7 arpeggio whose root is on the $\flat$3 of the chord: G dominant 7. The point is not to think about pitch-names, because these vary as the chords change, but to visualise arpeggio patterns laid out on the fingerboard. 

The idea of playing an arpeggio built on some note other than the root of the underlying chord goes back to nineteenth century classical music, where it was used to build up thick, complex harmonies and create doubt as to which key the music was in. Such tensions enabled composers such as Mahler to compose symphonic movements of previously unheard-of length; the need to resolve to the tonic, modulate to another key and then modulate back kept Baroque and Classical movements relatively short, but when the key-centre is ambiguous there's much more room for exploration. Many things in the history of jazz replayed the development of the classical tradition, but in fast-forward, and this innovation was no exception. Charlie Parker is perhaps the easiest example to point to; many Parker transcriptions find him playing arpeggios built on the second or seventh (which was known as `playing off the top of the chord') or other notes. The $\flat$3 superimposition -- G dominant played over E dominant, say -- is an example of this.

From the root, you can find the $\flat$3 by visualising the minor triad arpeggio in your current position. Here's the major triad with the $\flat3$ marked `$\times$':
%~E|5|b3x% 
Now take that note to be the root and play a dominant seventh arpeggio built on it (here the whole of the superimposed arpeggio is marked `$\times$'):
%~E|5|b2x,b3x,5x,b7x% 
This requires a bit of thinking at first. You need to hold in your head the arpeggio of the underlying chord (in this case, E dominant 7) and visualise the new arpeggio lying `over the top' of it on the fingerboard. It does become easy with practice, and it is an excellent training for the mental agility needed to play through tricky chord changes, where you will constantly have to find the next arpeggio shape to build your line around.

There is a close relationship between the idea of arpeggio superimpositions and the idea of modes of scales that might be of interest to the curious (if you're not the curious type, skip to the next paragraph). Remember that an arpeggio is just a particular type of scale. Most four-note scales form groups of four modes, one generated by picking each note as the root. Take the dominant seventh arpeggio, 1 3 5 $\flat$7. In C major that would be C E G B$\flat$. Now let's make E the root instead, giving the notes E G B$\flat$ C, which are 1 $\flat$3 $\flat$5 $\flat$6 in the key of E. This scale, 1 $\flat$3 $\flat$5 $\flat$6, must be a mode of the scale 1 3 5 $\flat$7. But another way to think of this is as the superimposition of the $\flat$6 dominant seventh arpeggio (C7) over an E-rooted harmony. In fact, those superimpositions that contain the root of the underlying harmony are modes of the original arpeggio. Those that don't are not scales at all because they don't contain the root note. You can find a lot more examples of this sort of thing in Part 2 of this book.

We begin with some simple superimpositions to present the idea. In chord progressions, the augmented and diminished chords appear most often as `approach' chords; their purpose is usually to create tension that can then be resolved to a more consonant major or minor chord. Below are some example applications based on this fact. In each case, try playing a phrase using the original arpeggio, then a phrase using the superimpositions, which will create tension, and then resolve the tension by playing another phrase using the original arpeggio. 

Note that there are, in each case, certain notes that lead naturally from one to the other, for example by being a semitone apart. These can make nice, smooth transitions from one arpeggio to the other, as can any notes the two arpeggios share in common. When practicing these, be sure to stay in one position, or very close to it; learn the shape of the superimposed arpeggio as it fits in with the shape of the original. To assist with this, the diagrams show the main arpeggio using the usual black dots and the superimposed arpeggio using crosses (you will have to determine, based on your knowledge of the arpeggio shapes, which is the root note of the superimposed one). At first you should visualise the main arpeggio but play only the crosses; in time you'll find you can switch between the two and mix up the notes at will.

The augmented arpeggio can be used to approach a $\Delta$ chord from a semitone below:
%~5|5|b3x,5x,7x% 
Note that because the superimposition contains the natural seventh, this will not sound so good over a dominant seventh chord because the two different sevenths will clash. A similar superimposition does, however, work on any major chord ($\Delta$ or dominant) if the augmented arpeggio is a semitone above, rather than below, the main chord:
%~5|5|b2x,4x,6x% 
This approach from above also works with minor chords, including minor sevenths:
%~5m|5|b2x,4x,6x% 

Similarly, the diminished seventh arpeggio can be used either a semitone above or below the main chord. Again, from below it works best over a $\Delta$ chord:
%~5|5|2x,4x,b6x,7x% 
whereas from above it works over a dominant seventh, adding only the $\flat$2 to the original arpeggio
%~5|5|b2x,3x,5x,b7x% 
which is a very common jazz substitution\footnote{A chord \emph{substitution} replaces one chord with another. A \emph{superimposition} plays the two simultaneously. When the band is playing E major and you play a G major arpeggio over it, that's superimposition. If you're playing the chords and you choose to play G major to replace part or all of the E major harmony, that's substitution. As you can see, the two concepts are very closely related.}. Neither of these works well with minor chords.

\section{The Charlie Parker Cycle}

The idea of the so-called `Charlie Parker Cycle' is to play a sequence of dominant seventh arpeggios over a static dominant harmony such as you might find, for example, in the first four bars of a blues. The cycle moves away from the main harmony and becomes more dissonant, then sets up a resolution back to the main harmony again. There are three superimpositions in the cycle; you can play all three in order, change the order, or choose any one or two of them as you like. The results contain a number of classic bebop sounds: Step 2 of the cycle is known as the `tritone substitution' and is found in a great deal of jazz, while Steps 1 and 3 are also quite common substitutions. As far as I'm aware Charlie Parker never knowingly used `the Charlie Parker cycle', but the name is now traditional and we can think of it as a fond tribute to the great saxophonist.

The idea of the cycle is to start by imagining (but not playing) a diminished seventh arpeggio whose root is the root of the chord you're in, which will usually be a dominant chord. Now play a sequence of dominant seventh arpeggios such that the root of each is based on a note from the diminished seventh arpeggio you're imagining. In a basic blues in D, the first four bars are D dominant seventh, so the diminished arpeggio will be D diminished seventh. This arpeggio contains the notes D, F, A$\flat$ and B, so the principle is to play arpeggios of any or all of D7, F7, A$\flat$7 and B7 over the static D dominant harmony.

The point is not, however, to think in terms of pitch names; that defeats the object. If you know the diminished and dominant seventh arpeggios then you can easily find any element of this cycle simply by visualisation. If this is too difficult then you aren't yet familiar enough with the basic arpeggios; when you are, this idea will become much easier to put into practice. Note that although this is known as a `cycle', the arpeggios don't have to be played in the sequence given.

Leaving aside the first arpeggio, which is just the arpeggio of the underlying chord, we have:
\subsection*{Charlie Parker Cycle: Step 1 ($\flat$3 dominant)}
%~5|5|b2x,b3x,5x,b7x% 
\subsection*{Charlie Parker Cycle: Step 2 ($\flat$5 dominant)}
%~5|5|3x,b5x,b7x,b2x% 
\subsection*{Charlie Parker Cycle: Step 3 (6 dominant)}
%~5|5|1x,b3x,b5x,b6x% 



\section{The Coltrane Cycle}

The `Coltrane cycle' works on exactly the same principle as the `Charlie Parker cycle', but uses an augmented arpeggio instead of the diminished seventh and $\Delta$ arpeggios rather than dominants. Its name is no doubt a reference to Coltrane's famous compositions `Countdown' and `Giant Steps', in which the underlying harmony moves in major thirds. Again leaving aside the first arpeggio, we have two others in the cycle:
\subsection*{Coltrane Cycle: Step 1 (3$\Delta$)}
%~5|5|$\sharp$2x,3x,#5x,7x% 
\subsection*{Coltrane Cycle: Step 2 ($\sharp$5$\Delta$)}
%~5|5|1x,b3x,5x,#5x% 
This cycle is more dissonant that the Charlie Parker cycle, and requires more listening and experimentation before it becomes natural. This cycle can be used to play `Coltrane substitutions', a widely-used sound in modern jazz. The simplest possible version is over a sequence such as ii-V7-I (for example, Dm7-G7-C in the key of C major). The whole sequence is replaced by the Coltrane Cycle beginning and ending on the I:
\begin{quote}
\begin{tabular}{|p{3cm}|p{3cm}|p{3cm}|p{3cm}|}
	Dm7 &  G7 & C$\Delta$  & 
\end{tabular}
\end{quote}
becomes
\begin{quote}
\begin{tabular}{|p{3cm}|p{3cm}|p{3cm}|p{3cm}|}
	C$\Delta$ &  E$\Delta$ & G$\sharp$$\Delta$  & C$\Delta$
\end{tabular}
\end{quote}
or
\begin{quote}
\begin{tabular}{|p{3cm}|p{3cm}|p{3cm}|p{3cm}|}
	C$\Delta$ &  G$\sharp$$\Delta$ & E$\Delta$  & C$\Delta$
\end{tabular}
\end{quote}
There are many variations and different ways to apply this idea.


\section{Cycles of Fifths}

The cycle of fifths is a classic jazz sound that provides a nice way to build a substitute chord sequence over a long static harmony. Say you're playing over the first four bars of a blues in C, which consist of just C7:
\begin{quote}
\begin{tabular}{|p{3cm}|p{3cm}|p{3cm}|p{3cm}|}
	C7 & &  &
\end{tabular}
\end{quote}
You can take any dominant seventh and replace the first part of it with the dominant seventh built on its own fifth\footnote{In other books you might see this referred to as the `V of V' substitution}. In this case, it will be G7, because G is the fifth of C major. The idea is that G7 resolves to C, C will resolve to something else (F in this case):
\begin{quote}
\begin{tabular}{|p{3cm}|p{3cm}|p{3cm}|p{3cm}|}
	G7 & &C7  &
\end{tabular}
\end{quote}
You can use this approach chord anywhere where you see a dominant chord, even on top of another superimposed arpeggio, such as in the Charlie Parker or Coltrane cycles. The important thing is that it sets up a resolution, so you should be able to provide some kind of cadence at the end of the dominant chord, although you may choose to defeat your listeners' expectations by not resolving it after all, or not doing so in an obvious way.

The really nice thing about this trick, though, is that this can be repeated. So the G can be approached by its fifth, which is D7, and that in turn can be approached by \emph{its} fifth, A7:
\begin{quote}
\begin{tabular}{|p{3cm}|p{3cm}|p{3cm}|p{3cm}|}
	A7 &D7 &G7  &C7
\end{tabular}
\end{quote}
This chain of dominant chords built on the fifth of the next chord provides a sequence that resolves step by step, but resolves imperfectly because each chord is a dominant seventh that points the way to the \emph{next} resolution.

The application of this idea requires some preparation. At the beginning of the chorus, you need to decide to play a cycle of fifths that will last four bars, with one change per bar. You then need to work out where to start in order to achieve this. One useful approach is to learn the most common jumps you need to make at the beginning; for example, you can see that starting on the sixth of the original chord (A of C) will give you a four-step cycle, whereas starting on the 2 (D of C) gives three steps. When improvising it's worth making the cycle move fairly quickly, since the beginning of the superimposition will sound quite dissonant; reducing the time spent on each arpeggio to two beats would provide a framework for building a quickly-moving line whose dissonance was resolved reasonably smoothly.


\section{Other Possibilities}

The applications that follow constitute a lot of material, and are not intended to be used by the beginning student of single-note guitar playing. What they do indicate is that triad arpeggios are not just for beginners. There's a wealth of complexity and sophistication to be had, and it's possible to develop some unique and personal ideas using this supposedly basic material. The material becomes gradually more advanced between this point and the end of the chapter; the next chapter gets back to the basics.

It is possible, of course, to play any type of arpeggio over any chord, although the results vary widely. For each of the six common arpeggio types -- major seventh, dominant seventh, minor seventh, half-diminished, diminished and augmented -- there are twelve possibilities, making a total of 72, which really is too many to learn effectively\footnote{South Indian musicians learn 72 seven-note \emph{melakatas}, although the author does not know whether all 72 are learned at an early stage. The guitarist who knew well all 72 applications of the seventh arpeggios could make very sophisticated music without ever learning a single scale.}.

The tables in this section list those 72 applications and the resulting notes. They tell you, for instance, that playing a dominant seventh arpeggio whose root is the 2 of the chord -- say, D dominant seventh over a C-rooted chord -- gives the notes 2 (the root), $\sharp$4 (the third), 6 (the fifth) and 1 (the seventh). In the remainder of this book, we will use a symbol like $2^{d}$ to mean `play the dominant seventh arpeggio whose root is the 2 of the underlying chord'. The `$d$' superscript indicates that the arpeggio is a dominant seventh; we will also use $\Delta$ for major seventh arpeggios, $m$ for minor seventh, $\varnothing$ for half diminished, $+$ for augmented and $\circ$ for diminished\footnote{Later you will also see `maj', `min' and `dim' to indicate triads used in this way, specifically as ways to define scales by combining superimposed arpeggios, but don't worry about this for now.}.

You should look through these tables for applications that look useful and try them out. Those that sound good to you are definitely worth learning. Practice throwing the new arpeggio into a solo based on material you already know, and get used to how it feels in context. As always, practice a new arpeggio application in all positions so that you can always find it when you need it.

Aside from providing you with new musical resources, one benefit of this work is that it will help when it comes to learning scales and applying them creatively. Every heptatonic (seven-note) scale contains seven arpeggios, among which are usually some of the six common ones. When learning the scale, it can be very helpful if you find which of the familiar arpeggio patterns it contains apart, of course, from the one built on the root. For example, here's something called the Lydian scale in the `E' position:
%~E|5|2,#4,6,7% 
and here's the same scale, but with `$\times$'-shaped notes indicating the minor seventh arpeggio built on the (natural) seventh of the chord (so in A major, for instance, you would be playing G$\sharp$m7, because G$\sharp$ is the seventh note of the A major scale):
%~E|5|2x,#4x,6x,7x% 
You can probably see that the two patterns are identical. You can get the notes of very many scales, including all the most common ones, by playing either the triad or seventh built on the root and mixing it up with the notes of another triad or seventh arpeggio built on some other chord.

This is a helpful way to relate the sound of a new scale to something that's already familiar. The text sometimes points out useful arpeggio relationships like these, but many more can be found by experimenting with the fingerboard shapes of a particular scale. Don't worry if you don't know any scales at all yet; the point of the observations that follow is to provide you with an idea of how the common heptatonic scales and their modes sound by constructing them out of familiar arpeggios. If you don't yet know what the `Melodic Minor' or `Phrygian' scales are, that's fine. Just try out the suggested superimpositions and file away the information for future reference.

\subsection*{Major 7 Applications}

The table below is read as follows. Each row gives the notes that result from playing the $\Delta$ arpeggio with its root given by the first column. So , for example, this line:\\
\begin{tabular}{lllll}
5	&    7	&   2	&   $\sharp$4   &  $\Delta$ [Lydian scale with 1$\Delta$]        
\end{tabular}\\
tells you first that the $\Delta$ arpeggio built on the fifth of the chord gives the pitches 2, $\sharp$4, 5 and 7. It then says that the most likely application of the superimposition is over another $\Delta$ chord. Contrast this with the $\Delta$ arpeggio built on the $\flat$3, which gives notes that are consonant over a minor seventh chord but dissonant over a $\Delta$. Be sure you understand this; just because we're playing a $\Delta$ arpeggio doesn't mean it will work smoothly over a $\Delta$ chord because the arpeggio and chord have different roots. Finally, this row indicates that you get all of the notes of the Lydian scale by combining this superimposed arpeggio with the $\Delta$ arpeggio built on the root; this will be helpful if you already know what the Lydian Scale is and you're interested in learning it. If a row is marked with squares around the notes, that row is a mode of the original arpeggio; that is, it contains a root. The other rows do not contain roots, and so they're not scales at all but `hypermodes' (see Chapter 1).

\begin{tabular}{lllll}
\begin{tabular}{|c|}\hline 1\\ \hline\end{tabular}	&    \begin{tabular}{|c|}\hline 3	\\ \hline\end{tabular}&   \begin{tabular}{|c|}\hline 5\\ \hline\end{tabular}	&   \begin{tabular}{|c|}\hline 7\\ \hline\end{tabular}   &   Basic application        \\  
\begin{tabular}{|c|}\hline $\flat$2\\ \hline\end{tabular}	&    \begin{tabular}{|c|}\hline 4\\ \hline\end{tabular}	&   \begin{tabular}{|c|}\hline $\flat$6\\ \hline\end{tabular}	&   \begin{tabular}{|c|}\hline 1  \\ \hline\end{tabular} &   Augmented or minor  [Phrygian scale with 1m7]      \\  
2	&    $\flat$5	&   6	&   $\flat$2   &  Possibly minor (or dominant)        \\  
$\flat$3	&    5	&   $\flat$7	&   2   &   Minor (or dominant) [Dorian scale when combined with 1m7]      \\  
3	&    $\flat$6	&   7	&   $\flat$3   &  $\Delta$         \\  
\begin{tabular}{|c|}\hline 4\\ \hline\end{tabular}	&    \begin{tabular}{|c|}\hline 6\\ \hline\end{tabular}	&   \begin{tabular}{|c|}\hline 1\\ \hline\end{tabular}	&   \begin{tabular}{|c|}\hline 3 \\ \hline\end{tabular}  &   $\Delta$ [Major scale with 1$\Delta$]        \\  
$\flat$5	&    $\flat$7	&   $\flat$2	&   4   &   Minor (or dominant)	    \\  
5	&    7	&   2	&   $\sharp$4   &  $\Delta$  [Lydian scale with 1$\Delta$]       \\  
\begin{tabular}{|c|}\hline $\flat$6\\ \hline\end{tabular}	&    \begin{tabular}{|c|}\hline 1\\ \hline\end{tabular}	&   \begin{tabular}{|c|}\hline $\flat$3\\ \hline\end{tabular}	&   \begin{tabular}{|c|}\hline 5 \\ \hline\end{tabular}  &   Possibly minor (or dominant)        \\  
6	&    $\flat$2	&   3	&   $\flat$6   &  Possibly $\Delta$        \\  
$\flat$7	&    2	&   4	&   6   &   Minor (or dominant) [Mixolydian scale with 1 dominant 7]       \\  
7	&    $\flat$3	&   $\flat$5	&   $\flat$7   &  Minor         
\end{tabular}



\subsection*{Dominant 7 Applications}

The dominant seventh applications include, of course, both the Charlie Parker and Coltrane cycles. The others are all worth experimenting with too, although none of them gives a common heptatonic scale except for the $\flat$7, which gives you either the Aeolian or a mode of the Melodic Minor, depending on which triad you have in the root position.

\begin{tabular}{lllll}
\begin{tabular}{|c|}\hline 1\\ \hline\end{tabular}	&    \begin{tabular}{|c|}\hline 3\\ \hline\end{tabular}	&   \begin{tabular}{|c|}\hline 5\\ \hline\end{tabular}	&   \begin{tabular}{|c|}\hline $\flat$7 \\ \hline\end{tabular} &   Basic application        \\  
$\flat$2	&    4	&   $\flat$6	&   7   &   $\Delta$       \\  
\begin{tabular}{|c|}\hline 2\\ \hline\end{tabular}	&    \begin{tabular}{|c|}\hline $\flat$5\\ \hline\end{tabular}	&   \begin{tabular}{|c|}\hline 6\\ \hline\end{tabular}	&   \begin{tabular}{|c|}\hline 1\\ \hline\end{tabular}    &  Any         \\  
$\flat$3	&    5	&   $\flat$7	&   $\flat$2  &   Dominant or minor; Charlie Parker Cycle, step one        \\  
3	&    $\flat$6	&   7	&   2    &  $\Delta$; Coltrane Cycle, step one         \\  
\begin{tabular}{|c|}\hline 4\\ \hline\end{tabular}	&    \begin{tabular}{|c|}\hline 6\\ \hline\end{tabular}	&   \begin{tabular}{|c|}\hline 1\\ \hline\end{tabular}	&   \begin{tabular}{|c|}\hline $\flat$3\\ \hline\end{tabular}  &   Any        \\  
$\flat$5	&    $\flat$7	&   $\flat$2	&   3   &   Dominant; Charlie Parker Cycle, step two        \\  
5	&    7	&   2	&   4    &  $\Delta$\\  
\begin{tabular}{|c|}\hline $\flat$6\\ \hline\end{tabular}	&    \begin{tabular}{|c|}\hline 1\\ \hline\end{tabular}	&   \begin{tabular}{|c|}\hline $\flat$3	\\ \hline\end{tabular}&   \begin{tabular}{|c|}\hline $\flat$5 \\ \hline\end{tabular} &   Any; Coltrane Cycle, step two        \\  
6	&    $\flat$2	&   3	&   5    &  $\Delta$ or dominant; Charlie Parker Cycle, step three         \\  
$\flat$7	&    2	&   4	&   $\flat$6  &   Minor or dominant  [Aeolian with 1m7]      \\  
7	&    $\flat$3	&   $\flat$5	&   6    &  $\Delta$         
\end{tabular}


\subsection*{Minor 7 Applications}

\begin{tabular}{lllll}
\begin{tabular}{|c|}\hline 1\\ \hline\end{tabular}	&    \begin{tabular}{|c|}\hline $\flat$3\\ \hline\end{tabular} &   \begin{tabular}{|c|}\hline 5\\ \hline\end{tabular}	&   \begin{tabular}{|c|}\hline $\flat$7 \\ \hline\end{tabular} &   Basic application        \\  
$\flat$2	&    3	&   $\flat$6	&   7   &   $\Delta$\\  
\begin{tabular}{|c|}\hline 2\\ \hline\end{tabular}	&    \begin{tabular}{|c|}\hline 4\\ \hline\end{tabular}	&   \begin{tabular}{|c|}\hline 6\\ \hline\end{tabular}	&   \begin{tabular}{|c|}\hline 1\\ \hline\end{tabular}    &  Any \\  
$\flat$3	&    $\flat$5	&   $\flat$7	&   $\flat$2  &   Minor (or dominant)\\  
3	&    5	&   7	&   2    &  $\Delta$\\  
\begin{tabular}{|c|}\hline 4\\ \hline\end{tabular}	&    \begin{tabular}{|c|}\hline $\flat$6\\ \hline\end{tabular}	&   \begin{tabular}{|c|}\hline 1\\ \hline\end{tabular}	&   \begin{tabular}{|c|}\hline $\flat$3 \\ \hline\end{tabular} &   Minor (or dominant, or $\Delta$)\\  
$\flat$5	&    6	&   $\flat$2	&   3   &   Dominant or $\Delta$\\  
5	&    $\flat$7	&   2	&   4    &  Minor or dominant\\  
$\flat$6	&    7	&   $\flat$3	&   $\flat$5  &   $\Delta$\\  
\begin{tabular}{|c|}\hline 6\\ \hline\end{tabular}	&    \begin{tabular}{|c|}\hline 1\\ \hline\end{tabular}	&   \begin{tabular}{|c|}\hline 3\\ \hline\end{tabular}	&   \begin{tabular}{|c|}\hline 5\\ \hline\end{tabular}    &  $\Delta$ or dominant \\  
$\flat$7	&    $\flat$2	&   4	&   $\flat$6  &   Minor or dominant [Phyrygian with 1m7]\\  
7	&    2	&   $\sharp$4	&   6    &  $\Delta$ [Lydian with 1$\Delta$]
\end{tabular}



\subsection*{Half-Diminished Applications}

Note that you can play a Mixolydian $\flat$6 in two ways over a dominant chord; with the 2$\varnothing$ or with the $\flat$7$\Delta$. Using these two superimpositions as alternatives over a dominant seven harmony can be a great jumping-off point for exploring the sound of this scale which, as you will discover in the next chapter, is a mode of the Melodic Minor. A similar observation goes for the Phrygian Major, which is a mode of the Harmonic Minor.

Note that playing a half-diminished built on the natural seventh over a minor triad in root position gives the melodic minor scale. This is a very nice, simple superimposition to play over a minor sixth or m$\Delta$ (minor triad with natural seventh) chord.

\begin{tabular}{lllll}
\begin{tabular}{|c|}\hline 1\\ \hline\end{tabular}	&    \begin{tabular}{|c|}\hline $\flat$3\\ \hline\end{tabular} &   \begin{tabular}{|c|}\hline $\flat$5 \\ \hline\end{tabular} &   \begin{tabular}{|c|}\hline $\flat$7\\ \hline\end{tabular}  &   Basic application        \\
$\flat$2	&    3	&   5	&   7   &   $\Delta$\\  
\begin{tabular}{|c|}\hline 2\\ \hline\end{tabular}	&    \begin{tabular}{|c|}\hline 4\\ \hline\end{tabular}	&   \begin{tabular}{|c|}\hline $\flat$6\\ \hline\end{tabular}	&   \begin{tabular}{|c|}\hline 1\\ \hline\end{tabular}    &  Any [Aeolian with 1m7]\\  
$\flat$3	&    $\flat$5	&   6	&   $\flat$2  &   Minor (or dominant)\\  
3	&    5	&   $\flat$7	&   2    &  Dominant\\  
4	&    $\flat$6	&   7	&   $\flat$3  &   $\Delta$\\  
\begin{tabular}{|c|}\hline $\flat$5\\ \hline\end{tabular}	&    \begin{tabular}{|c|}\hline 6\\ \hline\end{tabular}	&   \begin{tabular}{|c|}\hline 1\\ \hline\end{tabular}	&   \begin{tabular}{|c|}\hline 3\\ \hline\end{tabular}   &   $\Delta$ or dominant \\  
5	&    $\flat$7	&   $\flat$2	&   4    &  Minor or dominant [Phyrygian with 1m7]\\  
$\flat$6	&    7	&   2	&   $\flat$5  &   $\Delta$\\  
\begin{tabular}{|c|}\hline 6\\ \hline\end{tabular}	&    \begin{tabular}{|c|}\hline 1\\ \hline\end{tabular}	&   \begin{tabular}{|c|}\hline $\flat$3\\ \hline\end{tabular}	&   \begin{tabular}{|c|}\hline 5 \\ \hline\end{tabular}   &  Minor (or dominant)\\  
$\flat$7	&    $\flat$2	&   3	&   $\flat$6  &   Dominant\\  
7	&    2	&   4	&   6    &  $\Delta$ [Major scale with 1 $\Delta$; melodic minor with minor triad]
\end{tabular}

\subsection*{Diminished Applications}

\begin{tabular}{lllll}
\begin{tabular}{|c|}\hline 1\\ \hline\end{tabular}	&    \begin{tabular}{|c|}\hline $\flat$3\\ \hline\end{tabular} &   \begin{tabular}{|c|}\hline $\flat$5\\ \hline\end{tabular}  &   \begin{tabular}{|c|}\hline 6\\ \hline\end{tabular}  &   Basic application        \\
$\flat$2	&    3	&   5	&   $\flat$7  &   Dominant \\  
2	&    4	&   $\flat$6	&   7    &  $\Delta$\\  
\end{tabular}

  
\subsection*{Augmented Applications}

\begin{tabular}{llll}
\begin{tabular}{|c|}\hline 1\\ \hline\end{tabular}	&    \begin{tabular}{|c|}\hline 3\\ \hline\end{tabular}	&   \begin{tabular}{|c|}\hline $\sharp$5\\ \hline\end{tabular}	&   Basic application        \\  
$\flat$2	&    4	&   6	&   Any        \\  
2	&    $\flat$5	&   $\flat$7	 &  Minor or dominant        \\  
$\flat$3	&    5	&   7	&   $\Delta$       \\  
\end{tabular}


There are many ways to use the information in these tables, whose study alone could take an entire lifetime of playing. Consider, for example, David Liebman's suggestion\footnote{Liebman, p.19.} of playing ii-V7 in \emph{any} key as an approach to a major chord. So in a blues in E we might anticipate the A7 in bar 5 by means of the ii-V7 in, say, the key of E$\flat$. Then you would play Fm7 and then B$\flat$7 arpeggios over the E7 in bar 4 and resolve to the A7 arpeggio in bar 5. The result is curiously pleasing, producing considerable tension over E7 and a nice resolution down a half-step coming into the A7.

To see what effect the superimposition just suggested would really have, let 1, the root, be E, which is the underlying harmony of bar 4. Now look up the m7 arpeggio based on the $\flat$2 note, which is what the Fm7 is, and the dominant seventh based on the $\flat$5, which is the B$\flat$7:\\
\begin{tabular}{lllll}
$\flat$2	&    3	&   $\flat$6	&   7   &   $\flat$2 minor 7\\  
$\flat$5	&    $\flat$7	&   $\flat$2	&   3   &   $\flat$5 dominant 7
\end{tabular}\\
So over the E7, we will play 1-$\flat$2-3-$\sharp$4-$\sharp$5-$\sharp$6-7. This is a pretty dissonant set of notes, but it sounds logical because of the implied harmonic function; the clash over the minor seventh leads to A as its natural conclusion. We will meet these notes again, much later, as the Enigmatic scale, which was invented by Verdi, has been used by Joe Satriani and is a mode of a South Indian scale called Manavati. It's curious and rather heartwarming to see four largely-unrelated traditions meet over a set of seven unpromising-looking notes.

To study the implications of an idea like this further, consider that ii-V7 is a minor seventh arpeggio followed by the dominant seventh a fourth (five semitones) above. The tables above enable you to quickly analyse any ii-V7 superimposition to see what notes it will contain. Pick any minor seventh arpeggio, say the one built on the $\flat$7 this time (so in C we'll play B$\flat$m-F7). Now go to the row for the dominant seventh built on the $\flat$7 (the one with `$\flat$7' in the leftmost column, of course) and count up five rows, going back to the top of the table if you get to the bottom. In this case, the result will be the dominant arpeggio built on the 3 of the chord. The two rows thus identified tell you what pitches you will be playing over the underlying harmony. Try this for yourself using the tables above, starting with any numbered note you like.

A more theoretically complicated example, but easier in practice, is provided by a pair of arpeggios forming a bitonal chord such as C/D, a C major triad stacked on top of a D major triad. Look at the major seventh table, and ignore the right-hand column that contains the seventh; we'll only be playing triads in this example. Now pick any row, miss one out and pick the next one. These two rows give a pair of major triads separated by a semitone. You can immediately see what notes you would get by superimposing lines made up of these two arpeggios, freely mixed together, over the root harmony. For example, playing C major and D major over an A-rooted chord gives:\\
\begin{tabular}{lllll}
$\flat$3	&    5	&   $\flat$7	&   C triad  \\  
4	&    6	&   1	&   		   D triad  
\end{tabular}\\
This is a simple `inside' sound over a minor seventh, or a bluesy sound over a dominant seventh. If, on the other hand, you chose C$\sharp$ and D$\sharp$ -- the same pattern, but a semitone higher -- you would have\\
\begin{tabular}{lllll}
3	&    $\flat$6	&   7	&   C$\sharp$         \\  
$\flat$5	&    $\flat$7	&   $\flat$2	&   D$\sharp$ 
\end{tabular}\\
which is far more dissonant, although you can easily get away with it over A7.




\chapter{\mbox{The} \mbox{Commonest} \mbox{Scales}}

This chapter covers the scales that most guitar scale books include (although many don't get through all of them). Subsequent chapters cover more exotic scales that you can explore and experiment with, but those in this chapter are eventually learned by most serious students of the instrument. The `Common Pentatonic' scale has, as its name suggests, five notes, while the Major, the Melodic Minor and the Harmonic Minor are heptatonic (seven-note) scales. We also cover all of the modes of these scales, so there are 26 (3$\times$7 + 5) scales in this chapter. 

Beginners are advised to work as follows. First, learn the Common Minor and Common Major pentatonics. Next learn their related `Blues Scales', which only involve adding a single note (modes I and II only). Once these are familiar study all of the scales in the Major Scale group. Optionally, return to the Common Pentatonic group and complete that study. Then progress to the Harmonic Minor and Melodic Minor groups, choosing one group and focussing on most of its scales before moving onto the other. Bear in mind that this process should be done slowly and will take a considerable amount of time. Only move on to a new scale when the one you have been studying is fully integrated into your playing style and feels entirely comfortable. There is absolutely no need to hurry this process.

This chapter's presentation is similar to that of the rest of the book. The scales are arranged into scale groups; within each group, each scale is a mode of each of the others. For this reason, the fingerings of one scale will be the same as all the others, with an important difference: the root, and hence the other arpeggio notes, will be in different places. Because of this the position fingerings will often look very different when you look at them one by one. When you know them well enough to join them up, however, you will find they are all identical. As you work your way through the other scales in the same group you'll find that learning them becomes much easier and quicker.

Next to each scale's name is a spelling (also called a `formula') indicating which notes are in the scale and also which seventh chords are formed from them using only the notes from the scale:
\begin{quote}
	$1^{\Delta}$ $2^{m}$ $3^{m}$ $4^{\Delta}$ $5^{d}$ $6^{m}$ $7^{\varnothing}$
\end{quote}
You should always begin by looking at the seventh chord built on the root, and playing the scale over that chord. From the spelling just given, for example, the symbol $1^{\Delta}$ tells you that the scale can be played over a $\Delta$ chord. Of course, it may work over other kinds of chord; your ears will be the guide to that. All that the spelling tells you is that the scale contains all of the notes of the $\Delta$ chord built on its root, so it will probably be reasonably consonant over that type of chord.

The spelling also provides seventh arpeggios for the other scale tones. These are \emph{not} to be taken to indicate the types of chords the scale works with\footnote{Many instructional books do take this approach, but the author is firmly of the opinion that this is a mistake. It's hard to think on your feet about playing over, say, A$\flat$m7 if your thought process has to be, `I want the Dorian scale, so I need the major scale of which A$\flat$ is the second, so I need to play F$\sharp$ major', by which time the band will quite possibly have gone home. It is worth doing what appears to be extra work of learning each of the modes individually in order to gain the fluency necessary to make practical use of them.}; rather, you should use them exclusively as melodic resources to help make sense of a scale, and in particular to break up the monotonous stepwise motion that can easily develop when first learning a set of fingerboard patterns. For example, the symbol $2^{m}$ tells you that you can play the m7 arpeggio built on the second scale note, and the result will contain only notes from this scale. Playing two adjacent seventh arpeggios usually gives you all of the notes in a heptatonic scale. Pentatonics, on the other hand, often do not form seventh chords on all or even very many of their notes; in those cases the number stands on its own. 

\section{The Common Pentatonic Scale Group}

Every guitarist should be intimately familiar with the Common Minor pentatonic scale. The great majority of guitar solos in rock and blues use this scale very extensively, and it's often heard in jazz and various folk styles as well. Most guitarists learn this scale first, and many play it for several years before needing to look for new resources (although your appetite for this will depend largely on the type of music you want to play). The Common Major Pentatonic is important in North American folk and `country' music. The other modes in the group are fairly often heard over a dominant chord as variations on the Common Minor.

Searchers after `exotic' scales might like to know that some of these scales have a Japanese connection. The Common Minor is roughly similar to a scale called `Yo', the Mixolydian Pentatonic is similar to Ritusen and the Common Major itself is like Ryosen. In China a structure similar to the Common Major is called `Gong' and all of its modes are used. Something a little like the Common Major can even be heard in Korean music as `Pyeong Jo'. There will be more said later about the folly of such comparisons, but if you become tired of playing blues and country lines using these scales, try playing along with a recording of some Southeast Asian music instead and see what you discover.

%~M|5|1,2,3,5,6|Common Major Pentatonic@Dorian Pentatonic@Phrygian Pentatonic@Mixolydian Pentatonic@Common Minor Pentatonic%

\section{The Blues Scales}

After learning the Common Minor Pentatonic it's very common for blues and rock guitarists to learn something that's often referred to as the `Blues Scale'. The scale is simply a Common Minor Pentatonic with one note added, the $\flat5$. It's a great scale to learn at this point because it spices up the Common Minor Pentatonic with very little effort. In fact the $\flat5$ is not the only note that can be added to the pentatonics to give interesting new sounds with very little effort: students can get a lot of benefit out of learning all three scales listed in this section. The other two add the $\natural 3$ and $\natural 7$ to the Common Minor Pentatonic. 

Each addition creates a hexatonic scale with five other modes in its group. Beginners should initially learn the first two modes only, and relate them carefully to both the Common Minor or Major pentatonics and the underlying triads. The other modes need not be learned early on (most players do not use them at all) but they are easy to learn as extensions of the modes of the Common Pentatonic and offer some quite interesting variations. 

Most players in blues, jazz and rock styles will get a lot more out of learning these early on than ploughing on with the heptatonic scales, which seem to encourage mindless noodling in those who have not spent enough time thinking about chord tones and the use of tension. The scales in this section offer opportunities to do that thinking and also sound great.

\subsection*{The First Blues Scale Group}

This scale adds the $\flat 5$ to the Common Minor Pentatonic.

%~M|5|1,b3,4,b5,5, b7|First Blues Scale Mode I@First Blues Scale Mode II@First Blues Scale Mode III@First Blues Scale Mode IV@First Blues Scale Mode V%

\subsection*{The Second Blues Scale Group}

This scale adds the $\natural 3$ to the Common Minor Pentatonic.

%~M|5|1,b3,3,4,5, b7|Second Blues Scale Mode I@Second Blues Scale Mode II@Second Blues Scale Mode III@Second Blues Scale Mode IV@Second Blues Scale Mode V%

\subsection*{The Third Blues Scale Group}

This scale adds the $\natural 7$ to the Common Minor Pentatonic.

%~M|5|1,b3,4,5, b7,7|Third Blues Scale Mode I@Third Blues Scale Mode II@Third Blues Scale Mode III@Third Blues Scale Mode IV@Third Blues Scale Mode V%


\section{The Major Scale Group}
The Major scale provides all of the unaltered notes in a given key. In C major, these are just the natural notes, with no sharps or flats, but whichever key you want to work in, the major scale gives just those sharps or flats that appear in the key signature, no more and no less. For that reason alone, it is very important to be able to find a major scale in any key and in any position on the fingerboard because that's a huge help with music-reading. These scales have been used again and again in rock, jazz and those kinds of music that mix up the two. An amazing variety of different sounds can be squeezed from them, and just because they're commonly-used doesn't mean they sound boring.

As has already been mentioned, many heptatonic scales may be considered as a pair of interlocked seventh arpeggios. One is built on the root and corresponds to the underlying harmony. The other is built on another note, and includes the rest of the scale tones. Such superimposed arpeggios can help you to learn the patterns, because the shapes will already be familiar. You can also use them creatively, for example by playing the superimposition and then resolving to an underlying chord tone. The `dominant superimposition at the second', for example, means the dominant seventh arpeggio built on the 2 of the underlying harmony. You can find the 2 by going up two frets from the root or down two frets from the third (or down one fret from the minor third, of course). With time and practice you will be able to find any numbered note in any key in any position, but this facility will not come immediately and need not be forced.

In many cases, you will find one superimposition implying various different scales depending on the underlying (root-based) triad. For example, the dominant arpeggio built on the second is a superimposition that comes up several times in this chapter. If the underlying harmony is C$\Delta$ chord then a Lydian scale is obtained by mixing up the notes of the underlying harmony C$\Delta$ with D7, the dominant superimposition at the second, and there are more possibilities:
\begin{quote}
	\begin{tabular}{l|l}
		Underlying harmony & Resulting scale \\
		\hline $\Delta$ & Lydian \\
		Minor 7 & Dorian $\sharp$4 \\
		Dominant 7 & Lydian Dominant \\
		Augmented & Lydian Augmented
	\end{tabular}
\end{quote}
Hence, before you've seen a single fingerboard diagram, you can play all four of these scales (which cover all three common heptatonic scale groups) just by mixing up the notes of a few arpeggios that you should already know very well. This exercise is well worth doing and you should look out for other such connections, of which there are many.

Another suggestion for exploring the sounds of these modes is to play only the notes that are a semitone apart; in the case of the Major scale, that would be 1, 3, 4 and 7. With the Phrygian, say, it would be 1, $\flat 2$, 5 and $\flat 6$. Try to come up with some phrases using just those notes -- it helps to embed the visual pattern of the scale more deeply in your memory, and it also produces a striking sound which is very different in each mode. Any approach like this is also likely to help you to avoid just playing scales up and down, which can be very boring to listen to.

When faced with a group of seven scales, energy should be focused on a single scale first; in this case, it's best to use the Major Scale even though its sound isn't the most exciting in the group. Then learn a few of the other modes of this scale, preferably one for each type of underlying chord; you could choose the Lydian ($\Delta$), Mixolydian (dominant) and Aeolian (minor). Once these are familiar, complete the group by learning the Dorian, which you will probably use very often, and the Phrygian and Locrian, which you're likely to use less frequently. This approach helps to break up the work -- and there's a lot of work here to be done -- in a way that's preferable to rote memorisation.

%~M|5|1,2,3,4,5,6,7|
Major`For the most important scale in all Western music from an organisational standpoint, you might find the Major Scale sounds a bit dull. In fact this is a direct result of its overuse. Few guitar solos use the Major Scale but very many use one or more of its modes. If you have the patience, though, you should learn this scale before working on the others in the group.
@Dorian`This scale is a standard way to play over a minor or minor seventh harmony and also sounds good in a blues or rock context over a dominant chord.
@Phrygian`The Phrygian scale has a very distinctive sound and offers a chance to discover the effect of the $\flat$2. This scale works in a straightforward way over a minor seventh but can also sound quite sophisticated over a dominant chord. The dominant substitution at the flat third is step one of the Charlie Parker Cycle, but it doesn't contain all of the scale tones (in combination with the root minor seventh arpeggio); the minor substitution at the flat seventh does and is also worth playing with.
@Lydian`There are altered versions of the Lydian scale that offer more flavour, but this offers a basic jazz sound over a $\Delta$ harmony. To hear it applied in a rock context listen to Frank Zappa's solo on Hog Heaven\footnote{From \emph{Shut up and Play Yer Guitar}; there is a transcription by Steve Vai although for reasons that have nothing to do with its quality the book is currently difficult to obtain.}, which is almost pure E Lydian throughout with only a very small number of altered notes. Vai himself also frequently uses this scale.
@Mixolydian`A plain, unfussy way to play over a dominant seventh chord. This book contains many, many  more advanced sounds for dominant harmonies but this one is perhaps the simplest of all.
@Aeolian`Similar to the Dorian, but again a little plainer, this scale also provides all of the unaltered notes in a minor key. It's often referred to as the Natural Minor scale, probably because it contains all the unaltered notes in a given minor key (although they won't, of course, all be natural notes except in the key of A minor.).
@Locrian`This is the most dissonant scale in the group. It provides a natural way to play over half-diminished chords, but it can also be played over a dominant seventh chord to provide a whole collection of altered notes; it has  a sophisticated dissonance that can be resolved to a more consonant scale (such as the Mixolydian), to the bare underlying arpeggio, or to Common Minor Pentatonic-based blues phrases.
%

\section{The Harmonic Minor Scale Group}

The Harmonic Minor scale was invented, so the story goes, to provide the Aeolian scale with a `leading-note'; a natural 7 that provides a strong resolution back to the root note. It can be heard especially frequently in compositions of the Baroque period, including those of Bach, Vivaldi and Handel. Later composers in the classical tradition used it less frequently, but it was revived in the twentieth century by both jazz and heavy metal musicians. 

The Harmonic Minor scale itself works most naturally over a m$\Delta$ chord -- a minor triad with a natural seventh -- which is uncommon in much music. Among the harmonic minor modes are many scales used to create classic jazz and rock sounds; the Harmonic Minor itself makes an appearance in many a bebop solo. 

Since it's an Aeolian scale with one altered note, we would expect the other scales in its group to be modes of the Major scale with a single altered note. This is so in all but one case, the Super Locrian $\flat\flat$7; there the altered note is the root, which affects the spelling of the scale more dramatically. You can use this knowledge to your advantage if you already know the Major scale fingerings well, since each fingering differs in only two or three places from the corresponding fingering in the previous group. Compare, for example, the first fingering given for the Aeolian with that for the Harmonic Minor. 

One fruitful way to approach these modes is to notice that they will always contain, along with the arpeggio of the chord you're playing over, an augmented triad and a diminished seventh arpeggio. These can be used to construct runs through the scale-tones that avoid the rather disjointed quality of the scale when it's played stepwise. For example, if you're playing the Lydian $\sharp 2$, the arpeggio is of course a major triad. Playing the augmented arpeggio shape then picks out the notes $\flat 3$, 5 and 7, while the diminished seventh arpeggio picks out 1, $\flat 3$, $\flat 5$ and 6. Hence, an interesting way to play the scale would be to play one of these arpeggios to create tension, then resolve by playing one or more of the notes from the major triad. Of course this only works if the mode you're working on doesn't have the augmented or diminished chord as its root harmony; if you're playing the Augmented scale, for example, the underlying harmony is probably an augmented chord, so you can't use the augmented arpeggio to get this tension-and-release effect. You could still use the diminished arpeggio, though.

The following table summarises some suggested arpeggio superimpositions to use when learning the modes in this group:
\begin{quote}
	\begin{tabular}{ll}
		Harmonic Minor & $7^\circ$ \\
		Locrian $\natural 6$ & $\flat 2^+$\\
		Augmented & $3^d$\\
		Dorian $\sharp 4$ & $2^d$\\
		Phrygian Major & $\flat 2^{\Delta}$\\
		Lydian $\sharp 2$ & $7^d$\\ 
		Super Locrian $\flat\flat$7 & $\flat\flat 7^d$
	\end{tabular}
\end{quote}
Although all have different sounds, any superimposition that contains only notes within the scale is as legitimate as any other; these are just suggestions.

Again, it's worth using the `divide and conquer' approach described for the major scale group. Start by learning the Harmonic Minor itself, and spend as much time as you can getting familiar with all of its positions. Then you might like to concentrate on the Lydian $\sharp 2$ (for $\Delta$ chords), Phrygian Major (for dominants) and Dorian $\sharp 4$ (for minors). Once you have these under your fingers you can progress to the Augmented, Locrian $\natural 6$ and Super Locrian $\flat\flat$7, all of which you're likely to only want to use occasionally -- you may even choose not to bother with these at all at first. There's nothing to stop you from learning some of the scales in a group but then going on to another group before completing your study. These scales have a very distinctive sound that can get tiresome after a while; you need variety and contrast in your playing to set them off. 

%~M|5|1,2,b3,4,5,b6,7|
Harmonic Minor`This scale is rarely heard in a straightforward harmonic application; in rock music, you will most likely hear it played over a minor triad or a power chord. It also sounds good as a fairly dissonant choice over a $\Delta$ chord; try using it to add temporary tension to Lydian phrases.
@Locrian $\natural 6$`If the underlying harmony is a half-diminished chord, try the augmented superimposition at the flat second; this, combined with the underlying arpeggio notes, gives all the notes in the scale. When using the scale over a dominant chord, which introduces a very effective element of dissonance, you might also want to try the minor superimposition at the flat third, which contains most of the notes of the half-diminished arpeggio plus the $\flat$2, all of which are colourful sounds in that context. The same applies to minor harmonies, over which this scale also works nicely.
@Augmented`Compared with the Lydian $\sharp$2 (see below), this scale offers a more dissonant, less bluesy and in a sense more modern sound over a $\Delta$ chord; the $\sharp$5 clashes with the $\natural$5 in the chord, but in a good way (if you like that sort of thing). You can mix up this scale, the Lydian $\sharp$2 and the ordinary Lydian to get different levels of dissonance. Note also that the dominant substitution at the third is one step of the Coltrane cycle.
@Dorian $\sharp 4$`The $\sharp 4$ acts as a blue note (it's the same as a $\flat 5$), making this scale a good choice over any minor seventh chord or minor triad, and it also works in a bluesy way over dominant-type harmonies as well.
@Phrygian Major`This scale is a very common feature of Spanish music, and can easily be used to conjure up a kind of fake Flamenco. It's also often heard as an altered dominant sound in jazz.
@Lydian $\sharp 2$`This is an extremely useful scale, the $\sharp 2$ acting as a blue note and giving an unusually bluesy sound over a $\Delta$ harmony. It can therefore be used to provides a bridge between familiar minor-pentatonic-based blues lines, which don't sound good over $\Delta$ chords, and more traditional Lydian phrases that can sound plain by comparison.
@Super Locrian $\flat\flat$7`The $\Delta$ superimposition at the double-flat seventh is similar to the dominant superimposition at the sixth (since $\flat\flat$7 = 6), and is therefore similar to the third step in the Charlie Parker Cycle. This is an extremely dissonant scale; your ears should be the judge of whether you can use it and in what contexts. 
%

\section{The Melodic Minor Scale Group}

The Harmonic Minor sounds good, but it has a rather angular quality as a result of the minor third interval it contains. For modern players this is not necessarily a problem, but purely Harmonic-Minor-derived music can get tiresome after a while. Yet, historically, having a minor scale with a natural 7 was extremely useful for giving pieces in a minor key a firm resolution. Instead of raising the seventh of a minor scale, however, another way to achieve this would be to flatten the third of a major scale; in this way, the Melodic Minor scale, a Major scale with a flat third, was created\footnote{If you have read other scale books you might have seen the Melodic Minor presented with different descending and ascending forms. In Western music theory this is an historical quirk that has long ceased to be meaningful to contemporary musicians. Such nonsense is still, unfortunately, perpetuated occasionally in print, usually in relation to the Melodic Minor or certain `exotic' scales whose provenance is usually suspect anyway. If you don't know what this footnote is referring to, good.}

Because there's no large minor 3rd interval, these scales flow more evenly and have none of the spiky quality of the Harmonic Minor group (probably you liked this quality, but it's good to have some variety). The Melodic Minor itself is particularly smooth and soft-sounding; try improvising some slow melodies with it, over a minor triad. With each mode, you will again find there is an augmented arpeggio in the shape, and it is worth using this for tension, resolving back to a chord tone.

One of the nicest things about how the arpeggios of this scale turn out is the pair of dominant arpeggios a tone apart. Jazz musicians can have a lot of fun with these; just notice which scale tones they're built on and weave them into your solo lines, using either or both of them to build phrases.

Again, here are some suggested superimpositions that may help you to get started in the study of these scales:
\begin{quote}
	\begin{tabular}{ll}
	Melodic Minor & $7^{\varnothing}$\\
	Dorian $\flat 2$ & $\flat 2^+$\\
	Lydian Augmented & $3^d$\\
	Lydian Dominant & $2^d$\\
	Mixolydian $\flat 6$ & $\flat 7^d$\\
	Half-Diminished & $\flat 7^d$\\
	Super Locrian & $\flat 5^d$
	\end{tabular}
\end{quote}
Notice that some of these are the same as the suggested Harmonic Minor superimpositions; the differences between the scales are in the underlying arpeggios (at the root) in those cases. The Dorian $\flat 2$, for example, can be considered a minor seventh arpeggio plus the augmented arpeggio built on the $\flat$2, whereas the Locrian $\natural 6$ is a half-diminished arpeggio with the same augmented superimposition.

%~M|5|1,2,b3,4,5,6,7|
Melodic Minor
@Dorian $\flat 2$`A funky, bluesy sound with the sour $\flat$2 giving it a real edge. This is a great scale to use as a variation on the straight Dorian when building tension in a solo.
@Lydian Augmented`This one is a brave choice over a $\Delta$ harmony and an out-and-out reckless one over an altered dominant chord (although Coker recommends it there). Jazz players will find it's worth experimenting with in both contexts. Doug Munro, in his book on bebop, uses this scale to create a couple of licks in the style of Mike Stern and McCoy Tyner over $\Delta$ harmonies.
@Lydian Dominant`This scale is sometimes referred to as the Overtone Dominant due to the way it roughly approximates part of the overtone series. Some otherwise excellent writers (including the legendary George Russell) make the mistake of thinking this makes it a sort of scale of nature; on this see Serafine p.21.
@Mixolydian $\flat 6$`This quite rarely-heard scale provides a surprisingly usable blues-folk sound with a dash of dissonance.
@Half-Diminished`A useful alternative to the Locrian for handling half-diminished chords or a strongly dissonant choice for dominant sevenths (in which context most players will want to resolve to something more familiar).
@Super Locrian`Despite the highly-altered nature of this scale, it can be heard as a frequent choice for jazz and fusion soloists over the 7$\sharp$9 chord, a common altered dominant sound that also features in a number of well-known Hendrix songs and riffs.
%

A final perspective on the three common heptatonic groups is in order before we leave them, and it can help you to integrate these different sounds more fluidly into your playing. You can think of each scale in the Harmonic or Melodic Minor groups as being a slightly altered version of a scale in the Major group, and you can think of them as providing three different degrees of `tension'. For instance, take the three scales
\begin{quote}
	\begin{tabular}{l|l|l}
		Dorian & Dorian $\flat 2$ & Locrian $\natural 6$
	\end{tabular}
\end{quote}
and play a solo over a minor (or minor seventh) chord, moving from one scale to another and back again. You'll probably find that the Dorian scale is best for resolving phrases, but the other two introduce different levels of tension or -- perhaps a better way to put it -- different flavours. This way of thinking can help you to use the different scale groups creatively.

Some other combinations might be:
\begin{quote}
	\begin{tabular}{l|l|l|l}
		Harmony & Consonant & Medium & Dissonant \\
		\hline
		7 & Mixolydian & Mixolydian $\flat$6 & Phrygian Major \\
		7 & Mixolydian & Lydian Dominant & Dorian $\sharp$4 \\
		$\Delta$ & Lydian & Lydian $\sharp$2 & Lydian Augmented \\
		$\Delta$ & Major & Melodic Minor & Augmented \\
		$\varnothing$ & Half Diminished & Locrian & Locrian $\natural$6 \\
		$\varnothing$ & Locrian & Super Locrian & Super Locrian $\flat\flat$7
	\end{tabular}
\end{quote}
but you should certainly also come up with your own.





\chapter{\mbox{Intervals}}

This brief chapter contains full CAGED diagrams for the intervals within the octave. This material is worthwhile for all guitarists, so it's included here among the basic resources. An interval can be considered simply as a two-note scale and, as with other scales, they have `modes', which in the case of intervals are usually called `inversions'. The diminished fifth is, in our terminology, a symmetrical scale and has no other modes in its group.

It's true that they're laid out slightly oddly here, but this material is very widely available in the standard form: my point here is to treat these two-note scales exactly like any other. If you don't find that enlightening or helpful then that's OK, just ignore this chapter, but make sure you \emph{do} know your intervals one way or another.

\subsection*{Minor Second / Major Seventh}
%~M|5|1,b2|Minor Second@Major Seventh%

\subsection*{Major Second / Minor Seventh}
%~M|5|1,2|Major Second@Minor Seventh%

\subsection*{Minor Third / Major Sixth}
%~M|5|1,b3|Minor Third@Major Sixth%

\subsection*{Major Third / Minor Sixth}
%~M|5|1,3|Major Third@Minor Sixth%

\subsection*{Perfect Fourth / Perfect Fifth}
%~M|5|1,4|Perfect Fourth@Perfect Fifth%

\subsection*{Diminished Fifth}
%~M|5|1,b5|Diminished Fifth%













\part{Advanced Resources}

\chapter{\mbox{Introducing} \mbox{the} \titlebreak \mbox{Advanced} \mbox{Resources}}

The scales collected in the remainder of this book represent the largest such collection ever assembled for the guitar or, to this writer's knowledge, for any other instrument. They include many scales that have never been studied, at least not systematically, and may even include some that have never been used at all. It's worth reading the introduction to each chapter but there's no need to study this material in any kind of sequence; the rest of this book offers resources for you to investigate creatively, not a course of study.

At the beginning we defined a scale as a gamut -- that is, a selection from the twelve available notes -- with a root note. There are 2048 ($2^{11}$) possible scales\footnote{This count includes the eleven two-note intervals and the `scale' that only contains the root note, but subtracting these doesn't make much of a difference to the figure.}, but not all are equally interesting. In order to keep this book to a manageable size and reduce the amount of `noise' it contains, a number of these 2048 scales have been excluded. The choice of which scales to exclude was not arbitrary, but was made according to a simple rule.

The rule states that any scale having an interval map that contains more than two consecutive semitones will be met with scepticism\footnote{It has occasionally been suggested that scales having \emph{any} consecutive semitones are musically suspect. Since this excludes a number of scales that have been explicitly used by great musicians in the past, as well as one or two rather common ones (the hexatonic Blues Scale being the most obvious) the author has not taken this path. Tymoczko gives a useful analytical survey of this topic from a jazz perspective.}. The usual reason for studying scales is to provide harmonic and melodic material. Although fragments of the chromatic scale can provide valid and interesting material, the truth is that the study of scales like this one is not very fruitful:
\begin{quote}
\begin{tabular}{rrrrrr}
1	&$\flat$2	&2	&$\flat$3	&$\flat$5	&7
\end{tabular}
\end{quote}
There's unlikely to be a huge difference between musical phrases created with this scale and those created using, say, this one, which omits the $\flat$2: 
\begin{quote}
\begin{tabular}{rrrrr}
1	&2	&$\flat$3	&$\flat$5	&7
\end{tabular}
\end{quote}
particularly given most musicians' tendency to add chromatic passing tones between scale notes. There are not a huge number of these but printing them out would make this book heavier than it needs to be to very little advantage. In fact quite a few cases that break this rule are included where the author has sighted the scale in some actual musical context, and in the case of heptatonics it seemed to be worth including all possible scales with up to three consecutive semitones. All of the excluded scale groups are listed at the back of the book (in Appendix A).

We begin with a chapter on the symmetrical scales. It's helpful to collect these together, and anyway the intermediate player will want to learn the Whole Tone and Whole-Half scale groups, so it seemed logical to deal with them first. We then have a chapter on `arpeggios' -- that is, scales having fewer than five notes, which we treat completely (that is, the chapter contains every possibility). We then progress to pentatonic scales, which we cover in their entirety once the first and second rules have been applied. The following four chapters tackle heptatonics; first we look at those that can be considered close relatives of the common heptatonics, since you may find these easier to learn, and the final chapter sweeps up the remainder. Finally we cover the non-symmetrical octatonics. The heptatonics and octatonics are also given complete coverage. The hexatonics, which the attentive reader will notice are missing from this list, are dealt with in Part III of this book. The reader is \emph{not} expected to work through this material in order. Simply open the book more or less at random and start exploring.


\section*{Note on Non-Western `Scales'}

Throughout the study of heptatonics the 72 South Indian \emph{melakatas} are referred to, partly as a source of convenient names for highly altered scales and partly because these patterns will be very useful for the student of Carnatic music. The \emph{melakatas} are worked out according to an altered-note principle similar to that employed here and are further extended to give scales such as the Hanumatodi $\flat$5. The fifth is never altered in a \emph{melakata}, and the fourth is never flattened; there are a few other examples, such as those involving sharpened thirds, that arise naturally when considering modes of these scales but that are not part of the Carnatic system.

The fact that the unaltered versions of these scales are in regular use in the musical tradition of a great civilisation perhaps provides some evidence to the sceptical that they are resources of genuine value rather than mere theoretical constructs. The \emph{melakatas} are not scales as such, although we may certainly take their notes, translate them into the Western equal-tempered tuning system and treat them as scales as we do in this book. Since these objects have been transformed in the process it seems more appropriate to refer to, say, the Rishabhapriya Scale than \emph{mela rishabhapriya}.

The student of Carnatic music will find it useful to know that the following \emph{melkakatas} have already been covered among the common heptatonic scale groups:
\begin{quote}
	\begin{tabular}{ll}
		Dhirasankarabharani	&	Major \\
		Kharaharapriya	&	Dorian \\
		Hanumatodi	&	Phrygian \\
		Dharmavati	&	Lydian \\
		Harikhamboji	&	Mixolydian \\
		Nathabhairavi	&	Aeolian \\
		Kiravani	&	Harmonic Minor \\
		Hemavati	&	Dorian $\sharp$4 \\
		Vakulabharanam	&	Phrygian Major \\
		Kosalam		&	Lydian $\sharp$2 \\
		Gaurimanohari	&	Melodic Minor \\
		Natakapriya	&	Dorian $\flat$2 \\
		Vachaspati	&	Lydian Dominant \\
		Charukesi	&	Mixolydian $\flat$6
	\end{tabular}
\end{quote}
The chapters that follow contain all of the remaining \emph{melakatas}, even those containing more than two consecutive semitones, as a service to the guitarist who is interested in Carnatic music. This book aims to cover all of the scales that might prove useful, and it was certainly felt that the \emph{melakatas}, coming as they do from such an estimable tradition, were all worthy of inclusion.

It is very common for scale books to give vague regional or ethnic names to scales, such as the `Gypsy', `Oriental', `Arabic' scales and so on; the idea of such associations fascinated many nineteenth-century composers and continue to interest heavy metal musicians, some of whom can be just as quaintly Romantic. The Phrygian Major, for instance is sometimes referred to as the `Jewish' scale, although that term might just as well apply to its mode the Dorian $\sharp$4, which is similar to the scale klezmer musicians know as \emph{mi scheberach}. But, really, who cares? Jewish people, like most people, make many different kinds of music and use various different scales while doing so.

There is, in fact, little agreement between authors about the scales to which such names are supposed to refer, and often enough they're treated as gimmicks and presented in a half-hearted manner that tends to shrug in the face of error and confusion. One author who shall remain nameless mentions that no Hungarians he has met seem to have heard of the Hungarian Minor Scale, so he rebaptizes it the `Gypsy Minor' for no clear reason. He goes on to describe a version of the Enigmatic scale with the wrong spelling and a fingerboard diagram that doesn't match either the correct spelling or the one he gives. This is symptomatic of the widespread practice of including fancy-looking scales merely for the sake of a little exotic colour or to create an illusion of erudition. If you don't think these resources are worth taking seriously then, quite honestly, one wonders why they're worth mentioning at all.

On the face of it there appears to be a stronger case for adopting the various Arabic scale-names, but the truth is that the underlying musical system is so different as to make the use of its terminology misleading. In Arabic music the idea of a `scale' as defined in this book doesn't really apply and the tuning system is importantly different from the Western one\footnote{See Marcus Scott-Lloyd's PhD thesis for the gory details.}. Renaming the Dorian $\flat$5 `Maqam Karcigar' might sound impressively knowledgeable but probably just demonstrates a lack of understanding. This same lack of parity applies to other musical traditions in which the tempered scale is not used or that do not recognise a Western-style tonal system. It applies, therefore, to the \emph{melakatas} as well, but the systematic working-out that characterises them is so close in spirit to the present book that the author has chosen, quite arbitrarily, to overlook it.


\chapter{\mbox{The} \mbox{Symmetrical} \mbox{Scales}}

Among the triad arpeggios, one stands out as unusual: the augmented arpeggio, whose fingering repeats itself exactly several times between the first and twelfth frets. This triad -- and no other -- behaves this way because it's symmetrical. In this chapter, we look more closely at what it means for a scale to be symmetrical and we describe every possible such scale aside from the augmented triad and the diminished seventh arpeggio, which we've already seen\footnote{In theory the two-note scale given by the interval of a diminished fifth (say, C G$\flat$) is also symmetrical, but we're not considering `scales' of only two notes.}. There's one other exception, the 7$\flat$5 arpeggio, which we'll meet in the next chapter.

Here are the interval maps of the four main triads:
\begin{quote}
\begin{tabular}{lllllll}
Major:	&1	&3	&5		&MT	&mT	&$4^{th}$\\
Minor:	&1	&$\flat 3$	&5		&mT	&MT	&$4^{th}$\\
Dim.:	&1	&$\flat 3$&$\flat 5$		&mT	&mT	&$\sharp 4^{th}$\\
Aug.:	&1	&3	&$\sharp 5$		&MT	&MT	&MT
\end{tabular}
\end{quote}
The augmented triad is the one that interests us; notice that its interval map consists of just three MT (two-tone) intervals. 

Say that we wanted to play an augmented triad in C. You'd start with the root, go up two tones to E, and go up two more to G$\sharp$. Going up two more tones takes us back to C again, and there's nothing unusual about any of this so far. Now, imagine you've learned the augmented triad and would now like to investigate its modes. The other triads have three modes, one being given by each note in turn being taken as the root. In the case of the augmented triad, however, whichever note is the root you always end up with the same interval map of MT MT MT, and therefore the same scale (recall from the introduction that a scale is defined by its interval map). This means that \emph{there is no scale that's a mode of the augmented triad}.

A scale that has fewer modes than the number of notes it contains is commonly known as a `symmetrical' scale\footnote{If you are wondering why the word `symmetrical' is used here, it has to do with the mathematics underlying the theory of modes that was mentioned very briefly in Chapter One.}. Here is a more complicated example, the Whole-Half-Tone Scale:
\begin{quote}
\begin{tabular}{llllllllllllllllll}
1	&&2	&&$\flat$3	&&4	&&$\flat$5	&&$\flat$6	&&6	&&7	&&1 \\
   &t     &&  s          &&t         &&s         &&t              &&s        &&t    &&s  &  
\end{tabular}
\end{quote}
As its name implies, the interval map alternates between tones and semitones. This means that the scale repeats exactly the same notes for every alternate position in the pattern: starting on the b3 will give you just the same notes as you started with. Starting on the 2, though, will give you a different scale. Although this is an eight-note scale, there's only one scale that's a mode of it, the aptly-named Half-Whole-Tone Scale. For most eight-note scales, of course, there will be eight scales in their modal group, one for each note in the scale.

Because they tend to have more repetitive patterns, symmetrical scales are generally easier to learn than non-symmetrical ones. When working through them, then, it's advisable to look harder than usual for the recurring patterns in the fingerings. The payoff is that they have fewer modes and are therefore less flexible. Only the first two of the scale groups in this chapter are commonly used. The others have been used only rarely, but are worth checking out. As you might expect, augmented and diminished sounds are found in these scales much more commonly than major or minor.

\section{The Whole-Tone Scale}
%~M|5|1,2,3,#4,#5,b7|Whole-Tone%

The Whole-Tone scale was a favourite of composer Claude Debussy, and Scriabin introduced a new chord based on it that's become known as the `Scriabin sixth', which was called an `augmented $\flat$7' in Chapter Three. Note that the Scriabin Sixth is not a symmetrical arpeggio; it has four notes and four modes, as you can easily check for yourself. George Russell includes this among the `primary scales' of his Lydian Chromatic system, calling it the Auxiliary Augmented Scale, and it can sound great over a dominant 7 chord, especially if it has a $\sharp$5 (a common alteration when it's resolving up a fourth).

\section{The Whole-Half Scale Group}
The Whole-Half is also known as the Diminished scale, although this is ambiguous because there are many scales that contain a diminished seventh chord including both modes of this scale. The half-whole scale is, though, often used to play over such chords. George Russell refers to the Whole-Half Scale as the Auxiliary Diminished and the Half-Whole, which contains the $\flat3$-3 tension, as the Auxiliary Diminished Blues Scale. The Half-Whole is more commonly heard in a dominant 7 or blues kind of context, where as Russell's terminology suggests you get a lot of different `blue' notes. To my ears, whole-half sounds great over minor and half-diminished chords.

Both scales are in general use and most guitarists who know some scales can usually remember these patterns. They're also very common in heavy metal music, perhaps because of the numerous tritones they contain. You can also use them in the same way as the Whole-Tone, as vehicles for modulation into distant keys because they contain diminished arpeggios that, like the augmented, tend to resolve up or down a semitone.

%~M|5|1,2,b3,4,b5,b6,6,7|Whole-Half@Half-Whole%


\section{The $4^{\text{min}}$ + $7^{\text{min}}$ Scale Group}
The scales in this group are actually subsets of either the Half-Whole or Whole-half Diminished scales we just looked at.
%~M|5|1,2,4,b5,b6,7|$4^{\text{min}}$ + $7^{\text{min}}$
@$\flat 3^{\text{min}}$ + $6^{\text{min}}$
@$1^{\text{min}}$ + $\flat 5^{\text{min}}$
%

\section{The $2^{\text{maj}}$ + $\sharp 5^{\text{maj}}$ Scale Group}
These scales have many similarities to those in the previous group, although they are generally more dissonant. The $1^{\text{maj}}$ + $\flat 5^{\text{maj}}$ Scale is simply the major superimposition at the $\flat$5, which is related to what is commonly known in jazz as the `tritone substitution'.

%~M|5|1,2,b3,#4,#5,6|$2^{\text{maj}}$ + $\sharp 5^{\text{maj}}$@$1^{\text{maj}}$ + $\flat 5^{\text{maj}}$@$4^{\text{maj}}$ + $7^{\text{maj}}$%


\section{The $1^{\text{dom}}$ + $\flat 6^{\text{maj}}$ Scale Group}

This group is similar to the two above, except that it contains only two scales. It can be thought of as either a dominant chord plus a major triad or a pair of augmented triads. Both scales share a distinctive quality resulting from their combinations of minor thirds and semitones. Jason Lyons has claimed -- a bit mischievously, I think -- that this scale can be used to play over the whole `Giant Steps' progression (see http://www.opus28.co.uk/gsteps.pdf).

%~M|5|1,#2,3,5,b6,7|$1^{\text{dom}}$ + $\flat 6^{\text{maj}}$@$1^{\text{aug}}$ + $\flat 2^{\text{aug}}$%


\section{The Tritone Chromatic Scale Group}

%~M|5|1,b2,2,b5,5,b6|Tritone Chromatic I@Tritone Chromatic II@Tritone Chromatic III%

\section{The $1^{\text{aug}}$ + $\flat 7^{\text{maj}}$  + $7^{\text{min}}$ Scale Group}

There are four modes in this group, as a result of the fact that it's `less symmetrical' than the others. These are all extremely dissonant scales, which may take some getting used to if you want to make use of them.

%~M|5|1,2,3,4,b5,b6,b7,7|$1^{\text{aug}}$ + $\flat 7^{\text{maj}}$  + $7^{\text{min}}$@$1^{\text{aug}}$ + $1^\varnothing$  + $2^{\text{maj}}$@$1^{\text{maj}}$ + $\flat 2^{\text{min}}$  + $2^{\text{aug}}$@$1^{\text{min}}$ + $\flat 2^{\text{aug}}$  + $7^{\text{maj}}$%

\section{The Double Chromatic Scale}
The Double Chromatic is an exception to the no-more-than-two-consecutive-semitones rule; it's here purely for completeness. It does, though, provide an alternative to undiluted chromatic runs, which can sound very dull.
%~M|5|1,#2,3,4,b5,6,b7,7|Double Chromatic I@Double Chromatic II@Double Chromatic III@Double Chromatic IV@Double Chromatic V%

\section{The Triple Chromatic Scale}
This is a nonatonic (nine-note) scale with two other modes in its group. Perhaps surprisingly, it's more useable than the double chromatic because it consists of only three-semitone groups (as opposed to four semitones together, as in the double chromatic). Try both scales on the fingerboard to hear and see this difference. Allan Holdsworth has included the Tripe Chromatic I in a list of scales he likes to use.

%~M|5|1,b2,2,3,4,b5,b6,6,b7|Triple Chromatic I@Triple Chromatic II@Triple Chromatic III%

\section{The Tritone Coscale}
This scale is another exception to the rule regarding consecutive semitones: it's a chromatic scale with two notes a tritone apart removed\footnote{For the meaning of the term `coscale', see Part III.}. This is probably not a very useful scale but it's included here for the sake of completeness.

%~M|5|1,2,b3,3,4,b5,b6,6,b7,7|Tritone Coscale I@Tritone Coscale II@Tritone Coscale III@Tritone Coscale IV@Tritone Coscale V@Tritone Coscale VI@%




\chapter{\mbox{Further} \mbox{Arpeggios}}

Having gathered the symmetrical together in the previous chapter, we now progress through scales of four, five, seven and finally eight notes for the remainder of this book. This chapter is mostly about `arpeggios', that is, scales having four (or occasionally just three) notes. 

We'll approach the material starting with the most naturally `arpeggio-like' sounds -- that is, common triads with added or altered notes. Beginning with the four basic triads, we look at altering the fifth and adding different sevenths to them. We then look at adding other notes -- the ninth, eleventh and thirteenth, which are just the second, fourth and sixth with different names. Finally we look at arpeggios built on suspended triads. We therefore gradually move from common to rare and unusual sounds. 

Note that all of these sounds are found as components of larger structures such as pentatonic or heptatonic scales covered in subsequent chapters. These may prove richer and more rewarding for serious study, but the arpeggios in this chapter can be used in various ways. On the one hand, the more arpeggios you know the easier it is to learn larger scales that contain them, because there are fewer extra notes to memorise. On the other hand, looking at an arpeggio in isolation often gives you a different view of a scale you already know, encouraging you to think of new ways to use it.


\section{Seventh Arpeggios}

\subsection*{$\Delta$ $\flat$5}

\[
\begin{array}{ll}
	\begin{array}{c}
		\begin{array}{ccc}
			%~C1,NoDisplay|5|7%   
			&%~A1,NoDisplay|5|7%   
			&%~G1,NoDisplay|5|7%
		\end{array}
		\\
		\begin{array}{cc}
			%~E1,NoDisplay|5|7%
			&%~D1,NoDisplay|5|7%
		\end{array}
	\end{array}

&
	\text{
		%~T|0|1,3,b5,7%
	}
\end{array}
\]


\subsection*{7 $\flat$5}

It's a little-known fact that this arpeggio is `symmetrical', just like the augmented or diminished seventh arpeggios. Its modes start repeating halfway through, which is the telltale sign, but notice also the repetition in the fingerboard diagrams.

\[
\begin{array}{ll}
	\begin{array}{c}
		\begin{array}{ccc}
			%~C1,NoDisplay|5|b7%   
			&%~A1,NoDisplay|5|b7%   
			&%~G1,NoDisplay|5|b7%
		\end{array}
		\\
		\begin{array}{cc}
			%~E1,NoDisplay|5|b7%
			&%~D1,NoDisplay|5|b7%
		\end{array}
	\end{array}

&
	\text{
		%~T|0|1,3,b5,b7%
	}
\end{array}
\]


\subsection*{Minor $\natural$7}

\[
\begin{array}{ll}
	\begin{array}{c}
		\begin{array}{ccc}
			%~Cm,NoDisplay|5|7%   
			&%~Am,NoDisplay|5|7%   
			&%~Gm,NoDisplay|5|7%
		\end{array}
		\\
		\begin{array}{cc}
			%~Em,NoDisplay|5|7%
			&%~Dm,NoDisplay|5|7%
		\end{array}
	\end{array}

&
	\text{
		%~T|0|1,b3,5,7%
	}
\end{array}
\]

\subsection*{Diminished $\natural$7}

\[
\begin{array}{ll}
	\begin{array}{c}
		\begin{array}{ccc}
			%~Cd,NoDisplay|5|7%   
			&%~Ad,NoDisplay|5|7%   
			&%~Gd,NoDisplay|5|7%
		\end{array}
		\\
		\begin{array}{cc}
			%~Ed,NoDisplay|5|7%
			&%~Dd,NoDisplay|5|7%
		\end{array}
	\end{array}

&
	\text{
		%~T|0|1,b3,b5,7%
	}
\end{array}
\]


\subsection*{Augmented $\natural$7}

\[
\begin{array}{ll}
	\begin{array}{c}
		\begin{array}{ccc}
			%~Ca,NoDisplay|5|7%   
			&%~Aa,NoDisplay|5|7%   
			&%~Ga,NoDisplay|5|7%
		\end{array}
		\\
		\begin{array}{cc}
			%~Ea,NoDisplay|5|7%
			&%~Da,NoDisplay|5|7%
		\end{array}
	\end{array}

&
	\text{
		%~T|0|1,3,#5,7%
	}
\end{array}
\]


\subsection*{Augmented $\flat$7}

The Augmented Add 9 is given by playing this arpeggio at the 2 and is sometimes known as the `Whole-Tone Tetrachord'.

\[
\begin{array}{ll}
	\begin{array}{c}
		\begin{array}{ccc}
			%~Ca,NoDisplay|5|b7%   
			&%~Aa,NoDisplay|5|b7%   
			&%~Ga,NoDisplay|5|b7%
		\end{array}
		\\
		\begin{array}{cc}
			%~Ea,NoDisplay|5|b7%
			&%~Da,NoDisplay|5|b7%
		\end{array}
	\end{array}

&
	\text{
		%~T|0|1,3,#5,b7%
	}
\end{array}
\]


\section{Added Ninths, Elevenths and Thirteenths}

As well as adding natural and flattened sevenths to a triad, there are a number of other possibilities. Of the notes in the major scale, the triad contains 1, 3 and 5 (of which the 3 and 5 may be altered). We have discussed the seventh; the 2, 4 and 6 remain. For historical reasons, when we are talking about chords it is common to call these notes the 9, 11 and 13. Since we're still talking about arpeggios we'll follow this practice here, although it won't be followed in later chapters when we're talking about ordinary scales.

We would, of course, like to know what all of the possibilities are. The ninth can be flattened, natural or, if the third is not flattened, the ninth can be sharpened. The eleventh can be natural, flat if the third is flat, and sharp if the fifth is \emph{not} flat, but we do have to be careful about duplication. The augmented with an added $\sharp$9 is, for example, a mode of the augmented $\natural$7; the fingering diagrams will be the same except for the position of the root, so there is really no point in including both. Likewise, the major with an added $\flat$9 is a mode of the diminished $\natural$7, and the augmented $\flat$9 is a mode of the minor $\natural$7 (maybe you see a pattern here).

The final possibility is the sixth, sometimes called the thirteenth when talking about chord- rather than scale-tones. All of the possibilities for the natural sixth have already been covered. The sixth added to the diminished triad gives a fully diminished chord. Adding it to the minor triad gives 1 $\flat$3 5 6, which is a mode of the half-diminished arpeggio, while adding it to the augmented triad gives a mode of the minor $\natural$7. The only remaining possibility is the major triad, but that too results in a mode of a common arpeggio, the minor seventh.

The sharp sixth is, of course, the same as a flat seventh, so that's covered above. The flat sixth can be added to any triad except the augmented (because $\flat$6 = $\sharp$5), but all of the possibilities will already have been covered modally by the time we get to that point. Studying these extended chord tones is a worthwhile exercise because knowing exactly which chord tone you're playing at any given moment, in any position, can be extremely valuable during improvisation.

\subsection*{Major Add 9}

\[
\begin{array}{ll}
	\begin{array}{c}
		\begin{array}{ccc}
			%~C,NoDisplay|5|2%   
			&%~A,NoDisplay|5|2%   
			&%~G,NoDisplay|5|2%
		\end{array}
		\\
		\begin{array}{cc}
			%~E,NoDisplay|5|2%
			&%~D,NoDisplay|5|2%
		\end{array}
	\end{array}

&
	\text{
		%~T|0|1,2,3,5%
	}
\end{array}
\]

\subsection*{Minor Add 9}

\[
\begin{array}{ll}
	\begin{array}{c}
		\begin{array}{ccc}
			%~Cm,NoDisplay|5|2%   
			&%~Am,NoDisplay|5|2%   
			&%~Gm,NoDisplay|5|2%
		\end{array}
		\\
		\begin{array}{cc}
			%~Em,NoDisplay|5|2%
			&%~Dm,NoDisplay|5|2%
		\end{array}
	\end{array}

&
	\text{
		%~T|0|1,2,b3,5%
	}
\end{array}
\]


\subsection*{Diminished Add 9}

\[
\begin{array}{ll}
	\begin{array}{c}
		\begin{array}{ccc}
			%~Cd,NoDisplay|5|2%   
			&%~Ad,NoDisplay|5|2%   
			&%~Gd,NoDisplay|5|2%
		\end{array}
		\\
		\begin{array}{cc}
			%~Ed,NoDisplay|5|2%
			&%~Dd,NoDisplay|5|2%
		\end{array}
	\end{array}

&
	\text{
		%~T|0|1,2,b3,b5%
	}
\end{array}
\]

\subsection*{Minor Add $\flat$9}

\[
\begin{array}{ll}
	\begin{array}{c}
		\begin{array}{ccc}
			%~Cm,NoDisplay|5|b2%   
			&%~Am,NoDisplay|5|b2%   
			&%~Gm,NoDisplay|5|b2%
		\end{array}
		\\
		\begin{array}{cc}
			%~Em,NoDisplay|5|b2%
			&%~Dm,NoDisplay|5|b2%
		\end{array}
	\end{array}

&
	\text{
		%~T|0|1,b2,b3,5%
	}
\end{array}
\]


\subsection*{Diminished Add $\flat$9}

\[
\begin{array}{ll}
	\begin{array}{c}
		\begin{array}{ccc}
			%~Cd,NoDisplay|5|b2%   
			&%~Ad,NoDisplay|5|b2%   
			&%~Gd,NoDisplay|5|b2%
		\end{array}
		\\
		\begin{array}{cc}
			%~Ed,NoDisplay|5|b2%
			&%~Dd,NoDisplay|5|b2%
		\end{array}
	\end{array}

&
	\text{
		%~T|0|1,b2,b3,b5%
	}
\end{array}
\]

\subsection*{Major Add $\sharp$9}

\[
\begin{array}{ll}
	\begin{array}{c}
		\begin{array}{ccc}
			%~C,NoDisplay|5|#2%   
			&%~A,NoDisplay|5|#2%   
			&%~G,NoDisplay|5|#2%
		\end{array}
		\\
		\begin{array}{cc}
			%~E,NoDisplay|5|#2%
			&%~D,NoDisplay|5|#2%
		\end{array}
	\end{array}

&
	\text{
		%~T|0|1,#2,3,5%
	}
\end{array}
\]

\subsection*{Major Add 11}

\[
\begin{array}{ll}
	\begin{array}{c}
		\begin{array}{ccc}
			%~C,NoDisplay|5|4%   
			&%~A,NoDisplay|5|4%   
			&%~G,NoDisplay|5|4%
		\end{array}
		\\
		\begin{array}{cc}
			%~E,NoDisplay|5|4%
			&%~D,NoDisplay|5|4%
		\end{array}
	\end{array}

&
	\text{
		%~T|0|1,4,3,5%
	}
\end{array}
\]

\subsection*{Minor Add 11}

\[
\begin{array}{ll}
	\begin{array}{c}
		\begin{array}{ccc}
			%~Cm,NoDisplay|5|4%   
			&%~Am,NoDisplay|5|4%   
			&%~Gm,NoDisplay|5|4%
		\end{array}
		\\
		\begin{array}{cc}
			%~Em,NoDisplay|5|4%
			&%~Dm,NoDisplay|5|4%
		\end{array}
	\end{array}

&
	\text{
		%~T|0|1,4,b3,5%
	}
\end{array}
\]


\subsection*{Diminished Add 11}

\[
\begin{array}{ll}
	\begin{array}{c}
		\begin{array}{ccc}
			%~Cd,NoDisplay|5|4%   
			&%~Ad,NoDisplay|5|4%   
			&%~Gd,NoDisplay|5|4%
		\end{array}
		\\
		\begin{array}{cc}
			%~Ed,NoDisplay|5|4%
			&%~Dd,NoDisplay|5|4%
		\end{array}
	\end{array}

&
	\text{
		%~T|0|1,4,b3,b5%
	}
\end{array}
\]

\subsection*{Major Add $\sharp$11}

\[
\begin{array}{ll}
	\begin{array}{c}
		\begin{array}{ccc}
			%~C,NoDisplay|5|#4%   
			&%~A,NoDisplay|5|#4%   
			&%~G,NoDisplay|5|#4%
		\end{array}
		\\
		\begin{array}{cc}
			%~E,NoDisplay|5|#4%
			&%~D,NoDisplay|5|#4%
		\end{array}
	\end{array}

&
	\text{
		%~T|0|1,#4,3,5%
	}
\end{array}
\]

\subsection*{Minor Add $\sharp$11}

\[
\begin{array}{ll}
	\begin{array}{c}
		\begin{array}{ccc}
			%~Cm,NoDisplay|5|#4%   
			&%~Am,NoDisplay|5|#4%   
			&%~Gm,NoDisplay|5|#4%
		\end{array}
		\\
		\begin{array}{cc}
			%~Em,NoDisplay|5|#4%
			&%~Dm,NoDisplay|5|#4%
		\end{array}
	\end{array}

&
	\text{
		%~T|0|1,#4,b3,5%
	}
\end{array}
\]


\subsection*{Minor Add $\flat$11}

\[
\begin{array}{ll}
	\begin{array}{c}
		\begin{array}{ccc}
			%~Cm,NoDisplay|5|b4%   
			&%~Am,NoDisplay|5|b4%   
			&%~Gm,NoDisplay|5|b4%
		\end{array}
		\\
		\begin{array}{cc}
			%~Em,NoDisplay|5|b4%
			&%~Dm,NoDisplay|5|b4%
		\end{array}
	\end{array}

&
	\text{
		%~T|0|1,b3,b4,5%
	}
\end{array}
\]


\subsection*{Diminished Add $\flat$11}

\[
\begin{array}{ll}
	\begin{array}{c}
		\begin{array}{ccc}
			%~Cd,NoDisplay|5|b4%   
			&%~Ad,NoDisplay|5|b4%   
			&%~Gd,NoDisplay|5|b4%
		\end{array}
		\\
		\begin{array}{cc}
			%~Ed,NoDisplay|5|b4%
			&%~Dd,NoDisplay|5|b4%
		\end{array}
	\end{array}

&
	\text{
		%~T|0|1,b3,b4,b5%
	}
\end{array}
\]


\section{Suspensions}

In addition to the standard triads we sometimes also find what are called `suspended' triads. In these the 3 has been replaced by either the 2 or the 4. These are a fairly common sound in jazz and pop music, although all of the four-note arpeggios below are rare except for the sevenths. One nice fact is that the sus 4 arpeggio is a mode of the sus 2, so there's only one basic set of shapes to learn.

\subsection*{The Suspended Triads}

The first mode shown is the sus 2; the second one (spelled 1 4 5) is the sus 4.

\[
\begin{array}{ll}
	\begin{array}{c}
		\begin{array}{ccc}
			%~C2,NoDisplay|5|%   
			&%~A2,NoDisplay|5|%   
			&%~G2,NoDisplay|5|%
		\end{array}
		\\
		\begin{array}{cc}
			%~E2,NoDisplay|5|%
			&%~D2,NoDisplay|5|%
		\end{array}
	\end{array}

&
	\text{
		%~T|0|1,2,5%
	}
\end{array}
\]

\subsection*{Sus 4 $\Delta$}

\[
\begin{array}{ll}
	\begin{array}{c}
		\begin{array}{ccc}
			%~C4,NoDisplay|5|7%   
			&%~A4,NoDisplay|5|7%   
			&%~G4,NoDisplay|5|7%
		\end{array}
		\\
		\begin{array}{cc}
			%~E4,NoDisplay|5|7%
			&%~D4,NoDisplay|5|7%
		\end{array}
	\end{array}

&
	\text{
		%~T|0|1,4,5,7%
	}
\end{array}
\]

\subsection*{Sus 4 $\flat$7}

\[
\begin{array}{ll}
	\begin{array}{c}
		\begin{array}{ccc}
			%~C4,NoDisplay|5|b7%   
			&%~A4,NoDisplay|5|b7%   
			&%~G4,NoDisplay|5|b7%
		\end{array}
		\\
		\begin{array}{cc}
			%~E4,NoDisplay|5|b7%
			&%~D4,NoDisplay|5|b7%
		\end{array}
	\end{array}

&
	\text{
		%~T|0|1,4,5,7%
	}
\end{array}
\]

\subsection*{Sus 4 Add 9}

\[
\begin{array}{ll}
	\begin{array}{c}
		\begin{array}{ccc}
			%~C4,NoDisplay|5|2%   
			&%~A4,NoDisplay|5|2%   
			&%~G4,NoDisplay|5|2%
		\end{array}
		\\
		\begin{array}{cc}
			%~E4,NoDisplay|5|2%
			&%~D4,NoDisplay|5|2%
		\end{array}
	\end{array}

&
	\text{
		%~T|0|1,2,4,5%
	}
\end{array}
\]

\subsection*{Sus 2 Add $\flat$9}

\[
\begin{array}{ll}
	\begin{array}{c}
		\begin{array}{ccc}
			%~C2,NoDisplay|5|b2%   
			&%~A2,NoDisplay|5|b2%   
			&%~G2,NoDisplay|5|b2%
		\end{array}
		\\
		\begin{array}{cc}
			%~E2,NoDisplay|5|b2%
			&%~D2,NoDisplay|5|b2%
		\end{array}
	\end{array}

&
	\text{
		%~T|0|1,b2,2,5%
	}
\end{array}
\]

It's possible to take this `suspension' idea a step further. What if we invented new chords like the Sus $\flat$2 or Sus $\sharp$4? There's nothing to stop us from doing this.

\subsection*{Sus $\flat$2}

\[
\begin{array}{ll}
	\begin{array}{c}
		\begin{array}{ccc}
			%~C0,NoDisplay|5|b2,5%   
			&%~A0,NoDisplay|5|b2,5%   
			&%~G0,NoDisplay|5|b2,5%
		\end{array}
		\\
		\begin{array}{cc}
			%~E0,NoDisplay|5|b2,5%
			&%~D0,NoDisplay|5|b2,5%
		\end{array}
	\end{array}

&
	\text{
		%~T|0|1,b2,5%
	}
\end{array}
\]

\subsection*{Sus $\sharp$4}

\[
\begin{array}{ll}
	\begin{array}{c}
		\begin{array}{ccc}
			%~C0,NoDisplay|5|#4,5%   
			&%~A0,NoDisplay|5|#4,5%   
			&%~G0,NoDisplay|5|#4,5%
		\end{array}
		\\
		\begin{array}{cc}
			%~E0,NoDisplay|5|#4,5%
			&%~D0,NoDisplay|5|#4,5%
		\end{array}
	\end{array}

&
	\text{
		%~T|0|1,#4,5%
	}
\end{array}
\]

\subsection*{Sus $\flat$2 $\Delta$}

\[
\begin{array}{ll}
	\begin{array}{c}
		\begin{array}{ccc}
			%~C0,NoDisplay|5|b2,5,7%   
			&%~A0,NoDisplay|5|b2,5,7%   
			&%~G0,NoDisplay|5|b2,5,7%
		\end{array}
		\\
		\begin{array}{cc}
			%~E0,NoDisplay|5|b2,5,7%
			&%~D0,NoDisplay|5|b2,5,7%
		\end{array}
	\end{array}

&
	\text{
		%~T|0|1,b2,5,7%
	}
\end{array}
\]

\subsection*{Sus $\sharp$4 $\Delta$}

\[
\begin{array}{ll}
	\begin{array}{c}
		\begin{array}{ccc}
			%~C0,NoDisplay|5|#4,5,7%   
			&%~A0,NoDisplay|5|#4,5,7%   
			&%~G0,NoDisplay|5|#4,5,7%
		\end{array}
		\\
		\begin{array}{cc}
			%~E0,NoDisplay|5|#4,5,7%
			&%~D0,NoDisplay|5|#4,5,7%
		\end{array}
	\end{array}

&
	\text{
		%~T|0|1,#4,5,7%
	}
\end{array}
\]

\subsection*{Sus $\sharp$4 $\flat$7}

\[
\begin{array}{ll}
	\begin{array}{c}
		\begin{array}{ccc}
			%~C0,NoDisplay|5|#4,5,b7%   
			&%~A0,NoDisplay|5|#4,5,b7%   
			&%~G0,NoDisplay|5|#4,5,b7%
		\end{array}
		\\
		\begin{array}{cc}
			%~E0,NoDisplay|5|#4,5,b7%
			&%~D0,NoDisplay|5|#4,5,b7%
		\end{array}
	\end{array}

&
	\text{
		%~T|0|1,#4,5,b7%
	}
\end{array}
\]

\subsection*{Sus $\flat$2 Add $\sharp$11}

\[
\begin{array}{ll}
	\begin{array}{c}
		\begin{array}{ccc}
			%~C0,NoDisplay|5|b2,#4,5%   
			&%~A0,NoDisplay|5|b2,#4,5%   
			&%~G0,NoDisplay|5|b2,#4,5%
		\end{array}
		\\
		\begin{array}{cc}
			%~E0,NoDisplay|5|b2,#4,5%
			&%~D0,NoDisplay|5|b2,#4,5%
		\end{array}
	\end{array}

&
	\text{
		%~T|0|1,b2,#4,5%
	}
\end{array}
\]



\subsection*{Sus $\flat$2 Add $\flat$13}

\[
\begin{array}{ll}
	\begin{array}{c}
		\begin{array}{ccc}
			%~C0,NoDisplay|5|b2,5,b6%   
			&%~A0,NoDisplay|5|b2,5,b6%   
			&%~G0,NoDisplay|5|b2,5,b6%
		\end{array}
		\\
		\begin{array}{cc}
			%~E0,NoDisplay|5|b2,5,b6%
			&%~D0,NoDisplay|5|b2,5,b6%
		\end{array}
	\end{array}

&
	\text{
		%~T|0|1,b2,5,b6%
	}
\end{array}
\]

\subsection*{Sus $\sharp$4 $\flat$13}

\[
\begin{array}{ll}
	\begin{array}{c}
		\begin{array}{ccc}
			%~C0,NoDisplay|5|#4,5,b6%   
			&%~A0,NoDisplay|5|#4,5,b6%   
			&%~G0,NoDisplay|5|#4,5,b6%
		\end{array}
		\\
		\begin{array}{cc}
			%~E0,NoDisplay|5|#4,5,b6%
			&%~D0,NoDisplay|5|#4,5,b6%
		\end{array}
	\end{array}

&
	\text{
		%~T|0|1,#4,5,b6%
	}
\end{array}
\]

\subsection*{Sus $\flat$2 Add 13}

\[
\begin{array}{ll}
	\begin{array}{c}
		\begin{array}{ccc}
			%~C0,NoDisplay|5|b2,5,6%   
			&%~A0,NoDisplay|5|b2,5,6%   
			&%~G0,NoDisplay|5|b2,5,6%
		\end{array}
		\\
		\begin{array}{cc}
			%~E0,NoDisplay|5|b2,5,6%
			&%~D0,NoDisplay|5|b2,5,6%
		\end{array}
	\end{array}

&
	\text{
		%~T|0|1,b2,5,6%
	}
\end{array}
\]

\subsection*{Sus $\sharp$4 13}

\[
\begin{array}{ll}
	\begin{array}{c}
		\begin{array}{ccc}
			%~C0,NoDisplay|5|#4,5,6%   
			&%~A0,NoDisplay|5|#4,5,6%   
			&%~G0,NoDisplay|5|#4,5,6%
		\end{array}
		\\
		\begin{array}{cc}
			%~E0,NoDisplay|5|#4,5,6%
			&%~D0,NoDisplay|5|#4,5,6%
		\end{array}
	\end{array}

&
	\text{
		%~T|0|1,#4,5,6%
	}
\end{array}
\]

\section{Remaining Suspended Possibilities}

There are a number of other four-note arpeggios that can be formed from the suspended triads, but they're all modes of arpeggios already covered. In case you're interested in exploring these sounds further, the following table indicates where they can be found:

\begin{tabular}{lll}
	The Sus 2 Add $\sharp$11 &is a mode of the& Sus 4 $\Delta$ \\
	The Sus 2 $\Delta$ &is a mode of the& Major Add 11\\
	The Sus 2 $\flat$7 &is a mode of the& Minor Add 11\\
	The Sus 4 Add $\sharp$11 &is a mode of the& Sus 2 Add $\flat$9 \\
	The Sus 4 Add $\flat$9 &is a mode of the& $\Delta\flat$5\\
	The Sus 2 Add $\flat$13 &is a mode of the& $\Delta\flat$5\\
	The Sus 4 Add $\sharp$13 &is a mode of the& Minor Add 9\\
	The Sus 2 Add 13 &is a mode of the& Sus 4 $\flat$7\\
	The Sus 4 Add 13 &is a mode of the& Major Add 9\\
	The Sus $\flat$2 $\flat$7 &is a mode of the& $\varnothing$ Add 11 \\
	The Sus $\sharp$4 Add $\flat$2 &is a mode of the& Sus $\flat$2 Add $\sharp$11
\end{tabular}

\section{The Remaining Triad Possibilities}

There are twelve remaining possibilities for three-note scale groups not already covered. Most do not have very natural names, so in this section we use a simple numerical description. These groups are studied by musicians who work with certain kinds of atonal music, in which pitch relationships such as these often play an important part. Using them in different positions to contribute to the logical structure of a solo is a possibility, but you'll really need to experiment for yourself with these.

\subsection*{Containing Two Semitones}

\subsection*{$1,\flat2,2$}
\[
\begin{array}{ll}
	\begin{array}{c}
		\begin{array}{ccc}
			%~C0,NoDisplay|5|b2,2%   
			&%~A0,NoDisplay|5|b2,2%   
			&%~G0,NoDisplay|5|b2,2%
		\end{array}
		\\
		\begin{array}{cc}
			%~E0,NoDisplay|5|b2,2%
			&%~D0,NoDisplay|5|b2,2%
		\end{array}
	\end{array}

&
	\text{
		%~T|0|1,b2,2%
	}
\end{array}
\]


\subsection*{Containing One Semitone}


\subsection*{$1,\flat2,\flat3$}
\[
\begin{array}{ll}
	\begin{array}{c}
		\begin{array}{ccc}
			%~C0,NoDisplay|5|b2,b3%   
			&%~A0,NoDisplay|5|b2,b3%   
			&%~G0,NoDisplay|5|b2,b3%
		\end{array}
		\\
		\begin{array}{cc}
			%~E0,NoDisplay|5|b2,b3%
			&%~D0,NoDisplay|5|b2,b3%
		\end{array}
	\end{array}

&
	\text{
		%~T|0|1,b2,b3%
	}
\end{array}
\]

\subsection*{$1,2,\flat3$}
\[
\begin{array}{ll}
	\begin{array}{c}
		\begin{array}{ccc}
			%~C0,NoDisplay|5|2,b3%   
			&%~A0,NoDisplay|5|2,b3%   
			&%~G0,NoDisplay|5|2,b3%
		\end{array}
		\\
		\begin{array}{cc}
			%~E0,NoDisplay|5|2,b3%
			&%~D0,NoDisplay|5|2,b3%
		\end{array}
	\end{array}

&
	\text{
		%~T|0|1,2,b3%
	}
\end{array}
\]

\subsection*{$1,\flat2,3$}
\[
\begin{array}{ll}
	\begin{array}{c}
		\begin{array}{ccc}
			%~C0,NoDisplay|5|b2,3%   
			&%~A0,NoDisplay|5|b2,3%   
			&%~G0,NoDisplay|5|b2,3%
		\end{array}
		\\
		\begin{array}{cc}
			%~E0,NoDisplay|5|b2,3%
			&%~D0,NoDisplay|5|b2,3%
		\end{array}
	\end{array}

&
	\text{
		%~T|0|1,b2,3%
	}
\end{array}
\]

\subsection*{$1,\sharp2,3$}
\[
\begin{array}{ll}
	\begin{array}{c}
		\begin{array}{ccc}
			%~C0,NoDisplay|5|#2,3%   
			&%~A0,NoDisplay|5|#2,3%   
			&%~G0,NoDisplay|5|#2,3%
		\end{array}
		\\
		\begin{array}{cc}
			%~E0,NoDisplay|5|#2,3%
			&%~D0,NoDisplay|5|#2,3%
		\end{array}
	\end{array}

&
	\text{
		%~T|0|1,#2,3%
	}
\end{array}
\]


\subsection*{$1,\flat2,4$}
\[
\begin{array}{ll}
	\begin{array}{c}
		\begin{array}{ccc}
			%~C0,NoDisplay|5|b2,4%   
			&%~A0,NoDisplay|5|b2,4%   
			&%~G0,NoDisplay|5|b2,4%
		\end{array}
		\\
		\begin{array}{cc}
			%~E0,NoDisplay|5|b2,4%
			&%~D0,NoDisplay|5|b2,4%
		\end{array}
	\end{array}

&
	\text{
		%~T|0|1,b2,4%
	}
\end{array}
\]

\subsection*{$1,3,4$}
\[
\begin{array}{ll}
	\begin{array}{c}
		\begin{array}{ccc}
			%~C0,NoDisplay|5|3,4%   
			&%~A0,NoDisplay|5|3,4%   
			&%~G0,NoDisplay|5|3,4%
		\end{array}
		\\
		\begin{array}{cc}
			%~E0,NoDisplay|5|3,4%
			&%~D0,NoDisplay|5|3,4%
		\end{array}
	\end{array}

&
	\text{
		%~T|0|1,3,4%
	}
\end{array}
\]

\subsection*{Without Semitones}

\subsection*{$1,2,3$}
\[
\begin{array}{ll}
	\begin{array}{c}
		\begin{array}{ccc}
			%~C0,NoDisplay|5|2,3%   
			&%~A0,NoDisplay|5|2,3%   
			&%~G0,NoDisplay|5|2,3%
		\end{array}
		\\
		\begin{array}{cc}
			%~E0,NoDisplay|5|2,3%
			&%~D0,NoDisplay|5|2,3%
		\end{array}
	\end{array}

&
	\text{
		%~T|0|1,2,3%
	}
\end{array}
\]


\subsection*{$1,2,4$}
\[
\begin{array}{ll}
	\begin{array}{c}
		\begin{array}{ccc}
			%~C0,NoDisplay|5|2,4%   
			&%~A0,NoDisplay|5|2,4%   
			&%~G0,NoDisplay|5|2,4%
		\end{array}
		\\
		\begin{array}{cc}
			%~E0,NoDisplay|5|2,4%
			&%~D0,NoDisplay|5|2,4%
		\end{array}
	\end{array}

&
	\text{
		%~T|0|1,2,4%
	}
\end{array}
\]

\subsection*{$1,b3,4$}
\[
\begin{array}{ll}
	\begin{array}{c}
		\begin{array}{ccc}
			%~C0,NoDisplay|5|b3,4%   
			&%~A0,NoDisplay|5|b3,4%   
			&%~G0,NoDisplay|5|b3,4%
		\end{array}
		\\
		\begin{array}{cc}
			%~E0,NoDisplay|5|b3,4%
			&%~D0,NoDisplay|5|b3,4%
		\end{array}
	\end{array}

&
	\text{
		%~T|0|1,b3,4%
	}
\end{array}
\]

\subsection*{$1,2,\sharp4$}
\[
\begin{array}{ll}
	\begin{array}{c}
		\begin{array}{ccc}
			%~C0,NoDisplay|5|2,#4%   
			&%~A0,NoDisplay|5|2,#4%   
			&%~G0,NoDisplay|5|2,#4%
		\end{array}
		\\
		\begin{array}{cc}
			%~E0,NoDisplay|5|2,#4%
			&%~D0,NoDisplay|5|2,#4%
		\end{array}
	\end{array}

&
	\text{
		%~T|0|1,2,#4%
	}
\end{array}
\]

\subsection*{$1,3,\sharp4$}
\[
\begin{array}{ll}
	\begin{array}{c}
		\begin{array}{ccc}
			%~C0,NoDisplay|5|3,#4%   
			&%~A0,NoDisplay|5|3,#4%   
			&%~G0,NoDisplay|5|3,#4%
		\end{array}
		\\
		\begin{array}{cc}
			%~E0,NoDisplay|5|3,#4%
			&%~D0,NoDisplay|5|3,#4%
		\end{array}
	\end{array}

&
	\text{
		%~T|0|1,3,#4%
	}
\end{array}
\]


\section{The Remaining Four-Note Possibilities}

As with the previous section, here we collect together all the remaining possible combinations of four notes that have not already been covered earlier in the chapter, omitting of course the `arpeggio' of four consecutive notes, which contravenes the rule regarding consecutive semitones.

\subsection*{With Two Semitones}

\subsection*{$1,2,b3,3$}
\[
\begin{array}{ll}
	\begin{array}{c}
		\begin{array}{ccc}
			%~C0,NoDisplay|5|2,b3,3%   
			&%~A0,NoDisplay|5|2,b3,3%   
			&%~G0,NoDisplay|5|2,b3,3%
		\end{array}
		\\
		\begin{array}{cc}
			%~E0,NoDisplay|5|2,b3,3%
			&%~D0,NoDisplay|5|2,b3,3%
		\end{array}
	\end{array}

&
	\text{
		%~T|0|1,2,b3,3%
	}
\end{array}
\]


\subsection*{$1,b2,b3,3$}
\[
\begin{array}{ll}
	\begin{array}{c}
		\begin{array}{ccc}
			%~C0,NoDisplay|5|b2,b3,3%   
			&%~A0,NoDisplay|5|b2,b3,3%   
			&%~G0,NoDisplay|5|b2,b3,3%
		\end{array}
		\\
		\begin{array}{cc}
			%~E0,NoDisplay|5|b2,b3,3%
			&%~D0,NoDisplay|5|b2,b3,3%
		\end{array}
	\end{array}

&
	\text{
		%~T|0|1,b2,b3,3%
	}
\end{array}
\]

\subsection*{$1,b2,2,3$}
\[
\begin{array}{ll}
	\begin{array}{c}
		\begin{array}{ccc}
			%~C0,NoDisplay|5|b2,2,3%   
			&%~A0,NoDisplay|5|b2,2,3%   
			&%~G0,NoDisplay|5|b2,2,3%
		\end{array}
		\\
		\begin{array}{cc}
			%~E0,NoDisplay|5|b2,2,3%
			&%~D0,NoDisplay|5|b2,2,3%
		\end{array}
	\end{array}

&
	\text{
		%~T|0|1,b2,2,3%
	}
\end{array}
\]

\subsection*{$1,b3,3,4$}
\[
\begin{array}{ll}
	\begin{array}{c}
		\begin{array}{ccc}
			%~C0,NoDisplay|5|b3,3,4%   
			&%~A0,NoDisplay|5|b3,3,4%   
			&%~G0,NoDisplay|5|b3,3,4%
		\end{array}
		\\
		\begin{array}{cc}
			%~E0,NoDisplay|5|b3,3,4%
			&%~D0,NoDisplay|5|b3,3,4%
		\end{array}
	\end{array}

&
	\text{
		%~T|0|1,b3,3,4%
	}
\end{array}
\]

\subsection*{$1,b2,3,4$}
\[
\begin{array}{ll}
	\begin{array}{c}
		\begin{array}{ccc}
			%~C0,NoDisplay|5|b2,3,4%   
			&%~A0,NoDisplay|5|b2,3,4%   
			&%~G0,NoDisplay|5|b2,3,4%
		\end{array}
		\\
		\begin{array}{cc}
			%~E0,NoDisplay|5|b2,3,4%
			&%~D0,NoDisplay|5|b2,3,4%
		\end{array}
	\end{array}

&
	\text{
		%~T|0|1,b2,3,4%
	}
\end{array}
\]

\subsection*{$1,b2,2,4$}
\[
\begin{array}{ll}
	\begin{array}{c}
		\begin{array}{ccc}
			%~C0,NoDisplay|5|b2,2,4%   
			&%~A0,NoDisplay|5|b2,2,4%   
			&%~G0,NoDisplay|5|b2,2,4%
		\end{array}
		\\
		\begin{array}{cc}
			%~E0,NoDisplay|5|b2,2,4%
			&%~D0,NoDisplay|5|b2,2,4%
		\end{array}
	\end{array}

&
	\text{
		%~T|0|1,b2,2,4%
	}
\end{array}
\]

\subsection*{$1,b2,2,b6$}
\[
\begin{array}{ll}
	\begin{array}{c}
		\begin{array}{ccc}
			%~C0,NoDisplay|5|b2,2,b6%   
			&%~A0,NoDisplay|5|b2,2,b6%   
			&%~G0,NoDisplay|5|b2,2,b6%
		\end{array}
		\\
		\begin{array}{cc}
			%~E0,NoDisplay|5|b2,2,b6%
			&%~D0,NoDisplay|5|b2,2,b6%
		\end{array}
	\end{array}

&
	\text{
		%~T|0|1,b2,2,b6%
	}
\end{array}
\]


\subsection*{With One Semitone}


\subsection*{$1,2,3,4$}
\[
\begin{array}{ll}
	\begin{array}{c}
		\begin{array}{ccc}
			%~C0,NoDisplay|5|2,3,4%   
			&%~A0,NoDisplay|5|2,3,4%   
			&%~G0,NoDisplay|5|2,3,4%
		\end{array}
		\\
		\begin{array}{cc}
			%~E0,NoDisplay|5|2,3,4%
			&%~D0,NoDisplay|5|2,3,4%
		\end{array}
	\end{array}

&
	\text{
		%~T|0|1,2,3,4%
	}
\end{array}
\]

\subsection*{$1,2,b3,4$}
\[
\begin{array}{ll}
	\begin{array}{c}
		\begin{array}{ccc}
			%~C0,NoDisplay|5|2,b3,4%   
			&%~A0,NoDisplay|5|2,b3,4%   
			&%~G0,NoDisplay|5|2,b3,4%
		\end{array}
		\\
		\begin{array}{cc}
			%~E0,NoDisplay|5|2,b3,4%
			&%~D0,NoDisplay|5|2,b3,4%
		\end{array}
	\end{array}

&
	\text{
		%~T|0|1,2,b3,4%
	}
\end{array}
\]

\subsection*{$1,b2,b3,4$}
\[
\begin{array}{ll}
	\begin{array}{c}
		\begin{array}{ccc}
			%~C0,NoDisplay|5|b2,b3,4%   
			&%~A0,NoDisplay|5|b2,b3,4%   
			&%~G0,NoDisplay|5|b2,b3,4%
		\end{array}
		\\
		\begin{array}{cc}
			%~E0,NoDisplay|5|b2,b3,4%
			&%~D0,NoDisplay|5|b2,b3,4%
		\end{array}
	\end{array}

&
	\text{
		%~T|0|1,b2,b3,4%
	}
\end{array}
\]


\subsection*{$1,2,4,b5$}
\[
\begin{array}{ll}
	\begin{array}{c}
		\begin{array}{ccc}
			%~C0,NoDisplay|5|2,4,b5%   
			&%~A0,NoDisplay|5|2,4,b5%   
			&%~G0,NoDisplay|5|2,4,b5%
		\end{array}
		\\
		\begin{array}{cc}
			%~E0,NoDisplay|5|2,4,b5%
			&%~D0,NoDisplay|5|2,4,b5%
		\end{array}
	\end{array}

&
	\text{
		%~T|0|1,2,4,b5%
	}
\end{array}
\]


\subsection*{Without Semitones}

\subsection*{$1,2,3,b5$}
\[
\begin{array}{ll}
	\begin{array}{c}
		\begin{array}{ccc}
			%~C0,NoDisplay|5|2,3,b5%   
			&%~A0,NoDisplay|5|2,3,b5%   
			&%~G0,NoDisplay|5|2,3,b5%
		\end{array}
		\\
		\begin{array}{cc}
			%~E0,NoDisplay|5|2,3,b5%
			&%~D0,NoDisplay|5|2,3,b5%
		\end{array}
	\end{array}

&
	\text{
		%~T|0|1,2,3,b5%
	}
\end{array}
\]







\chapter{\mbox{Pentatonics} \mbox{With} \mbox{Small} \mbox{Intervals}}


A pentatonic scale is just a scale with five notes in it. Talk of `the' pentatonic scale is strictly speaking a mistake, although the scale we've called the Common Pentatonic is, in fact, the only one in regular use in Western music. This chapter considers pentatonics that contain no interval larger than a major third; these have the most even spread of notes through the octave and are more likely to be of general use. The next chapter covers the remaining scales, which of course contain larger intervals.

We begin with a couple of simple alterations of the Common Pentatomic and then move on to some pentatonics that are best considered to be five-note arpeggios. We then spend some time on scales that can be thought of as pairs of triad and/or seventh arpeggios merged together, with one or more notes in common. This is an excellent opening into the study of bitonality, the practice of implying a different key centre in your playing from the one in the underlying harmony, and even tritonality for those brave enough to attempt it.

Many pentatonic scales don't contain full seventh arpeggios and hence don't have natural harmonisations of the sort that most heptatonics do. Because they have only five notes, these scales have large gaps. Take, for instance, the Phrygian Kumoi scale. Its formula is 1 $\flat$2 4 5 $\flat$7. It contains the 5, so we can assume that it probably won't sound consonant over a diminished or augmented triad. It doesn't, though, contain either a 3 or a $\flat$3. In this situation we can consider the 3 (or $\flat$3) in the chord to fall in the gap between the 2 and 4 of the pentatonic. It doesn't create a clash because the gap is large enough for the extra chord note to fit into. Hence this scale will work over either a dominant chord (because of the $\flat$7) or a minor seventh. In the interests of brevity, these various options are not presented explicitly; some will, naturally, work better to your ears than others.

We end the chapter with the pentatonics that have two consecutive semitones. Because of the intervallic structure this implies, these are mostly more dissonant and angular than the others and hence you may find them more difficult to use. You may also, of course, find them interesting.

\section{Five-Note Arpeggios}

A pentatonic scale and a five-note arpeggio are really the same thing looked at in different ways, so any of the scales in this chapter could be called `five-note arpeggios'. This section collects a few that are based on tolerably common jazz chords that don't easily fit into the other categories.

\subsection{The $\Delta$ Add 11 Scale Group}

Note that in these scale names, for example, `4$\Delta$ Add 11' means the $\Delta$ Add 11 arpeggio with the root built on the fourth of the underlying chord, \emph{not} the 4$\Delta$ arpeggio with the $11^{\text{th}}$ of the underlying chord added. The latter would not make sense, since the 4$\Delta$ arpeggio, of course, already contains the $11^{\text{th}}$ of the underlying chord (which is just the 4 by another name). The first scale in this group is sometimes called the Pelog scale, after a similar structure found in Javanese music.

%~M|5|1,3,4,5,7|$\Delta$ Add 11@$1^{\text{min}}$ + $\flat6^{\text{sus4}}$@$1^{\text{sus2}}$ + $7^{\text{min}}$@4$\Delta$ Add 11@$\flat2 \Delta$ Add 11%

\subsection{The Kumoi Scale Group}
Kumoi is a koto tuning rather than a scale\footnote{So the author, who is not an expert in Japanese music by any stretch of the imagination, has been told.}. The Kumoi IV scale is the arpeggio of the $\Delta$ Add $\sharp4$. The third mode could also be named the `Hira scale', after another koto tuning.
%~M|5|1,b2,4,b5,b7|Kumoi@Kumoi II@Kumoi III@Kumoi IV@Kumoi V%

\subsection{The Dominant Add 11 Scale Group}
%~M|5|1,3,4,5,b7|Dominant Add 11@$1^{\text{dim}}$ + $\flat6^{\text{sus4}}$@$5^{\text{dom}}$@$4^{\text{maj}}$ + $\flat7^{\text{sus4}}$@$2^{\text{maj}}$ + $2^{\text{sus4}}$%

\subsection{The Dominant Add $\sharp$11 Scale Group}
%~M|5|1,3,#4,5,b7|Dominant Add $\sharp$4@$1^{\text{dim}}$ + $2^{\text{dim}}$@$\flat5^{\text{dom}}$@$1^{\text{dom}}$ Add 7@$2^{\text{dom}}$ Add $\sharp$5%

\subsection{$1^{\varnothing}$ Add $\natural3$}
Jazz musicians would be more likely to call this chord a $7\flat5\sharp9$.

%~M|5|1,#2,3,b5,b7|
7m6@$\sharp4^{\varnothing}$ Add $\natural3$@$2^{\varnothing}$ Add $\natural3$@$1^{\varnothing}$ Add $\natural3$@1m6$\flat9$%

\subsubsection{The Augmented Ninth Scale Group}
The first scale in this group is in fact just an arpeggio of the +9 chord -- that is, an augmented triad with $\flat7$ and 9 added to it.
%~M|5|1,2,3,#5,b7|Augmented Ninth Mode I@Augmented Ninth Mode II@Augmented Ninth Mode III@Augmented Ninth Mode IV@Augmented Ninth Mode V%

\section{Altered Note Pentatonics}

It's useful to study those pentatonic scales formed by altering one note of the Common Pentatonic, since the Common Pentatonic is so familiar that it makes them easy to learn (we'll use this approach a lot in the following chapters on heptatonics). We have already seen some examples in the foregoing sections; here are the remaining possibilities. Where the root is the altered note, the name `Altered Pentatonic' has been used.


\subsubsection{The Common Pentatonic $\flat$3 Scale Group}
%~M|5|1,2,b3,5,6|Common Pentatonic $\flat$3@Dorian Pentatonic $\flat$2@Altered Pentatonic I@Mixolydian Pentatonic $\flat$6@Common Minor Pentatonic $\flat$5%

 \subsubsection{The Common Pentatonic $\sharp$5 Scale Group}
 These scales are derived from the Harmonic Minor scale, and in most cases they contain an augmented triad at the root. They can be played wherever you would superimpose an augmented triad for added colour.
 %~M|5|1,2,3,#5,6|Common Pentatonic $\sharp$5@Dorian Pentatonic $\sharp$4@Phrygian Major Pentatonic@Altered Pentatonic II@Common Minor Pentatonic $\natural$7%

\subsubsection{The Common Pentatonic $\flat$6 Scale Group}
These scales are derivatives of the Melodic Minor, and the Common Pentatonic $\flat$6 is a useful jazz sound. 
%~M|5|1,2,3,5,b6|Common Pentatonic $\flat$6@Dorian Pentatonic $\flat$5@Phrygian Pentatonic $\flat$4@Mixolydian Pentatonic $\flat$2@Altered Pentatonic III%

\section{Pairs of Triads Sharing One Note}

\subsection{Built on the Major Triad}

\subsubsection{The $1^{\text{maj}}$ + $\flat 2^{\text{min}}$ Scale Group}
%~M|5|1,b2,3,5,b6|$1^{\text{maj}}$ + $\flat 2^{\text{min}}$@$1^{\text{min}}$ + $7^{\text{maj}}$@$\sharp 5^{\text{maj}}$ + $6^{\text{min}}$@$4^{\text{maj}}$ + $\flat 5^{\text{min}}$@$3^{\text{maj}}$ + $4^{\text{min}}$%

\subsubsection{The $1^{\text{maj}}$ + $3^{\text{maj}}$ Scale Group}
%~M|5|1,3,5,b6,7|$1^{\text{maj}}$ + $3^{\text{maj}}$@$1^{\text{maj}}$ + $\flat 6^{\text{maj}}$@$3^{\text{min}}$ + $4^+$@$3^{\text{maj}}$ + $\sharp 5^{\text{maj}}$@$\flat 2^{\text{maj}}$ + $4^{\text{maj}}$%

\subsubsection{The $1^{\text{maj}}$ + $5^{\text{maj}}$ Scale Group}
%~M|5|1,2,3,5,7|$1^{\text{maj}}$ + $5^{\text{maj}}$@$4^{\text{maj}}$ + $\flat 7^{\text{maj}}$@$\flat 3^{\text{maj}}$ + $\flat 6^{\text{maj}}$@$1^{\text{maj}}$ + $4^{\text{maj}}$@$\flat 2^{\text{maj}}$ + $\flat 6^{\text{maj}}$%

\subsubsection{The $1^{\text{maj}}$ + $5^{\text{dim}}$ Scale Group}
%~M|5|1,b2,3,5,b7|$1^{\text{maj}}$ + $5^{\text{dim}}$@$7^{\text{maj}}$ + $\flat 5^{\text{dim}}$@$\flat 3^{\text{dim}}$ + $\flat 6^{\text{maj}}$@$1^{\text{dim}}$ + $4^{\text{maj}}$@$1^{\text{dim}}$ + $2^{\text{maj}}$%

\subsubsection{The $1^{\text{maj}}$ + $5^+$ Scale Group}
%~M|5|1,#2,3,5,7|$1^{\text{maj}}$ + $5^+$@$1^+$ + $6^{\text{maj}}$@$1^{\text{min}}$ + $\flat 6^{\text{min}}$@$1^+$ + $4^{\text{maj}}$@$1^+$ + $\flat 2{\text{maj}}$%

\subsection{Built on the Minor Triad}

\subsubsection{The $1^{\text{min}}$ + $\flat3^{\text{min}}$ Scale Group}

The $1^{\text{min}}$ + $\flat3^{\text{min}}$ scale can also be thought of as a Common Minor Pentatonic with a $\sharp$4. More such alterations are discussed below.
%~M|5|1,b3,#4,5,b7|
$1^{\text{min}}$ + $\flat3^{\text{min}}$@$1^{\text{min}}$ + $6^{\text{min}}$@$\flat 5^{\text{min}}$ + $6^{\text{min}}$@$4^{\text{min}}$ + $\flat 6^{\text{min}}$@$2^{\text{min}}$ + $4^{\text{min}}$%

\subsubsection{The $1^{\text{min}}$ + $4^{\text{maj}}$ Scale Group}
The first scale in this group is sometimes known as the Minor 6 Pentatonic and is often heard in the work of modern blues players like Robert Cray. In each case, it's worth comparing the scale with the corresponding scale in the common pentatonic group, where one of the notes will be a semitone higher in pitch than here.
%~M|5|1,b3,4,5,6|$1^{\text{min}}$ + $4^{\text{maj}}$@$2^{\text{maj}}$ + $6^{\text{min}}$@$1^{\text{maj}}$ + $5^{\text{min}}$@$4^{\text{min}}$ + $\flat 7^{\text{maj}}$@$\flat 3^{\text{min}}$ + $\flat 6^{\text{maj}}$%

\subsubsection{The $1^{\text{min}}$ + $5^{\text{min}}$ Scale Group}
%~M|5|1,2,b3,5,b7|$1^{\text{min}}$ + $5^{\text{min}}$@$4^{\text{min}}$ + $\flat 7^{\text{min}}$@$6^{\text{min}}$ + $7^{\text{min}}$@$1^{\text{min}}$ + $4^{\text{min}}$@$2^{\text{min}}$ + $6^{\text{min}}$%

\subsection{Built on the Diminished Triad}

\subsection{The Three-Quarter-Diminished Scale Group}
This name is a slightly whimsical reflection of the fact that the first scale in the group is a blending of the half-diminished and diminished seventh arpeggios, and it therefore halfway between the two.
%~M|5|1,b3,b5,6,b7|Three-Quarter-Diminished@Diminished Add 5@Diminished Add 3@Diminished Add $\flat2$@$4^{\text{dim}}$%

\subsection{Built on the Augmented Triad}

\subsubsection{The $1^+$ + $\sharp 5^{\text{dim}}$ Scale Group}
%~M|5|1,2,3,#5,7|$1^+$ + $\sharp 5^{\text{dim}}$@$2^+$ + $\sharp 4^{\text{dim}}$@$1^+$ + $3^{\text{dim}}$@$1^+$ + $1^{\text{dim}}$@$\flat 2^+$ + $6^{\text{dim}}$%

\subsubsection{The $1^+$ + $\flat 2^{\text{min}}$ Scale Group}
%~M|5|1,b2,3,#5,b7|$1^+$ + $\flat 2^{\text{min}}$@$1^{\text{min}}$ + $7^+$@$1^+$ + $6^{\text{min}}$@$1^+$ + $4^{\text{min}}$@$2^+$ + $\flat 3^{\text{min}}$%


\subsection{A Triad and a Seventh Sharing Two Notes}

There is only one possibility not covered already. The scales in this group are five of the modes of the harmonic minor scale with, of course, two notes missing in each case. The Common Pentatonic $\flat$2 is the same as a major triad superimposed with a dominant chord built on the 6, which is the third step of the `Charlie Parker cycle'.

\subsubsection{The $1^{\text{maj}}$ + $6^d$ Scale Group}
%~M|5|1,b2,3,5,6|$1^{\text{maj}}$ + $6^d$@$\flat 6^d$ + $7^{\text{maj}}$@$4^d$ + $\flat 6^{\text{maj}}$@$2^d$ + $4^{\text{maj}}$@$1^d$ + $\sharp 2^{\text{maj}}$%


\subsection{Two Sevenths Sharing Three Notes}

\subsubsection{The $1^m$ + $\flat 3^d$ Scale Group}
%~M|5|1,b2,b3,5,b7|$1^m$ + $\flat 3^d$@$2^d$ + $7^m$@$1^d$ + $6^m$@$4^m$ + $\flat 6^d$@$2^m$ + $\flat 4^d$%

\subsubsection{The $1^\varnothing$ + $\flat 3^m$ Scale Group}
%~M|5|1,b2,b3,b5,b7|$1^\varnothing$ + $\flat 3^m$@$2^m$ + $7^\varnothing$@$1^m$ + $6^\varnothing$@$\sharp 4^\varnothing$ + $6^m$@$2^\varnothing$ + $4^m$%

\section{With Two Consecutive Semitones}

Because they only contain five notes altogether, pentatonics with two consecutive semitones tend to have a different quality from the others because of the bunched-up arrangement of the notes. This section collects all the possibilities that don't contain an interval bigger than a major third.

Nine of these scales take the form of an augmented arpeggio with two notes lying chromatically next to or around one of its pitches. I've named these `Augmented Leading Note' scales, purely because they're easy shapes to learn and so it's useful to be able to identify them. All but the first group in this section contain at least one of these scales.

\subsubsection{The $7\Delta^{\text{sus4}}$ Scale}
%~M|5|1,b2,4,5,7|1$\Delta^{\text{sus4}}$ Add $\flat2$@$7\Delta^{\text{sus4}}$@$\flat6^{\varnothing}$ Add $\natural7$@$7\flat5$ Add 11@$\flat2^{\text{sus4}}$ Add $\flat2$%

\subsubsection{Augmented 6/7}
%~M|5|1,3,#5,6,b7|Augmented 6/7@Augmented Leading Note I@Augmented Leading Note II@Minor $\Delta$ Add $\flat2$@$\flat7^{\text{aug}}$ Add $\flat2$%

\subsubsection{Augmented Leading Note III}
%~M|5|1,3,#5,#6,7|Augmented Leading Note III@Major Add $\sharp4\flat6$@Augmented Leading Note IV@$2^+$ Add $\natural7$@$\flat2^+$ Add $\flat7$%

\subsubsection{$\Delta$ Add $\flat2$}
%~M|5|1,b2,3,5,7|$\Delta$ Add $\flat2$@Half Diminished Add $\natural7$@Augmented Leading Note V@$4\Delta$ Add $\flat2$@$\flat2\Delta$ Add $\flat2$%

\subsubsection{Minor 7 Add $\natural7$}
%~M|5|1,b3,5,b7,7|Minor 7 Add $\natural7$@Major 6/$\flat6$@Augmented Leading Note VI@$\flat2^+$ Add $\flat2$@Augmented Leading Note VII%

\subsubsection{Augmented Leading Note VIII}
%~M|5|1,3,#5,#6,7|Augmented Leading Note VIII@Major Add $\sharp4\flat6$@Augmented Leading Note IX@$2^+$ Add $\natural7$@$\flat2^+$ Add $\flat7$%


\chapter{\mbox{Pentatonics} \mbox{Containing} \mbox{a} \mbox{Large} \mbox{Interval}}

The pentatonics in this chapter and the next are more `angular' than those considered in the previous one due to the presence of a fourth, augmented fourth or fifth in their interval maps. Each is probably best considered as the modal group generated by a triad with two added notes. As a result there's something arpeggio-like about these and I've chosen to name them as if they were arpeggios (using, for example, $\flat$9 instead of $\flat$2). Although the basic scale in each case is fairly easy to understand and learn the modes are often very exotic. This chapter considers the pentatonics in three main groups: those whose largest interval is a fourth, an augmented fourth and a fifth.

\section{With Largest Interval a Fourth}

\subsection{Based on a Major Triad}

\subsubsection{Major Add 9/$\sharp$9}
%~M|5|1,2,#2,3,5|Major Add 9/$\sharp$9 Mode I@Major Add 9/$\sharp$9 Mode II@Major Add 9/$\sharp$9 Mode III@Major Add 9/$\sharp$9 Mode IV@Major Add 9/$\sharp$9 Mode V%

\subsubsection{Major Add $\flat$9/$\natural$9}
%~M|5|1,b2,2,3,5|Major Add $\flat$9/$\natural$9 Mode I@Major Add $\flat$9/$\natural$9 Mode II@Major Add $\flat$9/$\natural$9 Mode III@Major Add $\flat$9/$\natural$9 Mode IV@Major Add $\flat$9/$\natural$9 Mode V%

\subsubsection{Major Add $\flat$9/$\sharp$9}
%~M|5|1,b2,#2,3,5|Major Add $\flat$9/$\sharp$9 Mode I@Major Add $\flat$9/$\sharp$9 Mode II@Major Add $\flat$9/$\sharp$9 Mode III@Major Add $\flat$9/$\sharp$9 Mode IV@Major Add $\flat$9/$\sharp$9 Mode V%

\subsubsection{Major Add 9/11}
%~M|5|1,2,3,4,5|Major Add 9/11 Mode I@Major Add 9/11 Mode II@Major Add 9/11 Mode III@Major Add 9/11 Mode IV@Major Add 9/11 Mode V%

\subsubsection{Major Add 9/$\sharp$11}
%~M|5|1,2,3,#4,5|Major Add 9/$\sharp$11 Mode I@Major Add 9/$\sharp$11 Mode II@Major Add 9/$\sharp$11 Mode III@Major Add 9/$\sharp$11 Mode IV@Major Add 9/$\sharp$11 Mode V%

\subsubsection{Major Add $\sharp$9/11}
%~M|5|1,#2,3,4,5|Major Add $\sharp$9/11 Mode I@Major Add $\sharp$9/11 Mode II@Major Add $\sharp$9/11 Mode III@Major Add $\sharp$9/11 Mode IV@Major Add $\sharp$9/11 Mode V%

\subsubsection{Major Add $\sharp$9/$\sharp$11}
%~M|5|1,#2,3,#4,5|Major Add $\sharp$9/$\sharp$11 Mode I@Major Add $\sharp$9/$\sharp$11 Mode II@Major Add $\sharp$9/$\sharp$11 Mode III@Major Add $\sharp$9/$\sharp$11 Mode IV@Major Add $\sharp$9/$\sharp$11 Mode V%

\subsubsection{Major Add $\flat$9/11}
%~M|5|1,b2,3,4,5|Major Add $\flat$9/11 Mode I@Major Add $\flat$9/11 Mode II@Major Add $\flat$9/11 Mode III@Major Add $\flat$9/11 Mode IV@Major Add $\flat$9/11 Mode V%

\subsubsection{Major Add $\flat$9/$\sharp$11}
%~M|5|1,b2,3,#4,5|Major Add $\flat$9/$\sharp$11 Mode I@Major Add $\flat$9/$\sharp$11 Mode II@Major Add $\flat$9/$\sharp$11 Mode III@Major Add $\flat$9/$\sharp$11 Mode IV@Major Add $\flat$9/$\sharp$11 Mode V%

\subsection{Based on a Minor Triad}

\subsubsection{Minor Add 11/$\sharp$11}
%~M|5|1,b3,4,#4,5|Minor Add 11/$\sharp$11 Mode I@Minor Add 11/$\sharp$11 Mode II@Minor Add 11/$\sharp$11 Mode III@Minor Add 11/$\sharp$11 Mode IV@Minor Add 11/$\sharp$11 Mode V%

\subsubsection{Minor Add 9/11}
%~M|5|1,2,b3,4,5|Minor Add 9/11 Mode I@Minor Add 9/11 Mode II@Minor Add 9/11 Mode III@Minor Add 9/11 Mode IV@Minor Add 9/11 Mode V%

\subsubsection{Minor Add 9/$\sharp$11}
%~M|5|1,2,b3,#4,5|Minor Add 9/$\sharp$11 Mode I@Minor Add 9/$\sharp$11 Mode II@Minor Add 9/$\sharp$11 Mode III@Minor Add 9/$\sharp$11 Mode IV@Minor Add 9/$\sharp$11 Mode V%

\subsubsection{Minor Add $\flat$9/$\sharp$11}
%~M|5|1,b2,b3,#4,5|Minor Add $\flat$9/$\sharp$11 Mode I@Minor Add $\flat$9/$\sharp$11 Mode II@Minor Add $\flat$9/$\sharp$11 Mode III@Minor Add $\flat$9/$\sharp$11 Mode IV@Minor Add $\flat$9/$\sharp$11 Mode V%

\subsubsection{Minor Add $\flat$9/11}
%~M|5|1,b2,b3,4,5|Minor Add $\flat$9/11 Mode I@Minor Add $\flat$9/11 Mode II@Minor Add $\flat$9/11 Mode III@Minor Add $\flat$9/11 Mode IV@Minor Add $\flat$9/11 Mode V%


\subsection{Based on a Suspended Triad}

\subsubsection{Sus 2 Add 11/$\sharp$11}
%~M|5|1,2,4,#4,5|Sus 2 Add 11/$\sharp$11 Mode I@Sus 2 Add 11/$\sharp$11 Mode II@Sus 2 Add 11/$\sharp$11 Mode III@Sus 2 Add 11/$\sharp$11 Mode IV@Sus 2 Add 11/$\sharp$11 Mode V%

\subsubsection{Sus 2 Add $\flat$13/$\natural$13}
%~M|5|1,2,5,b6,6|Sus 2 Add $\flat$13/$\natural$13 Mode I@Sus 2 Add $\flat$13/$\natural$13 Mode II@Sus 2 Add $\flat$13/$\natural$13 Mode III@Sus 2 Add $\flat$13/$\natural$13 Mode IV@Sus 2 Add $\flat$13/$\natural$13 Mode V%

\subsubsection{Sus $\flat$2 Add 11/$\sharp$11}
%~M|5|1,b2,4,#4,5|Sus $\flat$2 Add 11/$\sharp$11 Mode I@Sus $\flat$2 Add 11/$\sharp$11 Mode II@Sus $\flat$2 Add 11/$\sharp$11 Mode III@Sus $\flat$2 Add 11/$\sharp$11 Mode IV@Sus $\flat$2 Add 11/$\sharp$11 Mode V%

\subsubsection{Sus $\flat$2 Add $\sharp$11/$\flat$13}
%~M|5|1,b2,#4,5,b6|Sus $\flat$2 Add $\sharp$11/$\flat$13 Mode I@Sus $\flat$2 Add $\sharp$11/$\flat$13 Mode II@Sus $\flat$2 Add $\sharp$11/$\flat$13 Mode III@Sus $\flat$2 Add $\sharp$11/$\flat$13 Mode IV@Sus $\flat$2 Add $\sharp$11/$\flat$13 Mode V%



\section{With Largest Interval an Augmented Fourth}

\subsection{Based on a Diminished Triad}

\subsubsection{Diminished Add 9/$\flat$11}
%~M|5|1,2,b3,3,5|Diminished Add 9/$\flat$11 Mode I@Diminished Add 9/$\flat$11 Mode II@Diminished Add 9/$\flat$11 Mode III@Diminished Add 9/$\flat$11 Mode IV@Diminished Add 9/$\flat$11 Mode V%

\subsubsection{Diminished Add $\flat$9/11}
%~M|5|1,b2,b3,4,b5|Diminished Add $\flat$9/11 Mode I@Diminished Add $\flat$9/11 Mode II@Diminished Add $\flat$9/11 Mode III@Diminished Add $\flat$9/11 Mode IV@Diminished Add $\flat$9/11 Mode V%

\subsubsection{Diminished Add $\flat$9/$\flat$11}
%~M|5|1,b2,b3,3,b5|Diminished Add $\flat$9/$\flat$11 Mode I@Diminished Add $\flat$9/$\flat$11 Mode II@Diminished Add $\flat$9/$\flat$11 Mode III@Diminished Add $\flat$9/$\flat$11 Mode IV@Diminished Add $\flat$9/$\flat$11 Mode V%


\subsection{Based on a Suspended Triad}

\subsubsection{Sus $\flat$2 $\Delta$ Add 13}
%~M|5|1,b2,5,6,7|Sus $\flat$2 $\Delta$ Add 13 Mode I@Sus $\flat$2 $\Delta$ Add 13 Mode II@Sus $\flat$2 $\Delta$ Add 13 Mode III@Sus $\flat$2 $\Delta$ Add 13 Mode IV@Sus $\flat$2 $\Delta$ Add 13 Mode V%

\subsubsection{Sus $\flat$2 $\Delta$ Add $\flat$13}
%~M|5|1,b2,5,b6,7|Sus $\flat$2 $\Delta$ Add $\flat$13 Mode I@Sus $\flat$2 $\Delta$ Add $\flat$13 Mode II@Sus $\flat$2 $\Delta$ Add $\flat$13 Mode III@Sus $\flat$2 $\Delta$ Add $\flat$13 Mode IV@Sus $\flat$2 $\Delta$ Add $\flat$13 Mode V%

\subsubsection{Sus $\flat$2 $\flat$7 Add 13}
%~M|5|1,b2,5,6,b7|Sus $\flat$2 $\flat$7 Add 13 Mode I@Sus $\flat$2 $\flat$7 Add 13 Mode II@Sus $\flat$2 $\flat$7 Add 13 Mode III@Sus $\flat$2 $\flat$7 Add 13 Mode IV@Sus $\flat$2 $\flat$7 Add 13 Mode V%

\subsubsection{Sus $\flat$2 Add $\flat$13/$\natural$13}
%~M|5|1,b2,5,b6,6|Sus $\flat$2 Add $\flat$13/$\natural$13 Mode I@Sus $\flat$2 Add $\flat$13/$\natural$13 Mode II@Sus $\flat$2 Add $\flat$13/$\natural$13 Mode III@Sus $\flat$2 Add $\flat$13/$\natural$13 Mode IV@Sus $\flat$2 Add $\flat$13/$\natural$13 Mode V%

\subsubsection{Sus $\sharp$4 $\Delta$ Add $\flat$13}
%~M|5|1,#4,5,b6,7|Sus $\sharp$4 $\Delta$ Add $\flat$13 Mode I@Sus $\sharp$4 $\Delta$ Add $\flat$13 Mode II@Sus $\sharp$4 $\Delta$ Add $\flat$13 Mode III@Sus $\sharp$4 $\Delta$ Add $\flat$13 Mode IV@Sus $\sharp$4 $\Delta$ Add $\flat$13 Mode V%

\subsubsection{Sus $\sharp$4 $\flat$7 Add $\flat$13}
%~M|5|1,#4,5,b6,b7|Sus $\sharp$4 $\flat$7 Add $\flat$13 Mode I@Sus $\sharp$4 $\flat$7 Add $\flat$13 Mode II@Sus $\sharp$4 $\flat$7 Add $\flat$13 Mode III@Sus $\sharp$4 $\flat$7 Add $\flat$13 Mode IV@Sus $\sharp$4 $\flat$7 Add $\flat$13 Mode V%



\section{With Largest Interval a Fifth}

These scales do not contain any triad arpeggios and are perhaps best thought of as an interval of a fifth with three additional notes squeezed into the space between the 5 and the root an octave above. Because of the crowding-in of the notes their usefulness may be rather limited, but they do offer some surprising melodic material.

\subsubsection{Fifth Add $\flat$6/$\flat$7/$\natural$7}
%~M|5|1,5,b6,b7,7|Fifth Add $\flat$6/$\flat$7/$\natural$7 Mode I@Fifth Add $\flat$6/$\flat$7/$\natural$7 Mode II@Fifth Add $\flat$6/$\flat$7/$\natural$7 Mode III@Fifth Add $\flat$6/$\flat$7/$\natural$7 Mode IV@Fifth Add $\flat$6/$\flat$7/$\natural$7 Mode V%
\subsubsection{Fifth Add $\flat$6/$\natural$6/7}
%~M|5|1,5,b6,6,7|Fifth Add $\flat$6/$\natural$6/7 Mode I@Fifth Add $\flat$6/$\natural$6/7 Mode II@Fifth Add $\flat$6/$\natural$6/7 Mode III@Fifth Add $\flat$6/$\natural$6/7 Mode IV@Fifth Add $\flat$6/$\natural$6/7 Mode V%





\chapter{\mbox{Altered}-\mbox{Note} \mbox{Heptatonics} \titlebreak \mbox{Part 1:} \mbox{The Major} \mbox{Scale}}

We have already seen the three most common heptatonic scales and noted that the scales in the Harmonic Minor and Melodic Minor groups are exactly the same as scales in the Major scale group with one note altered, as for example:
\begin{quote}
\begin{tabular}{ll}
Dorian scale (in major scale group) $\Rightarrow$ &Dorian $\sharp 4$ (in Harmonic Minor group) \\
					&Dorian $\flat 9$ (in Melodic Minor group)
\end{tabular}
\end{quote}
This is true of every scale in the groups except for the one in which the root was the altered note, which affects all the other notes. There are several more scale groups like these in which one note of the major scale is altered and the modes of the resulting scale are then considered.

To work out all such possibilities, we begin by thinking of every possible note in the major scale that could be altered. There are 7 notes and, in theory, each one could be either sharpened or flattened:
\begin{quote}
\begin{tabular}{lllllll}
1	&2	&3	&4	&5	&6	&7  \\
$\flat$/$\sharp$&$\flat$/$\sharp$&$\flat$/$\sharp$&$\flat$/$\sharp$&$\flat$/$\sharp$&$\flat$/$\sharp$&$\flat$/$\sharp$
\end{tabular}
\end{quote}
Remember, though, that only one note is going to be altered, so we can immediately rule out certain alterations, because these notes would become the same as another note in the scale, producing in effect a hexatonic that was the same as the major scale but with one note missing rather than a new heptatonic:
\begin{quote}
\begin{tabular}{ll}
$\flat 1$ (same as 7)	&$\flat 4$ (same as 3) \\
$\sharp 7$ (same as 1)	&$\sharp 3$ (same as 4)
\end{tabular}
\end{quote}
We must also get rid of the alterations that create scales that are in one of the three common heptatonic groups:
\begin{quote}
\begin{tabular}{ll}
$\sharp 1$ (creates Super Locrian scale)		&$\flat 3$ (creates Melodic Minor scale) \\
$\sharp 4$ (creates Lydian scale)			&$\sharp 5$ (creates Ionian augmented scale) \\
$\flat 7$ (creates Mixolydian scale)& 
\end{tabular}
\end{quote}
which leaves these five alterations:
\begin{quote}
\begin{tabular}{lllllll}
1	&2	&3	&4	&5	&6	&7 \\
	&$\flat$/$\sharp$&&&$\flat$&$\flat$/$\sharp$
\end{tabular}
\end{quote}

These alterations describe five new scale groups, and a quick comparison of their interval maps shows that none is a mode of another. The alterations to all the other scales in the Major scale group will fall into place as we work through the modes in these groups. If you decide to learn some of these scales, remember to use your knowledge of the modes in the Major scale group to full effect; only one note will be different from a familiar pattern, and that note will only be raised or lowered by a single fret.

All of the scale groups in this chapter have interval maps containing three semitones, three tones and one minor third (3s 3t 1mT), and all but one have two consecutive semitones. That exception is the Lydian Minor, which also happens to be the only one that has a reasonably well-known jazz application. The consecutive semitones result in patterns that contain three notes on adjacent frets, and the easiest way to learn a pattern like this is to see where those three notes occur in relation to the arpeggio. For example, in the Sharp Blues scale the highest note of the three is the root. Along with the arpeggio, which you already know, that observation immediately gives you five of the seven notes in the scale.

A pattern like this works in a very simple way if the highest note is part of the arpeggio. Then those two notes below it are easy to use as leading-notes that resolve upwards to the chord tone. When the arpeggio note is the lowest of the three, or the middle one, getting a satisfactory resolution is trickier. If these scales interest you, try as a preliminary study adding three consecutive notes to the major triad, first with the chord tone on top:
\[
\begin{array}{ccc}
	%~E,NoDisplay|5|#6,7%   
	&%~E,NoDisplay|5|2,#2%   
	&%~Em,NoDisplay|5|4,b5%  \\
	\text{Root} & \text{Third} & \text{Fifth} \\
\end{array}
\]
then with the chord tone in the middle, a pattern Al Di Meola likes to use over minor chords:
\[
\begin{array}{ccc}
	%~E,NoDisplay|5|b2,7%   
	&%~E,NoDisplay|5|#2,4%   
	&%~E,NoDisplay|5|b5,b6% \\  
	\text{Root} & \text{Third} & \text{Fifth} \\
\end{array}
\]
and finally -- most difficult -- with the chord tone at the bottom:
\[
\begin{array}{ccc}
	%~E,NoDisplay|5|b2,2%   
	&%~Em,NoDisplay|5|4,b5%   
	&%~E,NoDisplay|5|b6,6%  \\ 
	\text{Root} & \text{Third} & \text{Fifth} \\
\end{array}
\]
Note that we had to use minor triads in a couple of places to avoid having three consecutive semitone intervals, something we've said we'll try to avoid in this book.

This is an exercise in hearing and using dissonance; you will probably like some results more than others, and you should let that be your guide in choosing scales to study from the pages that follow. You can always, if you like, omit the middle note to create a hexatonic scale that will be significantly less dissonant, although it will almost certainly be less interesting as well.

The extraction of seventh arpeggios for every note in the scale is often problematic, especially as we move into scales that are highly altered with respect to the Major scale. Often it's only possible to extract part of a seventh arpeggio, so that the scale `suggests' that type of harmony but doesn't fully include it. Here, for example, is the Locrian $\flat\flat$3 scale:
\begin{quote}
$1^{[\varnothing]}$ $\flat 2^{\Delta}$ $2^{+}$ $4^{[m]}$ $\flat 5^{\Delta}$ $\flat 6^{d\flat 5}$ $\flat 7^{m}$
\end{quote}
At the root, we have 1, $\flat$5 and $\flat$7, which implies a half-diminished sound, but because there is no $\flat$3 we don't actually have a full half-diminished arpeggio in the scale. This is indicated by enclosing the half-diminished symbol in brackets. The brackets indicate that you should try using this scale to play over this type of chord, but you can't use the arpeggio of the chord to find notes in the scale because the arpeggio contains notes that the scale doesn't.

The meaning is similar for sevenths built on notes other than the root. If you wish to play only notes from the Locrian $\flat\flat$3 scale, you can certainly play the major seventh arpeggio built on the $\flat 2$ or the minor seventh built on the $\flat$7, but you can't play the minor seventh built on the 4 because the arpeggio contains one or more notes that don't belong in the scale. That's the rule you must follow if you want to stay within the scale. Of course, because these arpeggios are \emph{almost} included in the scale, playing them diverts you from the scale slightly, but in a logical way, which can provide interest and help to make a smooth transition into some other scale or arpeggio choice. When learning a scale, avoid the bracketed arpeggios, but when exploring it and integrating it with the rest of your vocabulary you may want to investigate them as variations.

\section{The Ionian $\flat$2 (Suryakantam) Scale Group}
%~M|5|1,2,3,#4,5,b6,7|
Rishabhapriya`The Rishabhapriya scale could also be called a Lydian $\flat$6@Mixolydian $\flat 5$
@Aeolian $\flat 4$@Locrian $\flat\flat 3$@Ionian $\flat$2 (Suryakantam)@Superaugmented $\natural$3`The Superaugmented scale itself is covered in the last group in this chapter.@Phrygian $\flat\flat 7$%


\section{The Ionian $\sharp$6 (Naganandini) Scale Group}
%~M|5|1,b2,b3,#4,5,b6,b7|Bhavapriya@Lydian $\sharp$3@Vagadhisvani@Locrian Double-Flat Major@Locrian $\natural 7$@Ionian $\sharp$6 (Naganandini) @Dorian $\sharp$5%

\section{The Ionian $\sharp$2 (Sulini) Scale Group}
%~M|5|1,b2,3,4,b5,b6,b7|Phrygian $\flat$5@Ionian $\sharp$2 (Sulini)@Superdiminished Major@Dhenuka@Chitrambari@Mixolydian $\sharp$5@Shanmukhapriya%

\section{The Ionian $\flat$6 (Sarasangi) Scale Group}
This is the one group in this chapter that doesn't contain consecutive semitones. The Mixolydian $\flat 9$ and Phrygian $\flat 4$ are both sometimes heard over dominant chords in jazz solos, and the Lydian Minor is an important scale in George Russell's famous `Lydian Chromatic' system.
%~M|5|1,2,b3,#4,5,6,7|Lydian Minor@Mixolydian $\flat 9$@Kosalam $\sharp$5@Locrian $\flat\flat 7$@Ionian $\flat$6 (Sarasangi)`This scale is also often referred to as the Harmonic Major scale since it can also be considered a Harmonic Minor with a natural third. In the nineteenth century it was sometimes identified as the Hungarian Major, although that name is now more often attached to a mode of the Harmonic Minor $\flat$5 (see below).@Dorian $\flat$5@Phrygian $\flat 4$%

\section{The Ionian $\flat 5$ Scale Group}

Indian melakatas never contain the $\flat$5, so this scale has no equivalent melakata. Ratnangi is among its modes, however.

%~M|5|1,#2,#3,#4,#5,#6,7|Superaugmented@Aeolian $\flat \flat 7$@
Locrian Blues`Note that although the harmonisation of this scale gives a diminished triad, the $\flat\flat$6 is the same as a natural fifth, so it contains a minor triad and can be played fairly consonantly over a minor harmony as well.@
Ionian $\flat 5$@Dorian $\flat 4$@Ratnangi@Marva%






\chapter{\mbox{Altered-Note} \titlebreak \mbox{Heptatonics} \mbox{Part 2:}\\ The Harmonic\ Minor Scale}
Just as it's possible to create scales by altering the notes of the Major scale, so you can do it with the Harmonic and Melodic Minor scales to create additional melodic resources. In this chapter, we cover the Harmonic Minor possibilities; the Melodic Minor ones are in the next. Again, use the `altered note' principle to your advantage if you already know the related Harmonic Minor shapes.

The first two scales are most similar to those in the previous chapter, since they too have interval maps containing 3s 3t 1mT. Of them, the Augmented Minor doesn't contain consecutive semitones, while the Harmonic Minor $\sharp 6$ (the Varunapriya Scale) does. The Harmonic Minor $\flat 4$  and the Double Harmonic, on the other hand, have the unusual interval composition 4s 1t 2mT. Both contain consecutive semitones, which can't be avoided with that composition.

\section{The Harmonic Minor $\sharp$6 (Varunapriya) Group}
%~M|5|1,2,b3,4,5,#6,7|Harmonic Minor $\sharp$6 (Varunapriya)@Natakapriya $\sharp$5@Augmented $\sharp\sharp 11$@Lydian Dominant Suspended@Aeolian Major $\sharp 2$@Phrygian Double-Flat@Super Locrian $\natural 7$%

\section{The Harmonic Minor $\flat$5 Scale Group}
%~M|5|1,2,b3,4,#5,6,7|Augmented Minor@Dorian Marva@Lydian Augmented Sus 4@Nasika Bhushani`Some other scale books refer to this as the Hungarian Major Scale. In fact there are several different Hungarian scales (Bartok collected them and used them in his compositions), and what counts as the Hungarian scale varies a bit depending on which book you read; Liszt, for example, called the Ionian $\flat$6 the Hungarian Major.@Ultra Locrian $\flat\flat$7@Harmonic Minor $\flat$5@Super Locrian $\natural 6$%

\section{The Harmonic Minor $\flat 4$ Group}
Notice the two augmented arpeggios a semitone apart; adding the flat fifth to the lower one gives the whole scale. As a first step in learning the scales in this group it would be worth practicing visualising and playing this pair of arpeggios in different positions.

%~M|5|1,b2,3,4,5,b6,bb7|Gayakapriya@Dhatuvardhani@Super Locrian $\flat\flat$5 $\flat\flat$7@Harmonic Minor $\flat 4$@Vanaspati $\flat$5@Augmented $\flat 9$@Gayakapriya Mode 4%


\section{The Harmonic Minor $\flat$6 Scale Group}

The Harmonic Minor contains two major triads separated by a semitone, and they can be extended to a $\Delta$ arpeggio with a dominant arpeggio a semitone below. Here the roles are reversed; you have a dominant arpeggio with a $\Delta$ arpeggio a semitone below. 

%~M|5|1,#2,3,#4,5,#6,7|Rasikapriya@Senavati $\flat$4@Harmonic Minor $\sharp$4`Some books refer to this as the Hungarian Minor scale, probably due to its inclusion in a collection of Hungarian scales by Liszt. The author has also seen this referred to as the Algerian Scale in a list of scales used by Robert Fripp but can't confirm whether the list is reliable.@Chakravakam $\flat$5@Augmented $\sharp 9$@Kanakangi $\flat$5@Double Harmonic%




\chapter{\mbox{Altered-Note} \titlebreak \mbox{Heptatonics} \mbox{Part 3:} \\ The Melodic\ Minor Scale}

There are only three possibilities, because several others have already been covered in the previous chapters. The Melodic Minor with a $\sharp 4$, for example, is a mode of the Lydian Minor. These three sets of scales complete the coverage of all of the possibilities for the altered-note principle -- a total of 98 scales, each one reasonably easy to learn if you're already familiar with the common heptatonics.

The Melodic Minor $\flat 4$ and Marva Dominant are both 3s 3t 1mT scale groups; the former does not have consecutive semitones while the latter does. The Neapolitan is one of only three scale groups having the composition 2s 5t; it is the same as a Whole Tone scale with an extra note added (the Whole Tone is covered in the chapter on symmetrical scales). 

\section{The Melodic Minor $\flat 4$ Scale Group}
%~M|5|1,2,b3,b4,5,6,7|Melodic Minor $\flat 4$@Vanaspati@Marva Augmented@Superaugmented $\sharp\sharp$4@Mararanjani@Dorian $\flat$5 $\flat\flat$6@Super Locrian $\flat\flat$5%

\section{The Melodic Minor $\flat$5 Scale Group}
Allan Holdsworth once included the Melodic Minor $\flat$5 in a list he presented in an instructional video, indicating that it was one of the scales he used quite frequently.
%~M|5|1,b2,3,#4,5,6,b7|Marva Dominant@Susdim Major@Jhankaradhvani $\flat$5@Super Locrian $\flat\flat$6@Melodic Minor $\flat$5@Natakapriya $\flat$4@Lydian $\flat$5%

\section{The Melodic Minor $\flat$2 (Neapolitan) Scale Group}

You can hear a brief lick using the Neapolitan scale in D on Frank Zappa's solo `Five-Five-FIVE'. He plays over a D pedal and starts with D Dorian, moves to the Phrygian and then plays a rhythmically very tricky section using the Neapolitan (see Steve Vai's book of transcriptions for the gory details). Knowing various different scales gives you the ability to give structure to a solo over a static harmony, which was something Zappa did masterfully. The author

%~M|5|1,b2,b3,4,5,6,7|Melodic Minor $\flat$2 (Neapolitan)@Lydian Augmented $\sharp$6@Lydian Dominant $\sharp$5@Overtone $\flat 6$@Charukesi $\flat$5`This scale has been referred to as the Arabian scale in certain other books, for no reason the present author can see.@Super Locrian $\flat 2$@Super Locrian $\flat\flat$3%


\chapter{\mbox{Further} \titlebreak \mbox{Heptatonics}}
The altered-note scales covered in the foregoing chapters account for the following possible interval compositions:
\begin{quote}
\begin{tabular}{ll}
		3s 3t 1mT		&9 of 16 already covered \\
		4s 1t 2mT		&2 of 6 already covered \\
		4s 2t 1MT		&0 of 6 already covered \\
		2s 5t			&3 of 3 already covered \\
\end{tabular}
\end{quote}
This chapter treats the rest of the 3s 3t 1mT and 4s 1t 2mT scales, and all of the 4s 2t 1MT groups, completing our coverage of all possible heptatonics taking into account the rules of usefulness introduced at the beginning of Part Two.

We also include a number of scales that are exceptions to the general rule that excludes scales containing more than two consecutive semitones. In fact we include \emph{all} possible heptatonics that contain three consecutive semitones, largely because they're somewhat interesting, many of them are \emph{melakatas} (or modes of them) and there aren't too many of them. We include only two scale groups that contain more than three consecutive semitones, however, since these really do seem to be beyond the pale.


\section{With One Minor Third}

These are all of the remaining heptatonic scales whose interval map contains a single minor third; we have already seen a large number of these among the `altered-note' heptatonics. All contain consecutive semitones. Kokilapriya has its consecutive semitones adjacent to the minor third whereas in the remaining three these two `features' are separated.

\subsection{The Suvarnangi Scale Group}
%~M|5|1,2,b3,b4,b5,6,b7|Dorian Blues@Ratnangi $\flat$4@Suvarnangi@Superaugmented $\natural$2@Lydian Dominant $\sharp$2$\sharp$5@Locrian $\flat\flat$6 $\flat\flat$7@Sarasangi $\flat$5%

\subsection{The Kokilapriya $\sharp$5 Scale Group}

There's a sour, unresolved and restless quality to all of the scales in this group; some of them really sound as if they could come from some Eastern European folk tradition, although of course they don't. Try playing around with these after listening to, say, some of Bartok's violin music.

%~M|5|1,b2,b3,4,#5,6,7|Kokilapriya $\sharp$5@Semidominant Augmented@Suspended Augmented@Hungarian $\sharp$9@Ultra Locrian $\flat\flat$5@Semidiminished@Phrygian Semidominant%

\subsection{The Vanaspati $\flat$4 Scale Group}
%~M|5|1,#2,#3,#4,#5,6,b7|Superaugmented $\natural$6 $\flat$7@Melodic Minor $\flat$5 $\flat\flat$6 $\flat\flat$7@Super Locrian $\flat\flat$5 $\flat\flat$6
@Midlocrian`The name of this scale is invented based on the fact that the middle three notes are flattened, while the outer notes are natural, following the logic of names like Super Locrian and Ultra Locrian.
@Vanaspati $\flat$4@Suvarnangi $\sharp$5@Superaugmented $\natural$2 $\sharp\sharp$4%

\subsection{The Melodic Minor $\flat$4 Scale Group}
%~M|5|1,b2,bb3,b4,bb5,b6,b7|Super Locrian $\flat\flat$3 $\flat\flat$5@Melodic Minor $\flat$4@Nitimati $\sharp$5@Ramapriya $\sharp$5@Superaugmented $\sharp\sharp$4 $\natural$6@Mararanjani $\flat$5@Super Locrian $\natural$2 $\flat\flat$6%

\subsection{The Varunapriya $\sharp$5 Scale Group}
%~M|5|1,b2,bb3,b4,bb5,bb6,b7|Super Locrian $\flat\flat$3 $\flat\flat$5 $\flat\flat$6@Super Locrian $\natural$6 $\natural$7@Varunapriya $\sharp$5@Shadvidha Margini $\sharp$5@Augmented $\sharp$3 $\sharp\sharp$4@Lydian Dominant $\sharp$2 $\sharp$3`This is closely related to the more dissonant octatonic scale that George Russell calls the African-American Blues Scale; it can also be thought of as the Blues Scale with an added 6.@Super Locrian $\natural$2 $\flat\flat$6 $\flat\flat$7%


\section{With Two Minor Thirds}

Each of the four scale groups in this section contains two minor thirds in its interval map. Being heptatonic, they are therefore forced to contain four semitones and one tone. In the Yagapriya group, the scales have adjacent minor thirds, creating a very striking sound as half of the scale is just a diminished triad, with all of the remaining notes being crammed in the other half. In the Nitimati group the minor thirds are separated by one semitone while in the Kamavardhani and Jhalavarali they're separated by two semitones. These differences account for the differences in sound of the scales across the groups.

A few of the scales that follow are so extremely altered compared with familiar scales that it's very awkward to name them using the `sharps and flats' method employed so far. In many cases, however, the scale can still be usefully thought of as a combination of several triad or seventh arpeggios. Consider, for example, the first group below, which contains the scale $2^{\Delta}$ + $4^{\Delta}$. If we were to name it as an alteration from a common heptatonic then we would probably have to call it `Super Locrian $\flat\flat$3 $\flat\flat$5 $\flat\flat\flat$6', which is not very informative. On the other hand, noticing that this scale is just a combination of two major seventh arpeggios (plus the root) is much more useful. This form will be used occasionally in the remainder of this chapter, and very often among the pentatonics where the `altered note' style doesn't work well.

In addition to the symbols for seventh arpeggios ($\Delta$, $d$, $\varnothing$ and $\circ$), we use `maj', `min' and `dim' to indicate major, minor and diminished triads; the symbol `+' is used for augmented triads as usual. The two uses of `+' shouldn't cause confusion; in the case mentioned above, something appears on either side of the symbol, whereas when it represents an augmented triad it stands alone, or next to a single number. For instance, it should be clear enough what `$2^{\Delta} + 4+$' would mean; the major seventh built on the second plus the augmented built on the fourth.

\subsection{The Yagapriya Scale Group}

The easiest scale to learn in this group is the Varunapriya $\flat$4, since it contains a major triad with two sets of consecutive semitones, one leading up to the root and the other to the major third. These give a sound that's very characteristic of early jazz; you can hear the clarinet playing figures from this scale\footnote{This is not, of course, to say that New Orleans clarinettists learned the Varunapriya $\flat$4 scale as a scale, or that it would have had that name.} on many tracks from the New Orleans period. Once these simple patterns are learned, however, the modes of this scale provide many other, fascinating sounds.

%~M|5|1,#2,#4,5,b6,#6,7|Susdim Dominant@Yagapriya
@$2^{\Delta}$ + $4^{\Delta}$@$\flat 2^{\Delta}$ + $3^{\Delta}$@Varunapriya $\flat$4@Vanaspati $\sharp$5@Phrygian Semiaugmented%

\subsection{The Nitimati Scale Group}
%~M|5|1,b2,bb3,b4,bb5,b6,bb7|$\flat 2^{\text{min}}$ + $2^{\text{min}}$@Dhenuka $\flat$4@Nitimati@Chakravakam $\sharp$5@Augmented $\sharp$2 $\sharp\sharp$4@Gayakapriya $\flat$5@Gangeyabhushani%

\subsection{The Kamavardhani Scale Group}
In Adam Kadmon's \emph{Guitar Grimoire} what I call the Kamavardhani Scale is called the `Composite II' scale.
%~M|5|1,b2,3,#4,5,b6,7|Kamavardhani@Semidominant Suspended@Jhankaradhvani $\flat$4@Phrygian Dominant Suspended@Suryakantam $\flat 5$@Chalanata $\sharp$5@Kanakangi%

\subsection{The Kanakangi $\flat$4 Scale Group}
%~M|5|1,b2,bb3,b4,5,b6,bb7|Kanakangi $\flat$4@Phrygian $\sharp$4 $\natural$7@Chitrambali $\sharp$3@Vagadhisvani $\sharp$5@
Ultra Locrian $\natural$4 $\flat$5`The Ultra Locrian scale itself contains the largest possible number of double-flat notes in any heptatonic scale, and is described below. This scale is a fairly close relative.
@Mayamalava Gaula $\flat$5@Chalanata%


\section{With One Major Third}

The interval maps of these scale groups all contain four semitones, two tones and one major third. This large interval gives them a very different sound from any of the scales examined thus far. The first three groups contain two pairs of consecutive semitones (separated, of course, in different ways) while the second set of three groups contains only one such pair.

\subsection{The Navanitam Scale Group}
%~M|5|1,b2,bb3,#4,5,6,b7|Navanitam@
Superaugmented $\flat$2 $\natural$6`This could also be referred to as the $4^{\text{maj}}$ + $\flat 2^d$ + $\flat$5 scale.
@Superaugmented $\sharp\sharp$2 $\sharp\sharp$4@Super Locrian $\flat\flat$6 $\flat\flat\flat$7@Harmonic Minor $\flat$5 $\flat\flat$6@Super Locrian $\flat\flat$5 $\natural$6@Harmonic Minor $\flat$4 $\sharp$5%	

\subsection{The $6^{\text{maj}}$ + $\flat 7^{\text{maj}}$ Scale Group}
The scales in this group are highly altered compared with the Major scale, but all of them contain a pair of major triads separated by a semitone. One additional note is, of course, required to give a heptatonic scale (since two triads can only contain at most six notes), and this can be thought of as  either the $\flat$3 of the lower triad or the 2 of the upper one. The former is perhaps more natural for most guitarists, since the addition of the $\flat$3 to a major triad is a common blues sound. 

%~M|5|1,b2,bb3,b4,4,6,b7|$6^{\text{maj}}$ + $\flat 7^{\text{maj}}$@$\sharp 5^{\text{maj}}$ + $6^{\text{maj}}$@$5^{\text{maj}}$ + $\flat 6^{\text{maj}}$@$4^{\text{maj}}$ + $\flat 5^{\text{maj}}$@$3^{\text{maj}}$ + $4^{\text{maj}}$@$1^{\text{maj}}$ + $\flat 2^{\text{maj}}$@$1^{\text{maj}}$ + $7^{\text{maj}}$%

\subsection{The $1^{\text{maj}}$ + $\flat 2^{\text{maj}}$ + 2 Scale Group}

These scales can all be considered as a pair of adjacent major triads with one note added above the root of the upper triad.

%~M|5|1,b2,bb3,b4,4,5,b6|$1^{\text{maj}}$ + $\flat 2^{\text{maj}}$ + 2@$1^{\text{maj}}$ + $7^{\text{maj}}$ + $\flat$2@$\flat 7^{\text{maj}}$ + $\natural 7^{\text{maj}}$@$\sharp 5^{\text{maj}}$ + $6^{\text{maj}}$ + $\flat$7@$5^{\text{maj}}$ + $\flat 6^{\text{maj}}$ + 6@$4^{\text{maj}}$ + $\flat 5^{\text{maj}}$ + 5@$3^{\text{maj}}$ + $4^{\text{maj}}$ + $\flat$5%


\subsection{The $\flat 5^d$ + $\flat 7^{\text{maj}}$ Scale Group}
%~M|5|1,b2,bb3,b4,4,b5,b7|$\flat 5^d$ + $\flat 7^{\text{maj}}$@$4^{d} + 6^{\text{maj}}$@$\flat4^{d} + \flat6^{\text{maj}}$@Navanitam $\sharp$5@$\flat2^{d} + \sharp3^{\text{maj}}$@$1^{d} + 3^{\text{maj}}$@$1^{\text{maj}} + \flat6^{d}$%

\subsection{The Jalarnavam Scale Group}
%~M|5|1,b2,bb3,b4,b5,bb6,b6|Super Locrian $\flat\flat$3 $\flat\flat$6 $\flat\flat\flat$7@Locrian $\flat\flat$6 $\natural$7@Naganandini $\flat$5@Dorian $\flat$4 $\sharp$5@Jalarnavam@Gamanasrama $\sharp$3@Superaugmented $\sharp\sharp$3%

\subsection{The Lydian Dominant $\sharp\sharp$2 $\sharp$3 $\sharp$5 Scale Group}
The scales in this group are very highly altered compared with any common scales, and do not contain enough conventional triad or seventh arpeggios to `cover' the whole scale. They therefore provide especially challenging and unusual sounds.
%~M|5|1,b2,bb3,b4,4,b5,b6|Super Locrian $\flat\flat$3 $\flat\flat$5 $\flat\flat\flat$6 $\flat\flat\flat$7@Super locrian $\flat\flat$5 $\flat\flat$6 $\natural$7
@Harmonic Minor $\flat$4 $\flat$5 $\sharp$6@Mixolydian $\flat\flat$3 $\flat$4 $\sharp$5@Augmented $\flat$2 $\flat$3 $\sharp\sharp$4@Augmented $\sharp\sharp$3 $\sharp\sharp$4 $\sharp$6@Lydian Dominant $\sharp\sharp$2 $\sharp$3 $\sharp$5%

\section{\emph{Melakatas} With Three Consecutive Semitones}

The twelve scale groups in this section are generated from the \emph{melakatas} containing three consecutive semitones. Included among their modes is the so-called Enigmatic scale, which was probably invented by Verdi and was made famous among guitarists by Joe Satriani in his track `The Enigmatic'. These scales are much more challenging than the foregoing scale groups when used in Western musical styles that have a tonal harmonic framework; notice how few of the modes listed here contain a standard seventh arpeggio, or even a triad. They do, however, provide very striking and interesting sounds.

As with pairs of consecutive semitones, you may want to explore the characteristic pattern of three consecutive semitones on its own as well as in the context of scales. First around the root:
\[
\begin{array}{cccc}
	%~Ex,NoDisplay|5|6,#6,7%   
	&%~Ex,NoDisplay|5|b2,#6,7%   
	&%~Ex,NoDisplay|5|7,b2,2%
	&%~Ex,NoDisplay|5|b2,2,b3%  \\
	\begin{array}{c}\text{From sixth}\\ \text{to root}\end{array} & 
	\begin{array}{c}\text{From flat seventh} \\ \text{to flat second}\end{array} & 
	\begin{array}{c}\text{From seventh} \\ \text{to second}\end{array} &
	\begin{array}{c}\text{From root} \\ \text{to flat third}\end{array} 
\end{array}
\]
then around the third:
\[
\begin{array}{ccc}
	%~Ex,NoDisplay|5|2,b3,3,4%   
	&%~Ex,NoDisplay|5|b3,3,4,b5%   
	&%~Ex,NoDisplay|5|3,4,b5,5%\\
	\begin{array}{c}\text{From second}\\ \text{to fourth}\end{array} & 
	\begin{array}{c}\text{From flat third} \\ \text{to flat fifth}\end{array} & 
	\begin{array}{c}\text{From third} \\ \text{to fifth}\end{array} 
\end{array}
\]
and finally around the fifth, completing all of the possibilities:
\[
\begin{array}{ccc}
	%~Ex,NoDisplay|5|4,b5,5,b6%   
	&%~Ex,NoDisplay|5|b5,5,b6,6%   
	&%~Ex,NoDisplay|5|5,b6,6,b7%\\
	\begin{array}{c}\text{From fourth}\\ \text{to sharp fifth}\end{array} & 
	\begin{array}{c}\text{From flat fifth} \\ \text{to sixth}\end{array} & 
	\begin{array}{c}\text{From fifth} \\ \text{to flat seventh}\end{array} 
\end{array}
\]
On their own these aren't very rewarding to study, but in the context of one of the scales below this kind of focus can be useful. It often seems to be the case that the more work a resource requires to make sense of it, the more rewards it offers in the end; possibly the work itself is as beneficial as the resource.

\subsection{The Ganamurti Scale Group}
%~M|5|1,b2,2,4,5,b6,7|Ganamurti@Visvambhari@Superaugmented $\sharp\sharp$5@Syamalangi@Phrygian Major $\flat$5 $\flat\flat$6@Ionian $\sharp$2 $\flat$5@Super Locrian $\flat\flat$3 $\flat\flat$7%

\subsection{The Manavati Scale Group}
%~M|5|1,b2,2,4,5,6,7|Manavati@Enigmatic@Superaugmented $\sharp$3 $\sharp\sharp$4 $\sharp\sharp$5@Kantamani@Mixolydian $\flat$5 $\flat\flat$6@Super Locrian $\natural$2 $\flat\flat$5@Super Locrian $\flat\flat$3%

\subsection{The Rupavati Scale Group}
%~M|5|1,b2,b3,4,5,#6,7|Rupavati@Ionian $\sharp$4 $\sharp\sharp$5 $\sharp$6@Mixolydian $\sharp\sharp$4 $\sharp$5@Lydian Dominant $\sharp$3 $\flat$6@Mixolydian $\sharp$2 $\flat$5 $\flat$6
@Ultra Locrian`This scale is named following the same logic as the Super Locrian; in that scale, every note than can be flat is flat, whereas here every note that can be double-flat is double-flat, the remaining notes being single-flat.
@Super Locrian $\flat\flat$3%

\subsection{The Salagam Scale Group}
%~M|5|1,b2,2,#4,5,b6,bb7|Salagam@Lyidan $\flat$2 $\sharp$3 $\flat$6@Lydian $\sharp\sharp$2 $\sharp$3 $\sharp$6@Ultra Locrian $\flat$5 $\flat\flat\flat$7@Ultra Locrian $\flat$5 $\natural$7@Ionian $\flat$2 $\flat$5 $\sharp$6@Ionian $\sharp$2 $\sharp\sharp$5 $\sharp$6%

\subsection{The Jhalavarali Scale Group}
%~M|5|1,b2,2,#4,5,b6,7|Jhalavarali
@Superaugmented $\flat$2 $\natural$5
@Superaugmented $\sharp\sharp$2 $\sharp\sharp$5
@Locrian $\flat\flat$2 $\flat\flat$6 $\flat\flat\flat$7
@Ionian $\flat$2 $\flat$5 $\flat\flat$6@Ionian $\sharp$2 $\flat$5 $\sharp$6@Super Locrian $\flat\flat$3 $\natural$5 $\flat\flat\flat$7%

\subsection{The Pavani Scale Group}
%~M|5|1,b2,2,#4,5,6,7|Pavani@Superaugmented $\flat$2@Ultra Augmented $\sharp$3@Locrian $\flat\flat$6 $\flat\flat\flat$7@
Major $\flat$5 $\flat\flat$6`This is a very funky sound to play over a $\Delta$ chord; it combines the $\sharp$4 (=$\flat$5) of the Lydian with the consecutive 4-$\flat$5-5 that's characteristic of the blues.
@Dorian $\flat$4 $\flat\flat$5@Locrian $\flat\flat$3 $\flat\flat$4 $\natural$5%


\subsection{The Gavambodhi Scale Group}
%~M|5|1,b2,b3,#4,5,b6,bb7|Gavambodhi@Lydian $\sharp$3 $\flat$6@
Mixolydian $\sharp$2 $\flat$5`The 7$\sharp$9 chord (sometimes called the Hendrix Chord) is a bluesy altered dominant sound that is often heard in jazzy vamps or brassy big-band arrangements. This scale suggests it and also provides the $\flat$5, which is another a blue note.
@Double Flat $\flat$5@Super Locrian $\flat\flat$3 $\flat$4@Hatakambari@Lydian $\sharp$2 $\sharp\sharp$5 $\sharp$6%

\subsection{The Divyamani Scale Group}
%~M|5|1,b2,b3,#4,5,#6,7|Divyamani@Lydian $\sharp$3 $\sharp\sharp$5 $\sharp$6@Mixolydian $\sharp$2 $\sharp\sharp$4 $\sharp$5@Locrian $\natural$3 $\flat\flat$6 $\flat\flat$7@Ionian $\sharp$2 $\flat$5 $\flat$6@Double Flat $\flat$6@Super Locrian $\flat\flat$3 $\natural$5 $\natural$7%

\subsection{The Dhavalambhari Scale Group}
%~M|5|1,b2,3,#4,5,b6,bb7|Dhavalambhari@Lydian $\sharp$2 $\sharp$3 $\flat$6@Aeolian $\flat$4 $\flat\flat$5 $\flat\flat$7@Double Flat $\flat$5 $\flat$7@Double Flat $\flat$5 $\natural$6 $\natural$7@Augmented $\flat$2 $\sharp$6@Ionian $\sharp$2 $\sharp\sharp$4 $\sharp\sharp$5 $\sharp$6%

\subsection{The Sucharitra Scale Group}
%~M|5|1,#2,3,#4,5,b6,bb7|Sucharitra@Super Locrian $\flat\flat$5 $\flat\flat\flat$6 $\flat\flat$7@Aeolian $\flat$4 $\flat\flat$5@Phrygian $\flat\flat$3 $\flat\flat$4 $\natural$6@Augmented $\flat$2 $\flat\flat$3@Augmented $\flat$2 $\sharp\sharp$4 $\sharp$6@Double Sharp $\sharp$2%

\subsection{The Super Locrian $\natural$2 $\flat\flat$7 Scale Group} %~M|5|1,2,b3,b4,b5,b6,bb7|Super Locrian $\natural$2 $\flat\flat$7@Super Locrian $\flat\flat$3 $\flat\flat$6@Melodic Minor $\flat$2 $\flat$5@Augmented $\sharp$6@Dorian $\sharp$4 $\sharp$5@Phrygian Major $\sharp$4@Lydian $\sharp$2 $\sharp$3%

\section{\emph{Melakatas} With Four Consecutive Semitones}

These two rather monstrous scale groups consist of five notes separated by a semitone, plus two more notes further away. They are included only because each contains a \emph{melakata}. As you might expect, these really do sound more like chromatic fragments than traditional scales.

\subsection{The Tanarupi Scale Group}
%~M|5|1,b2,2,4,5,#6,7|Tanarupi I@Tanarupi II@Tanarupi III@Tanarupi IV@Tanarupi V@Tanarupi VI@Tanarupi VII%

\subsection{The Raghupriya Scale Group}
%~M|5|1,b2,2,#4,5,#6,7|Raghupriya I@Raghupriya II@Raghupriya III@Raghupriya IV@Raghupriya V@Raghupriya VI@Raghupriya VII%


\section{With Three Consecutive Semitones: Remaining Possibilities}

The scales in this section cover all the remaining possibilities with three consecutive semitones but no more (i.e. we do not include heptatonics having four, five or six consecutive semitones). They are included mostly because they are not too numerous (there are only eight scale groups) rather than because they have known musical applications. But who knows? Weirder scales have been used in the past, and maybe you'll find an application for some of these. Because really useful names were hard to come by, and because in impressionist music long runs of chromatic notes like these are sometimes used to depict the wind, I've named each group after one of the eight gods of the winds in Greek mythology. I admit this is rather arbitrary and silly, but there it is; if you can think of better names for these scales then, as always, feel free to use them.

\subsection{The Borean Scale Group}
%~M|5|1,4,#4,5,6,b7,7|Borean Mode I@Borean Mode II@Borean Mode III@Borean Mode IV@Borean Mode V@Borean Mode VI@Borean Mode VII%
\subsection{The Notoean Scale Group}
%~M|5|1,4,#4,5,b6,b7,7|Notoean Mode I@Notoean Mode II@Notoean Mode III@Notoean Mode IV@Notoean Mode V@Notoean Mode VI@Notoean Mode VII%
\subsection{The Zephyrean Scale Group}
%~M|5|1,3,#4,5,6,b7,7|Zephyrean Mode I@Zephyrean Mode II@Zephyrean Mode III@Zephyrean Mode IV@Zephyrean Mode V@Zephyrean Mode VI@Zephyrean Mode VII%
\subsection{The Eurean Scale Group}
%~M|5|1,3,#4,5,b6,6,7|Eurean Mode I@Eurean Mode II@Eurean Mode III@Eurean Mode IV@Eurean Mode V@Eurean Mode VI@Eurean Mode VII%
\subsection{The Kaikian Scale Group}
%~M|5|1,3,4,5,b6,6,b7|Kaikian Mode I@Kaikian Mode II@Kaikian Mode III@Kaikian Mode IV@Kaikian Mode V@Kaikian Mode VI@Kaikian Mode VII%
\subsection{The Apeliotean Scale Group}
%~M|5|1,3,4,#4,5,6,7|Apeliotean Mode I@Apeliotean Mode II@Apeliotean Mode III@Apeliotean Mode IV@Apeliotean Mode V@Apeliotean Mode VI@Apeliotean Mode VII%
\subsection{The Skironean Scale Group}
%~M|5|1,3,4,#4,5,6,b7|Skironean Mode I@Skironean Mode II@Skironean Mode III@Skironean Mode IV@Skironean Mode V@Skironean Mode VI@Skironean Mode VII%
\subsection{The Lipsean Scale Group}
%~M|5|1,b3,4,5,b6,6,b7|Lipsean Mode I@Lipsean Mode II@Lipsean Mode III@Lipsean Mode IV@Lipsean Mode V@Lipsean Mode VI@Lipsean Mode VII%



\chapter{\mbox{Octatonics I:} \titlebreak \mbox{Two} \mbox{Consecutive} \mbox{Semitones}}

This chapter lists all of the possible octatonic (eight-note) scales with the exception of those that are symmetrical or that contain more than two consecutive semitones. There are two possible interval compositions for these scales: 5s 2t 1mT and 4s 4t, and in total only seven scale groups meet the criteria. For each octatonic there are several heptatonics with which it is identical except for one `chromatic' note. Since as improvisers we often insert passing notes into our phrases, the chances are you have played some of these scales many times before without knowing it.

The naming of these scales is rather arbitrary. A number of different conventions have been mixed in what follows to give the reader some idea of their structures. Of course, any group that contains a scale called the $4^{\text{dom}}$ + $5^\circ$ Scale can have all of its members named after a dominant chord rooted somewhere plus the diminished seventh chord rooted a tone above it. Likewise, you will see names indicating combinations of three or even four triads, the Common Pentatonic with one or more arpeggios added and so on. If you are going to study a scale like the ones in this chapter then you can and should investigate all such possibilities; the names given here offer mere hints to the complexities they contain.

\section{The `Parent Scales' Concept}

LEGO Bricks:

Major Parent Scale = Major Add b6
Dominant Parent Scale = Major Add b7
Minor 7 Parent Scale = Dorian Add nat3 (I think Major Add b5 but check) c d eb e f g  bb c
Half Diminished Parent Scale = c db eb f gb g ab bb c
Altered DOminant Parent Scale = c db eb e gb ab bb b c
Minor Major Prent Scle = c d eb f g ab a b c
Lydian Dominant Parent Scale = c de e gb g a bb b c
Diminished Parent Scale = c db eb e gb g a bb c


\section{The Major Add $\flat$6 Scale Group}
The $\flat$6 is the only note that can be added to the Major scale without creating an interval map of three or more consecutive semitones. For example, adding the $\flat$2 gives a cluster of four notes, 7 1 $\flat$2 2, separated by semitones. The Mixolydian Add $\flat$2 could also be thought of as the $2^{\text{m7}}$ + $5^\circ$ scale, since it's a combination of those two seventh arpeggios; one of the scales is named this way because its root is the `added' note, and so it can't be named after one of the Major Scale's modes. It could also be considered a mode of the Harmonic Minor Add $\flat$7 (see the third scale in this group). The Major Add $\flat$6 and the Aeolian Add $\natural$7 were at one time named by Allan Holdsworth as two of his favourite scales.

%~M|5|1,b2,2,3,4,5,6,b7|Mixolydian Add $\flat$2@$\flat 2^{\text{m7}}$ + $\flat 5^{\circ}$@Aeolian Add $\natural$7@Locrian Add $\natural$6@Major Add $\flat$6@Dorian Add $\sharp$4@Phrygian Add $\natural$3@Lydian Add $\sharp$2%

\section{The $\sharp 4^{\text{m7}}$ + $5^{\text{min}}$ Scale Group}

Although these scales are derived from the Melodic Minor with an added $\sharp$4, the names become unwieldy, so the combination-of-arpeggios approach has been taken instead. Five of the scales in this group contain a mode of the Common Minor Pentatonic (the others contain a Common Pentatonic shape that doesn't include the scale's root). This observation is not always immediately useful but it's been mentioned in the names where it seems apposite.

%~M|5|1,b2,2,3,#4,5,6,b7|$\sharp 4^{\text{m7}}$ + $5^{\text{min}}$@$4^{\varnothing}$ + $\flat 5^{\text{min}}$@$3^{\varnothing}$ + $4^\circ$@$\flat 3^\circ$ + $\flat 7^{\text{dom}}$@$1^{\text{maj}}$ + $\flat$6 Common Minor Pentatonic@Common Minor Pentatonic + $7^{\text{dom}}$@$4^{\text{dom}}$ + $5^\circ$@$1^\circ$ + $3^{\text{dom}}$%

\section{The $1^{\text{maj}}$ + $\flat 2^{\text{maj}}$ + $\flat 7^{\text{maj}}$ Scale Group}

Again, although derived from the Melodic Minor with an added $\flat$6, the arpeggio names for these scales are more enlightening. Allan Holdsworth listed the $1^{\text{min}}$ + $2^{\text{min}}$ + $7^\circ$ Scale (under a different name, of course) among his more commonly-used scales.

%~M|5|1,b2,2,3,4,5,b6,b7|$1^{\text{maj}}$ + $\flat 2^{\text{maj}}$ + $\flat 7^{\text{maj}}$@$1^{\text{maj}}$ + $6^{\text{maj}}$ + $7^{\text{maj}}$@$1^{\text{dim}}$ + $\flat 7^{\text{maj}}$ + $7^{\text{dim}}$@$1^\circ$ + $\flat 7^\varnothing$@$1^{\text{min}}$ + $2^{\text{min}}$ + $7^\circ$@$1^{\text{min}}$ + $\flat 2^+$ + $\flat 3^{\text{min}}$@$1^+$ + $2^{\text{min}}$ + $7^{\text{min}}$@$1^{\text{maj}}$ + $2^{\text{maj}}$ + $\flat 3^{\text{min}}$%


\section{The $2^{\text{maj}}$ + $6^{\text{min}}$ + $\flat 7^{\text{min}}$ Scale Group}
%~M|5|1,b2,2,3,4,b5,6,b7|$2^{\text{maj}}$ + $6^{\text{min}}$ + $\flat 7^{\text{min}}$@$\flat 2^+$ + $\sharp 5^{\text{maj}}$ + $6^{\text{maj}}$@$1^{\text{maj}}$ + $5^{\text{min}}$ + $\flat 6^{\text{min}}$@$\flat 2^{\text{maj}}$ + $2^d$ + $\flat 7^{\text{maj}}$@$1^{\text{maj}}$ + $\flat 2^d$ + $6^{\text{maj}}$@$1^d$ + $\sharp 5^{\text{maj}}$ + $7^{\text{maj}}$@$1^{\text{maj}}$ + $\flat 2^{\text{maj}}$ + $6^{\text{dim}}$@$1^{\text{maj}}$ + $\flat 6^{\text{dim}}$ + $7^{\text{maj}}$%

\section{The $1^{\text{maj}}$ + $\flat 2^{\text{maj}}$ + $2^{\text{min}}$ Scale Group}
%~M|5|1,b2,2,3,4,5,b6,bb7|$1^{\text{maj}}$ + $\flat 2^{\text{maj}}$ + $2^{\text{min}}$@$1^{\text{maj}}$ + $\flat 2^{\text{min}}$ + $7^{\text{maj}}$@$1^{\text{min}}$ + $\flat 7^{\text{maj}}$ + $7^{\text{maj}}$@$1^+$ + $6^{\text{dim}}$ + $\flat 7^{\text{min}}$@$1^{\text{maj}}$ + $\flat 6^{\text{dim}}$ + $6^{\text{dim}}$@$\flat 2^+$ + $2^+$ + $5^{\text{dim}}$@$\flat 2^+$ + $3^{\text{maj}}$ + $\flat 5^{\text{dim}}$@Minor Pentatonic + $3^{\text{maj}}$%

\section{The $1^+$ + $\flat 2^+$ + $2^{\text{maj}}$ Scale Group}
%~M|5|1,b2,2,3,4,b5,b6,bb7|$1^+$ + $\flat 2^+$ + $2^{\text{maj}}$@$1^{\text{maj}}$ + $\flat 2^{\text{maj}}$ + $7^+$@$1^{\text{maj}}$ + $2^+$ + $7^{\text{maj}}$@$1^+$ + $\flat 2^+$ + $\flat 7^{\text{maj}}$@$1^{\text{maj}}$ + $\flat 6^{\text{maj}}$ + $6^{\text{maj}}$@$1^{m7}$ + $7^{m6}$@$1^+$ + $\flat 2^+$ + $\flat 5^{\text{maj}}$@$1^{\text{maj}}$ + $4^d$ + $\flat 6^{\text{min}}$%

\section{The $1^+$ + $\flat 5^+$ + $\flat 7^{\text{min}}$ Scale Group}
%~M|5|1,b2,2,3,4,b5,b6,b7|$1^+$ + $\flat 5^+$ + $\flat 7^{\text{min}}$@$1^{\text{maj}}$ + $\flat 2^+$ + $\flat 7^+$@$1^\varnothing$ + $3^d$@$1^+$ + $2^+$ + $6^{\text{maj}}$@$1^{m6}$ + $\flat 2^d$@$1^d$ +  $7^{m6}$@$1^+$ + $2^+$ + $2^{\text{min}}$@$1^{\text{maj}}$ + $2^+$ + $\flat 6^{\text{maj}}$%







\chapter{\mbox{Octatonics II:} \titlebreak \mbox{Three} \mbox{Consecutive} \mbox{Semitones}}

This chapter collects all octatonics that have exactly three consecutive semiones in their interval maps. Previous editions of this book omitted these scales but they do have applications and may be worth looking into more deeply. A couple are used quite frequently in jazz and related musics, or so some theorists claim; the author suspects, but can hardly be expected to prove, that lines that appear to be made up of these `scales' are often the result of some other, not entirely scalar, thought process. In any case, theorists often give some of them as `parent scales' and it may be useful to have fingerings for all of them available for reference. Also included is the African-American Blues Scale Group, which contins four consecutive semitones but is an important scale in George Russell's system (in previous editions this scale was found in the previous chapter).

Attempts to find sensible names for the modes in these groups have been largely fruitless, so a simple numbering approach has been taken.

\section{The Major Scale with an Added Note}

\subsection{The Major Add $\flat$2 Scale Group}
%~M|5|1,2,3,#4,5,b6,6,7|x@x@x@x@x@x@x@x%

\subsection{The Major Add $\sharp$2 Scale Group}
%~M|5|1,2,#2,3,4,5,6,7|Major Add $\sharp$2 Mode I@Major Add $\sharp$2 Mode II@Major Add $\sharp$2 Mode III@Major Add $\sharp$2 Mode IV@Major Add $\sharp$2 Mode V@Major Add $\sharp$2 Mode VI@Major Add $\sharp$2 Mode VII@Major Add $\sharp$2 Mode VIII%


\subsection{The Major Add $\flat$5 Scale Group}
%~M|5|1,2,3,4,b5,5,6,7|x@x@x@x@x@x@x@x%

\subsection{The Major Add $\flat$7 Scale Group}
Allan Holdsworth once listed the first scale in this group as one of his most commonly-used scales.
%~M|5|1,2,3,4,5,6,b7,7|Major Add $\flat$7 Mode I@Major Add $\flat$7 Mode II@Major Add $\flat$7 Mode III@Major Add $\flat$7 Mode IV@Major Add $\flat$7 Mode V@Major Add $\flat$7 Mode VI@Major Add $\flat$7 Mode VII@Major Add $\flat$7 Mode VIII%


\section{Other Heptatonics with One Added Note}

Each group in this section is an altered-note heptatonic (or, in the case of the first one, a Harmonic Minor) with an added note.

\subsection{The Harmonic Minor Add $\sharp$4 Scale Group}
%~M|5|1,2,b3,4,#4,5,b6,7|x@x@x@x@x@x@x@x%

\subsection{The Sulini Add $\flat$7 Scale Group}
%~M|5|1,#2,3,4,5,6,b7,7|Sulini Add $\flat$7 Mode I@Sulini Add $\flat$7 Mode II@Sulini Add $\flat$7 Mode III@Sulini Add $\flat$7 Mode IV@Sulini Add $\flat$7 Mode V@Sulini Add $\flat$7 Mode VI@Sulini Add $\flat$7 Mode VII@Sulini Add $\flat$7 Mode VIII%

\subsection{The Naganandini Add $\sharp$2 Scale Group}
%~M|5|1,2,#2,3,4,5,b6,7|Naganandini Add $\sharp$2 Mode I@Naganandini Add $\sharp$2 Mode II@Naganandini Add $\sharp$2 Mode III@Naganandini Add $\sharp$2 Mode IV@Naganandini Add $\sharp$2 Mode V@Naganandini Add $\sharp$2 Mode VI@Naganandini Add $\sharp$2 Mode VII@Naganandini Add $\sharp$2 Mode VIII%

\subsection{The Ionian $\flat$5 Add $\flat$2 Scale Group}
%~M|5|1,b2,2,3,4,b5,6,7|Ionian $\flat$5 Add $\flat$2 Mode I@Ionian $\flat$5 Add $\flat$2 Mode II@Ionian $\flat$5 Add $\flat$2 Mode III@Ionian $\flat$5 Add $\flat$2 Mode IV@Ionian $\flat$5 Add $\flat$2 Mode V@Ionian $\flat$5 Add $\flat$2 Mode VI@Ionian $\flat$5 Add $\flat$2 Mode VII@Ionian $\flat$5 Add $\flat$2 Mode VIII%

\subsection{The Ionian $\flat$5 Add $\sharp$2 Scale Group}
%~M|5|1,2,#2,3,4,b5,6,7|Ionian $\flat$5 Add $\sharp$2 Mode I@Ionian $\flat$5 Add $\sharp$2 Mode II@Ionian $\flat$5 Add $\sharp$2 Mode III@Ionian $\flat$5 Add $\sharp$2 Mode IV@Ionian $\flat$5 Add $\sharp$2 Mode V@Ionian $\flat$5 Add $\sharp$2 Mode VI@Ionian $\flat$5 Add $\sharp$2 Mode VII@Ionian $\flat$5 Add $\sharp$2 Mode VIII%

\subsection{The Ionian $\flat$5 Add $\flat$7 Scale Group}
%~M|5|1,2,3,4,b5,6,b7,7|Ionian $\flat$5 Add $\flat$7 Mode I@Ionian $\flat$5 Add $\flat$7 Mode II@Ionian $\flat$5 Add $\flat$7 Mode III@Ionian $\flat$5 Add $\flat$7 Mode IV@Ionian $\flat$5 Add $\flat$7 Mode V@Ionian $\flat$5 Add $\flat$7 Mode VI@Ionian $\flat$5 Add $\flat$7 Mode VII@Ionian $\flat$5 Add $\flat$7 Mode VIII%

\subsection{The Sarasangi Add $\flat$2 Scale Group}
%~M|5|1,b2,2,3,4,5,b6,7|Sarasangi Add $\flat$2 Mode I@Sarasangi Add $\flat$2 Mode II@Sarasangi Add $\flat$2 Mode III@Sarasangi Add $\flat$2 Mode IV@Sarasangi Add $\flat$2 Mode V@Sarasangi Add $\flat$2 Mode VI@Sarasangi Add $\flat$2 Mode VII@Sarasangi Add $\flat$2 Mode VIII%

\subsection{The Sarasangi Add $\sharp$2 Scale Group}
%~M|5|1,2,#2,3,4,5,b6,7|Sarasangi Add $\sharp$2 Mode I@Sarasangi Add $\sharp$2 Mode II@Sarasangi Add $\sharp$2 Mode III@Sarasangi Add $\sharp$2 Mode IV@Sarasangi Add $\sharp$2 Mode V@Sarasangi Add $\sharp$2 VI@Sarasangi Add $\sharp$2 Mode VII@Sarasangi Add $\sharp$2 Mode VIII%


\subsection{The Ultra Locrian Add $\sharp$4 Scale Group}
%~M|5|1,b2,bb3,b4,bb5,#4,bb6,bb7|Ultra Locrian Add $\sharp$4 Mode I@Ultra Locrian Add $\sharp$4 Mode II@Ultra Locrian Add $\sharp$4 Mode III@Ultra Locrian Add $\sharp$4 Mode IV@Ultra Locrian Add $\sharp$4 Mode V@Ultra Locrian Add $\sharp$4 Mode VI@Ultra Locrian Add $\sharp$4 Mode VII@Ultra Locrian Add $\sharp$4 Mode VIII%

\subsection{The Kantamani Add $\sharp$2 Scale Group}
%~M|5|1,2,#2,3,#4,5,b6,bb7|Kantamani Add $\sharp$2 Mode I@Kantamani Add $\sharp$2 Mode II@Kantamani Add $\sharp$2 Mode III@Kantamani Add $\sharp$2 Mode IV@Kantamani Add $\sharp$2 Mode V@Kantamani Add $\sharp$2 Mode VI@Kantamani Add $\sharp$2 Mode VII@Kantamani Add $\sharp$2 Mode VIII%


\subsection{The Pavani Add $\flat$6 Scale Group}
This is the only octatonic that contains a major third and still manages to avoid having more than three consecutive semitones.

%~M|5|1,b2,2,#4,5,b6,6,7|Pavani Add $\flat$6 Mode I@Pavani Add $\flat$6 Mode II@Pavani Add $\flat$6 Mode III@Pavani Add $\flat$6 Mode IV@Pavani Add $\flat$6 Mode V@Pavani Add $\flat$6 Mode VI@Pavani Add $\flat$6 Mode VII@Pavani Add $\flat$6 Mode VIII%

\section{Others}

\subsection{The African-American Blues Scale Group}
The scale that names this group (the first in the list below) is one of George Russell's. The name itself is probably more of historical than music-theoretical interest today. There are four consecutive semitones in these scales; the group is included because of its important place in the Lydian Chromatic system, and students of Russell's book will no doubt find it useful to have the fingerings to hand.

%~M|5|1,#2,3,4,#4,5,6,b7|African-American Blues I@African-American Blues II@African-American Blues III@African-American Blues IV@African-American Blues V@African-American Blues VI@African-American Blues VII@African-American Blues VIII%




\chapter{\mbox{Nonatonics}}

Since they all have more than two consecutive semitones, there are no nonatonic (nine-note) scales that meet our criteria. Yet there are two rather special non-symmetrical nonatonic groups that have only three consecutive semitones. It so happens that they can both be thought of as a Major scale with an added $\flat$6 and either an added $\flat$2 or $\sharp$2.

\section{The Major Add $\flat$2$\flat$6 Scale Group}
This scale can be visualised as a Major Scale plus a major triad built on the $\flat$2; the third of the major triad falls on the fourth of the scale.
%~M|5|1,b2,2,3,4,5,b6,6,7|Major Add $\flat$2$\flat$6 Mode I@Major Add $\flat$2$\flat$6 Mode II@Major Add $\flat$2$\flat$6 Mode III@Major Add $\flat$2$\flat$6 Mode IV@Major Add $\flat$2$\flat$6 Mode V@Major Add $\flat$2$\flat$6 Mode VI@Major Add $\flat$2$\flat$6 Mode VII@Major Add $\flat$2$\flat$6 Mode VIII@Major Add $\flat$2$\flat$6 Mode IX%

\section{The Major Add $\sharp$2$\flat$6 Scale Group}
This scale can be visualised as a Major Scale plus a major triad built on the $\flat$6; the third of the major triad falls on the root of the scale.
%~M|5|1,2,#2,3,4,5,b6,6,7|Major Add $\sharp$2$\flat$6 Mode I@Major Add $\sharp$2$\flat$6 Mode II@Major Add $\sharp$2$\flat$6 Mode III@Major Add $\sharp$2$\flat$6 Mode IV@Major Add $\sharp$2$\flat$6 Mode V@Major Add $\sharp$2$\flat$6 Mode VI@Major Add $\sharp$2$\flat$6 Mode VII@Major Add $\sharp$2$\flat$6 Mode VIII@Major Add $\sharp$2$\flat$6 Mode IX%




\part{Hexatonics}

\chapter{Coscales and Reflections}

I've separated the hextonic scales and placed them at the end of this book for two reasons. The first is that they are probably not as useful for serious study as the pentatonics and heptatonics from which, in different ways, they can easily be derived. The second is that the theoretical framework used to present these scales will be very different from that used in the foregoing chapters. 

This chapter is more abstract and theoretical than the rest of the book. This is intentional. The language of symmetry developed in this chapter provides us with an interesting and, I think, fairly compelling categorisation of the hexatonics without considering them to be either pentatonics with one added note or heptatonics with one note omitted, neither of which is especially illuminating. You can still use the diagrams in the chapters that follow to experiment with the sounds of these scales in any way you wish even if you don't read any further in this chapter. If you do but find the terminology introduced here to be unhelpful you should of course disregard it.

What follows in this chapter has roots in the set-theoretical work of musicologists such as Allen Forte, George Perle and David Lewin. Their work contains far more than I introduce here, in part because their concerns are usually rather distant from our concern with objects such as scales and arpeggios with root notes. As a consequence I freely adapt their ideas and terminologies to fit more easily into our current context.

\section{Hexatonic Shapes on the Fingerboard}

As you will know by now, all the resources in this book are presented in CAGED form in order to give a consistent view of the whole fingerboard. Many hexatonics, however, can be played rather logically in a shape that ascends the neck as you go up from low strings to high and \emph{vice versa}. This is based on the observation about the way octaves are laid out on the neck, and it works especially well when the hexatonic makes a fairly easy `shape' with three notes on one string and three more on the next. In this case the shape can often conveniently be moved up the fingerboard to create nice, simple fingerings that work well for quick scalar runs.

Start with the shape on the bottom E and A strings and notice how the six notes of the scale sit on these two strings. Now jump up two frets and play the same pattern on the D and G strings. Finally jump up three frets and play the same thing again on the B and E strings. If there's a bit of a stretch in the patterns then that can actually make these jumps easier to execute. This idea, of course, works for any kind of scale, but it's particularly applicable to hexatonics since playing three notes consistently on every string is convenient for economy picking.

\section{Binary Representations of Scales}

In order to work easily with the new kinds of relationship described here it will be helpful to have a slightly different way to represent a scale. It's a very simple binary representation, and the easiest way to explain it is with an example. Here is the binary representation of the major scale:
\begin{quote}
\begin{tabular}{rrrrrrrrrrrrrrr}
	1 & 0 &1 & 0 &1 & 1 & 0 & 1 & 0 & 1 & 0 & 1
\end{tabular}
\end{quote}
This number -- 101011010101, or 2773 in decimal -- uniquely represents the Major scale. The method of representation is simple: each position, reading left to right, represents a note: 1, $\flat2$, 2 and so on. A `1' in that position means the note is included in the scale and a `0' means it is not included. So we have:
\begin{quote}
\begin{tabular}{rrrrrrrrrrrrrrr}
	1 & 0 &1 & 0 &1 & 1 & 0 & 1 & 0 & 1 & 0 & 1 \\
	1 & $\flat2$ &2 & $\flat3$ &3 & 4 & $\flat5$ & 5 & $\flat6$ & 6 & $\flat7$ & 7
\end{tabular}
\end{quote}
I'm sure you can see that the `1's match up with the notes of the major scale and the `0's match up with notes that the Major scale does \emph{not} contain. This simple device will enable us to work a little more easily with hexatonic scales.

As an example, note that modes of a scale are obtained by the technique known as `bit rotation' -- that is, by repeatedly moving the last digit to the start of the number until we end up back where we started:
\begin{quote}
\begin{tabular}{rrrrrrrrrrrrrrr}
	1 & 0 &1 & 0 &1 & 1 & 0 & 1 & 0 & 1 & 0 & 1 \\
	1 & 1  & 0 &1 & 0 &1 & 1 & 0 & 1 & 0 & 1 & 0 \\
	0 & 1 & 1 & 0 &1 & 0 &1 & 1 & 0 & 1 & 0 & 1 \\
	1 & 0 & 1 & 1 & 0 &1 & 0 &1 & 1 & 0 & 1 & 0 \\
	0 & 1 & 0 & 1 & 1 & 0 &1 & 0 &1 & 1 & 0 & 1 \\
	1 & 0 & 1 & 0 & 1 & 1 & 0 &1 & 0 &1 & 1 & 0 \\
	0 & 1 & 0 & 1 & 0 & 1 & 1 & 0 &1 & 0 &1 & 1  \\
	1 & 0 & 1 & 0 & 1 & 0 & 1 & 1 & 0 &1 & 0 &1 \\
	1 & 1 & 0 & 1 & 0 & 1 & 0 & 1 & 1 & 0 &1 & 0 \\
	0 &1 & 1 & 0 & 1 & 0 & 1 & 0 & 1 & 1 & 0 &1 \\
	1 & 0 &1 & 1 & 0 & 1 & 0 & 1 & 0 & 1 & 1 & 0 \\
	0 &1 & 0 &1 & 1 & 0 & 1 & 0 & 1 & 0 & 1 & 1 \\
\end{tabular}
\end{quote}
We would normally eliminate all those rows that begin with a zero, since they don't contain a root note. This leaves us with the Major scale modes, as you can check for yourself:
\begin{quote}
\begin{tabular}{rrrrrrrrrrrrrrr}
	1 & 0 &1 & 0 &1 & 1 & 0 & 1 & 0 & 1 & 0 & 1 & Ionian\\
	1 & 1  & 0 &1 & 0 &1 & 1 & 0 & 1 & 0 & 1 & 0 & Locrian \\
	0 & 1 & 1 & 0 &1 & 0 &1 & 1 & 0 & 1 & 0 & 1 & \\
	1 & 0 & 1 & 1 & 0 &1 & 0 &1 & 1 & 0 & 1 & 0 & Aeolian\\
	0 & 1 & 0 & 1 & 1 & 0 &1 & 0 &1 & 1 & 0 & 1 &\\
	1 & 0 & 1 & 0 & 1 & 1 & 0 &1 & 0 &1 & 1 & 0 &  Mixolydian \\
	0 & 1 & 0 & 1 & 0 & 1 & 1 & 0 &1 & 0 &1 & 1  & \\
	1 & 0 & 1 & 0 & 1 & 0 & 1 & 1 & 0 &1 & 0 &1 &  Lydian \\
	1 & 1 & 0 & 1 & 0 & 1 & 0 & 1 & 1 & 0 &1 & 0 & Phrygian \\
	0 &1 & 1 & 0 & 1 & 0 & 1 & 0 & 1 & 1 & 0 &1 & \\
	1 & 0 &1 & 1 & 0 & 1 & 0 & 1 & 0 & 1 & 1 & 0 & Dorian \\
	0 &1 & 0 &1 & 1 & 0 & 1 & 0 & 1 & 0 & 1 & 1 & \\
\end{tabular}
\end{quote}
This enables us to generate the scale group from a given scale very quickly and easily, and was in fact the method used to make the initial survey of all possible scales in the preparation of this book\footnote{This idea, and an initial version of the algorithm for generating the scales, was provided to the author by drummer, computer programmer and mathematician Phil O'Donnell.}.

\section{The Coscale Relationship}

An obvious operation to perform on a binary representation is what might be called `complementation' -- that is, flipping all the `1's to `0's and \emph{vice versa}. In the case of the Major scale (top row) we obtain:
\begin{quote}
\begin{tabular}{rrrrrrrrrrrrrrr}
	1 & 0 &1 & 0 &1 & 1 & 0 & 1 & 0 & 1 & 0 & 1 \\
	0 & 1 &0 & 1 &0 & 0 & 1 & 0 & 1 & 0 & 1 & 0 \\
\end{tabular}
\end{quote}
Clearly no scale's complement is itself a scale because a scale must always have a root note and the root note will be removed by complementation. By rotating the pattern obtained by complementation, however, we arrive at
\begin{quote}
\begin{tabular}{rrrrrrrrrrrrrrr}
	1 &0 & 1 &0 & 0 & 1 & 0 & 1 & 0 & 1 & 0 & 0 \\
\end{tabular}
\end{quote}
which is the Common Major Pentatonic. This relationship is a relationship between scale \emph{groups}: we say that the Common Major Pentatonic scale group is the complement of the Major scale group. As a useful shorthand, we can say that the Common Major Pentatonic is \emph{a coscale of} the Major scale. To see why this is shorthand, and why the relationship is really at the group level, consider that the C Major scale and F$\sharp$ Common Major Pentatonic are coscales, yet so are the C Major scale and E$\flat$ Common Minor Pentatonic. We can pick any modes in the two scale groups and describe a coscale relationship between them simply by choosing roots that are the right distance apart.

The coscale relationship clearly goes both ways, since performing complementation on a scale from the Common Pentatonic group just gives us (after rotation) one from the Major scale group again. We say that the coscale relation is `reflexive': if $X$ is a coscale of $Y$ it follows that $Y$ is a coscale of $X$.

Any scale of $n$ notes has a coscale of $12 - n$ notes. This tells us that the coscale of a hexatonic scale is another hexatonic, and agrees with the fact that the Major scale, a heptatonic, would have a pentatonic coscale (since $7 + 5 = 12$). The Whole Tone scale is a coscale of itself, and since any transformation that leaves what it acted on unchanged is usually called a `symmetry' we say that the Whole Tone scale exhibits \emph{coscale symmetry}.

A scale plus its complement contain together all the notes in the Chromatic scale without any overlap. This fact has been used by classical composers, especially of atonal music, since the early twentieth century. It should be noted, however, that composers in this tradition have not usually thought about hexatonic \emph{scales} so much as of complementary pairs of six-note sets without root notes; such structures are subtly but importantly different from those discussed in this book.

\section{Scale Reflections}

The second type of scale relationship is what I call a `reflection'. It simply involves reversing the binary representation of a scale left-to-right. Taking the Major scale once again as our canonical example we obtain:
\begin{quote}
\begin{tabular}{rrrrrrrrrrrrrrr}
	1 & 0 &1 & 0 &1 & 1 & 0 & 1 & 0 & 1 & 0 & 1 \\
	1 & 0 &1 & 0 &1 & 0 & 1 & 1 & 0 & 1 & 0 & 1 \\
\end{tabular}
\end{quote}
which is the Lydian scale. Hence the Major scale group is a reflection of itself, in the sense that any scale in the group, when reflected, generates the same group again.

Note that a reflection may or may not be a scale as it stands; it will depend on whether the $\natural7$ occurs in the original since this is the note that becomes the root when the scale is reflected. As in the coscale case, however, since we're interested in scale groups rather than individual scales we'll naturally rotate the result as necessary to generate the group in question.

Not all scale groups are their own reflections; only five of the hexatonic groups we consider in the chapters that follow have this property. They exhibit what I call \emph{reflexive symmetry}; diagrams of their reflections are not given in the text (since they're the same as the original diagrams) but for the sake of clarity I point out where this is the case.

We'll also consider the reflections of coscales, and where a scale group is identical to its own coscale reflection, we say that it exhibits \emph{coscale reflexive symmetry}. As with reflexive symmetry, we do not print the redundant scale group. It is perhaps initially surprising that a scale can exhibit coscale reflexive symmetry but not reflexive symmetry, yet multiple examples are found in the chapters that follow.

There is a very close affinity between what I call `reflection' and what is usually called `inversion'. Within some theoretical frames, in fact, the operations are the same. I've chosen not to use the term `inversion' here, however, since it has a number of subtly different uses in music theory. Under the most common definition it's actually incoherent to talk about an inversion without stating the pitch about which the inversion takes place, which doesn't make much sense for us: this is a difference in emphasis between classical and pitch-class theories that's likely to cause confusion and, since this is not a book about either much less one in which that confusion can be properly dispelled, the adoption of `reflection' rather than `inversion' would seem to be sensible. Hardcore pitch class theorists will be more than capable, I'm sure, of doing the translation in their heads.


\section{Categorisation of Hexatonics}

The hextonic scales are categorised according to the following system, building on our previous practice of using interval composition as a key criterion. This is simply because interval composition has a significant effect on how a scale sounds, and we want our classification to be as musically useful as possible.

First we group scale groups into sets containing the original scale group, the coscale group and their reflections. Second we separate those sets in which the scale and coscale have different interval compositions: these are surprisingly small in number and are dealt with in the next chapter, where they're called `variant sets'. I do not know why there should be so few variant sets, except that the consecutive semitones rule eliminates some obvious cases; this might make an interesting mathematical puzzle for someone thus inclined. We then categorise the remaining, `invariant' sets into those with a major third and those without and then, within each of the two, by the number of minor thirds.

The existence of reflexive symmetry is an interesting property of a scale set but is probably not musically significant for most people, so it's not used for categorisation (reflexive symmetry itself is significant, at least for experienced listeners, and has been used as a musical device in composed music for a long time). It's remarked on in the text where present, and sets exhibiting reflexive or coscale reflexive symmetry are placed at the end of the section to which they belong.

It is also useful to pay attention to some of the features of these scales that we have called attention to among the pentatonics and heptatonics. I remark where the scales in a group are composed of two common triad or seventh arpeggios in a way that seems enlightening, or where one of the scales in a group is clearly just a Common Pentatonic (minor or major) with an added note. I also remark on cases where `leading notes', by which I just mean groups of notes separated by semitones, seem to me to be a prominent feature of a particular pattern. These comments are only designed to help you to understand these scales and their relationships, and as usual you are at liberty to ignore them.

A few readers may be interested in the algebraic properties of these operations. Consider four functions: reflection, complementation, complementation-followed-by-reflection and the identity operation that leaves a scale as it is. Then these functions with the composition operation form the Klein 4-group. In cases where symmetry exists we have smaller, cyclic groups. For example, in sets with coscale reflexive symmetry the coscale reflection function is exactly the same as the identity function (for our purposes). Hence the group is reduced to three elements, meaning it can only be $\mathbb{Z}_3$. A similar reduction happens in the presence of either reflexive or coscale symmetry (but not both simultaneously), giving the cyclic group $\mathbb{Z}_2$ that contains only the identity function and complementation. Finally, there is a small class identified by the trivial group $\mathbb{Z}_1$ containing only the identity, since these scales exhibit both coscale and reflexive symmetry.

In total there are two possible scale sets exhibiting coscale symmetry only, eleven exhibiting reflexive symmetry only, twenty exhibiting coscale reflexive symmetry and four exhibiting all of the symmetries at once. None of these sets contains more than three consecutive semitones so they're all represented in the chapters that follow except for those we've already met under the heading of `symmetrical scales'.


\chapter{\mbox{Variant} \mbox{Sets}}

This chapter considers those sets of hexatonic scale groups whose interval compositions are not symmetrical under the coscale relationship (i.e., in each case the scale's interval composition differs from that of its coscale) and that obey the rule about intervals wider than a major third. As we remarked in the previous chapter the majority of hexatonics we consider in this book are \emph{invariant} in this sense -- that is, their interval compositions stay the same when the coscale transformation is applied. In all cases the original scale in a variant set has interval composition 2s 3t MT, whereas its coscale's composition is 2s 2t 2mT. There are more variant sets in the final chapter, which lists the hexatonics that break the rule regarding large intervals.

\section{The 100010101101 Scale Set}
\subsection*{Original Scale}
These scales can be thought of as a major triad with a diminished triad a whole tone above it. This is most obvious in the fingerings of Mode II, here the major triad is at the root, and Mode III where the diminished triad is at the root. As Mode III indicates, you can also think of this group as the modal applications of the $\varnothing$13 arpeggio.

%~M|5|1,3,#4,b6,6,7|Mode I@Mode II@Mode III@Mode IV@Mode V@Mode VI%
\subsection*{Coscale}
Mode III shows the easiest way to visualise these fingerings -- as a major triad with an added 9 and two chromatic leading notes going up to the root (the $\flat7$ and $\natural7$).

%~M|5|1,b2,2,3,#4,6|Mode I@Mode II@Mode III@Mode IV@Mode V@Mode VI%
\subsection*{Reflection}
These scales are composed of a minor triad with the diminished triad a semitone below it. The Mode I scale shows this most obviously.

%~M|5|1,2,b3,4,5,7|Mode I@Mode II@Mode III@Mode IV@Mode V@Mode VI%
\subsection*{Coscale Reflection}
Mode III is probably the easiest to start with in this group: a dominant arpeggio with chromatic leading notes going up to the 3, giving a classic jazzy blues sound.

%~M|5|1,b3,4,5,b6,6|Mode I@Mode II@Mode III@Mode IV@Mode V@Mode VI%


\section{The 100010110101 Scale Set}
\subsection*{Original Scale}
This can be thought of as a minor triad with a diminished triad built a whole tone above it -- see Mode III. Contrast this with the original scale of the previous set, which was a \emph{major} triad with a diminished a whole tone above.
%~M|5|1,3,#4,5,6,7|Mode I@Mode II@Mode III@Mode IV@Mode V@Mode VI%
\subsection*{Coscale}
The Mode VI scale is just the Common Minor Pentatonic with an added $\natural3$, with the others having a similar relationship to those familiar shapes.
%~M|5|1,b2,2,3,5,6|Mode I@Mode II@Mode III@Mode IV@Mode V@Mode VI%
\subsection*{Reflection}
These scales are all composed of major triads with diminished triads a semitone below them.. Again compare them with the reflection scale group from the previous set.
%~M|5|1,2,3,4,5,7|Mode I@Mode II@Mode III@Mode IV@Mode V@Mode VI%
\subsection*{Coscale Reflection}
This scale group is the same as the Blues Scale Group we mentioned in Part I; a minor pentatonic with added $\flat5$, which acts as a blue note. The diagrams are not, therefore, reprinted here.

\section{The 100010110110 Scale Set}
This set exhibits reflexive symmetry; this means the set contains only the original and coscale groups.
\subsection*{Original Scale}
Mode I is perhaps best thought of as a dominant chord with added $\sharp11$ and 13.
%~M|5|1,3,#4,5,6,b7|Mode I@Mode II@Mode III@Mode IV@Mode V@Mode VI%
\subsection*{Coscale}
The easiest way to visualise these patterns is probably by looking at Mode I, which is a dominant chord with added chromatic notes leading up from the root.
%~M|5|1,b2,2,3,5,b7|Mode I@Mode II@Mode III@Mode IV@Mode V@Mode VI%

\section{The 100011010101 Scale Set}
This set also exhibits reflexive symmetry.
\subsection*{Original Scale}
Mode I is perhaps best thought of as a dominant chord with added $11$ and 13 (compare with the previous scale set).
%~M|5|1,3,4,5,6,7|Mode I@Mode II@Mode III@Mode IV@Mode V@Mode VI%
\subsection*{Coscale}
The Mode III scale is a Common Minor Pentatonic with an added $\natural7$, which suggests a way to `see' the other fingerings in this group.
%~M|5|1,b2,2,4,5,6|Mode I@Mode II@Mode III@Mode IV@Mode V@Mode VI%







\chapter{\mbox{Containing a} \mbox{Major} \mbox{Third}}

This chapter covers the invariant sets of hexatonics that contain a major third in their interval maps. These are mostly more dissonant or exotic in sound than those that do not contain a major third because the large interval causes the remaining notes to crowd together. This is even more pronounced when, in addition to the major third, the scale contains a minor third. In all but the first of these cases we find pairs of adjacent semitones, usually isolated from the other notes by large intervals. The scales in this chapter may therefore be very generally considered more difficult to use than those of the next. As usual, however, this suggests that they may also be more interesting.

\section{Containing a Minor Third}

\subsection{The 100011001101 Scale Set}
\subsection*{Original Scale}
This scale is formed from two major triads separated by a semitone.
%~M|5|1,3,4,b6,6,7|Mode I@Mode II@Mode III@Mode IV@Mode V@Mode VI%
\subsubsection*{Coscale}
The Mode III scale is probably the most obvious pattern here: a major triad with added $\sharp2$ ($=\flat3$) and leading notes going down to the root.
%~M|5|1,b2,2,4,#4,6|Mode I@Mode II@Mode III@Mode IV@Mode V@Mode VI%
\subsubsection*{Reflection}
This one is made up of two minor scales separated by a semitone.
%~M|5|1,2,b3,#4,5,7|Mode I@Mode II@Mode III@Mode IV@Mode V@Mode VI%
\subsubsection*{Coscale Reflection}
The Mode VI scale can be thought of as a minor triad with added $\flat5$ (a common blue note) and leading notes going up to the root. Alternatively, the Mode I scale is a major triad with added $\sharp2$ ($=\flat3$) and leading notes going down to the fifth -- compare this with the coscale in this set.
%~M|5|1,b3,3,5,b6,6|Mode I@Mode II@Mode III@Mode IV@Mode V@Mode VI%

\subsection{The 100010110011 Scale Set}
\subsection*{Original Scale}
The Mode I scale can be thought of as a major triad with added $\flat5$ (a common blue note) and leading notes going up to the root. Compare it with the original scale from the previous set. 
%~M|5|1,3,#4,5,b7,7|Mode I@Mode II@Mode III@Mode IV@Mode V@Mode VI%
\subsubsection*{Coscale}
This group can perhaps be thought of as a major triad with added $\sharp5$ and leading notes going down to the root (see Mode I).
%~M|5|1,b2,2,3,5,b6|Mode I@Mode II@Mode III@Mode IV@Mode V@Mode VI%
\subsubsection*{Reflection}
Mode III is a minor triad with added $\flat2$ and leading notes going down to the fifth.
%~M|5|1,b2,3,4,5,7|Mode I@Mode II@Mode III@Mode IV@Mode V@Mode VI%
\subsubsection*{Coscale Reflection}
The Mode II scale is a minor triad with a $\natural7$ and leading notes going up to the fifth.
%~M|5|1,b2,3,#4,5,b6|Mode I@Mode II@Mode III@Mode IV@Mode V@Mode VI%

\subsection{The 100010011101 Scale Set}
This set exhibits coscale  symmetry but not reflexive symmetry, meaning there are only two scale groups in this set.
\subsection*{Original Scale}
Mode V is a minor triad with an added 2 and leading notes going up to the root.
%~M|5|1,3,5,b6,6,7|Mode I@Mode II@Mode III@Mode IV@Mode V@Mode VI%
\subsubsection*{Reflection}
The Mode I scale is just a $\Delta$ with leading notes going up to the third.
%~M|5|1,2,b3,3,5,7|Mode I@Mode II@Mode III@Mode IV@Mode V@Mode VI%

\subsection{The 100011010011 Scale Set}
This set exhibits reflexive symmetry, meaning there are only two scale groups in this set.
\subsection*{Original Scale}
The Mode I scale is a major triad with added 4 (11) and leading notes going up to the root.
%~M|5|1,3,4,5,b7,7|Mode I@Mode II@Mode III@Mode IV@Mode V@Mode VI%
\subsubsection*{Coscale}
The Mode IV scale is a minor triad with added 2 and leading notes going down to the fifth.
%~M|5|1,b2,2,4,5,b6|Mode I@Mode II@Mode III@Mode IV@Mode V@Mode VI%

\subsection{The 100010111001 Scale Set}
This set also exhibits reflexive symmetry.
\subsection*{Original Scale}
The Mode II scale can be thought of as a major triad with added $\flat6$ ($\flat13$) and leading notes going up to the third.
%~M|5|1,3,#4,5,b6,7|Mode I@Mode II@Mode III@Mode IV@Mode V@Mode VI%
\subsubsection*{Coscale}
The Mode II scale is a major triad with added $\flat6$ ($\flat13$) and leading notes on either side of the root.
%~M|5|1,b2,2,3,b6,6|Mode I@Mode II@Mode III@Mode IV@Mode V@Mode VI%



\subsection{The 100010011011 Scale Set}
This set also exhibits reflexive symmetry.
\subsection*{Original Scale}
The Mode I scale can be thought of as a major triad with added $\flat6$ ($\flat13$) and leading notes going up to the root.
%~M|5|1,3,5,b6,b7,7|Mode I@Mode II@Mode III@Mode IV@Mode V@Mode VI%
\subsubsection*{Coscale}
The Mode II scale is as a major triad with added $\sharp2$ ($\sharp9$) and leading notes on either side of the root.
%~M|5|1,b2,2,3,4,b6|Mode I@Mode II@Mode III@Mode IV@Mode V@Mode VI%



\section{Containing No Minor Thirds}

All of the scales in this section have interval maps containing two semitones, three tones and one major third. Both sets exhibit reflexive symmetry, limiting the number of such scales to 24 in total. In the case of the second set it may be somewhat interesting to note that the reflection is the same as the coscale, not the original group.

\subsection{The 100010101110 Scale Set}
\subsection*{Original Scale}
The Move V scale is a minor triad with added 6 (13) and leading notes on either side of the root.
%~M|5|1,3,#4,b6,6,b7|Mode I@Mode II@Mode III@Mode IV@Mode V@Mode VI%
\subsubsection*{Coscale}
The Move V scale is a dominant chord with leading notes on either side of the fifth.
%~M|5|1,b2,2,3,#4,b7|Mode I@Mode II@Mode III@Mode IV@Mode V@Mode VI%

\subsection*{The 100010101011 Scale Set}
\subsection*{Original Scale}
The Mode II scale is a major triad with added 2 (9) and leading notes on either side of the fifth.
%~M|5|1,3,#4,b6,b7,7|Mode I@Mode II@Mode III@Mode IV@Mode V@Mode VI%
\subsubsection*{Coscale}
The Mode II scale is a minor triad with added 4 (11) and leading notes on either side of the root.
%~M|5|1,b2,2,3,#4,b6|Mode I@Mode II@Mode III@Mode IV@Mode V@Mode VI%







\chapter{\mbox{Without a} \mbox{Major} \mbox{Third}}

All of the scale sets in this chapter exhibit reflexive symmetry.

\section{Containing Two Minor Thirds}

\subsection{The 100100101101 Scale Set}
This is one of only two hexatonic scale sets we consider that contains two adjacent minor thirds. The other sets in this section have their minor thirds separated by other intervals in different ways. You can think of the fingerings of the original as a subset of those for the octatonic Whole-Half Diminished scale group, and the coscale group similarly resembles the Half-Whole Diminished group.
\subsection*{Original Scale}
This can be thought of as a dominant arpeggio with added $\flat2$ and $\sharp2$ (see Mode IV).
%~M|5|1,b3,#4,b6,6,7|Mode I@Mode II@Mode III@Mode IV@Mode V@Mode VI%
\subsubsection*{Coscale}
This can be thought of as a major triad plus the diminished triad a minor third above it (see Mode VI).
%~M|5|1,b2,b3,3,#4,6|Mode I@Mode II@Mode III@Mode IV@Mode V@Mode VI%

\subsection{The 100100110101 Scale Set}
This is the other scale set that contains two adjacent minor thirds. 
\subsection*{Original Scale}
This scale is a dominant arpeggio plus the minor triad a semitone above (see Mode VI).
%~M|5|1,b3,#4,5,6,7|Mode I@Mode II@Mode III@Mode IV@Mode V@Mode VI%
\subsubsection*{Coscale}
This scale is a major triad plus a minor triad a minor third above it (see Mode VI)
%~M|5|1,b2,b3,3,5,6|Mode I@Mode II@Mode III@Mode IV@Mode V@Mode VI%

\subsection{The 100101011001 Scale Set}
\subsection*{Original Scale}
These scales can be thought of as a minor triad plus a diminished triad a fourth above it (see the Mode I scale).
%~M|5|1,b3,4,5,b6,7|Mode I@Mode II@Mode III@Mode IV@Mode V@Mode VI%
\subsubsection*{Coscale}
These scales can be thought of as a major triad plus a diminished triad an augmented fifth above it (see the Mode II scale).
%~M|5|1,b2,b3,4,b6,6|Mode I@Mode II@Mode III@Mode IV@Mode V@Mode VI%

\subsection{The 100101100110 Scale Set}
\subsection*{Original Scale}
This is composed of a minor triad plus the major triad a whole tone above it (see Mode I).
%~M|5|1,b3,4,#4,6,b7|Mode I@Mode II@Mode III@Mode IV@Mode V@Mode VI%
\subsubsection*{Coscale}
This is a major and minor triad separated by a tritone (see the next scale set for some similar possibilities).
%~M|5|1,b2,b3,#4,5,b7|Mode I@Mode II@Mode III@Mode IV@Mode V@Mode VI%

\subsection{The 100101100101 Scale Set}
\subsection*{Original Scale}
This group can be thought of as a pair of major triads separated by a tritone (diminished fifth).
%~M|5|1,b3,4,#4,6,7|Mode I@Mode II@Mode III`This scale is sometimes referred to as the Petrushka Scale because of its repeated use (as a chord) in Stravinsky's composition of the same name.@Mode IV@Mode V@Mode VI%
\subsubsection*{Coscale}
This group can be thought of as a pair of minor triads separated by a tritone (diminished fifth) -- compare this with the original group, in which the triads were both major.
%~M|5|1,b2,b3,#4,5,6|Mode I@Mode II@Mode III@Mode IV@Mode V@Mode VI%

\section{Containing One Minor Third}

All of these scale sets are, understandably, very similar to the Whole Tone scale. Their relationships to this scale are explained in each case.

\subsection{The 100101010101 Scale Set}

The original scale can be thought of as a Whole Tone scale with one note flattened, whereas in the coscale the note is sharpened. 

\subsection*{Original Scale}
This can be thought of as an augmented triad with a major triad a whole tone above it (see Mode II).
%~M|5|1,b3,4,5,6,7|Mode I@Mode II@Mode III@Mode IV@Mode V@Mode VI%
\subsubsection*{Coscale}
This can be thought of as a minor triad with an augmented triad a semitone above it (see Mode I).
%~M|5|1,b2,b3,4,5,6|Mode I@Mode II@Mode III`This scale contains the notes of the so-called mystic chord from Scriabin's tone poem \emph{Prometheus}; as a consequence some scale lists include it as the Prometheus Scale.@Mode IV@Mode V@Mode VI%

\subsection{The 100101010110 Scale Set}

The original scale can be thought of as a Whole Tone scale with two adjacent notes flattened; in the coscale the two notes are sharpened instead. 

\subsection*{Original Scale}
This is best thought of as two major triads separated by a whole tone (see Mode II)
%~M|5|1,b3,4,5,6,b7|Mode I@Mode II@Mode III@Mode IV@Mode V@Mode VI%
\subsubsection*{Coscale}
This is best thought of as two minor triads separated by a whole tone (see Mode II)
%~M|5|1,b2,b3,4,5,b7|Mode I@Mode II@Mode III@Mode IV`The author has seen this referred to as the Piongio Scale, although the origin of the term is not clear.@Mode V@Mode VI%

\subsection{The 100101011010 Scale Set}

This set exhibits total symmetry -- its coscale and reflection are identical to it -- and is the only non-symmetrical scale group that does so. It is also the only hexatonic we consider that contains only one minor third and no larger interval. It may be considered a Whole Tone scale in which three adjacent notes have been flattened -- that is, the second half of the scale has been shifted down one semitone relative to the first half.

\subsection*{Original Scale}
This scale is a major triad plus a minor triad a whole tone above it (see Mode II)
%~M|5|1,b3,4,5,b6,b7|Mode I@Mode II@Mode III@Mode IV@Mode V@Mode VI%

\chapter{\mbox{With a Large} \mbox{Interval}}

The scales in this chapter all contravene the rule about scales that contain intervals larger than a major third but that don't have more than two consecutive semitones. They are probably less useful than the others covered in previous chapters in this part, but there aren't too many of them. As a result I've included them in this chapter in case you want to explore them.

Note that all of these scale sets are variant, meaning the interval compositions of the coscales differ from those of the originals. The coscales, however, all disobey the rule about consecutive semitones, for reasons that should be obvious, and so only the original scale is included in each case, along with the reflection in the two cases that don't have reflexive symmetry.

\section{The 100001011011 Scale}
\subsection*{Original Scale}
%~M|5|1,4,5,b6,b7,7|Mode I@Mode II@Mode III@Mode IV@Mode V@Mode VI%
\subsection{Reflection}
%~M|5|1,b2,b3,3,#4,7|Mode I@Mode II@Mode III@Mode IV@Mode V@Mode VI%

\section{The 100001011101 Scale}
\subsection*{Original Scale}
%~M|5|1,4,5,b6,6,7|Mode I@Mode II@Mode III@Mode IV@Mode V@Mode VI%
\subsection{Reflection}
%~M|5|1,2,b3,3,#4,7|Mode I@Mode II@Mode III@Mode IV@Mode V@Mode VI%

\section{The 100000111011 Scale}
%~M|5|1,#4,5,b6,b7,7|Mode I@Mode II@Mode III@Mode IV@Mode V@Mode VI%

\section{The 100001101011 Scale}
%~M|5|1,4,#4,b6,b7,7|Mode I@Mode II@Mode III@Mode IV@Mode V@Mode VI%

\section{The 100001101101 Scale}
%~M|5|1,4,#4,b6,6,7|Mode I@Mode II@Mode III@Mode IV@Mode V@Mode VI%

\section{The 100001110011 Scale}
%~M|5|1,4,#4,5,b7,7|Mode I@Mode II@Mode III@Mode IV@Mode V@Mode VI%













\appendix
\chapter{Omitted Scales}

This appendix contains a complete list of all possible scale groups that are \emph{not} included anywhere in this book. Only one representative scale is given from each group; from this you can work out their modes and fingerings if you wish to do so. We occasionlly need the following abbrevations for large intervals:
\begin{tabular}{rl}
	Tr & Tritone \\
	m6 & Minor sixth
\end{tabular}

\subsection*{Four-Note Arpeggios (Tetratonics)}

This `arpeggio' is just four consecutive notes from the Chromatic Scale, the only four-note structure we don't cover:

\begin{tabular}{ll}
    $1$ $6$ $\flat7$ $7$   &   3s 1M6   
\end{tabular}

\subsection*{Pentatonics}
\begin{tabular}{ll}
    $1$ $\flat6$ $6$ $\flat7$ $7$   &   4s 1m6      \\
    $1$ $5$ $6$ $\flat7$ $7$   &   3s 1t 1Fi      \\
    $1$ $5$ $\flat6$ $6$ $\flat7$   &   3s 1t 1Fi      \\
    $1$ $\sharp4$ $6$ $\flat7$ $7$   &   3s 1mT 1Tr      \\
    $1$ $\sharp4$ $5$ $\flat6$ $6$   &   3s 1mT 1Tr      \\
    $1$ $4$ $6$ $\flat7$ $7$   &   3s 1MT 1Fo      \\
    $1$ $4$ $\sharp4$ $5$ $\flat6$   &   3s 1MT 1Fo      
\end{tabular}

\subsection*{Hexatonics}
The following hexatonics contain three consecutive semitones:

\begin{tabular}{ll}
    $1$ $\sharp4$ $5$ $6$ $\flat7$ $7$   &   4s 1t 1Tr      \\
    $1$ $\sharp4$ $5$ $\flat6$ $6$ $7$   &   4s 1t 1Tr      \\
    $1$ $4$ $5$ $6$ $\flat7$ $7$   &   3s 2t 1Fo      \\
    $1$ $4$ $5$ $\flat6$ $6$ $\flat7$   &   3s 2t 1Fo      \\
    $1$ $4$ $\sharp4$ $6$ $\flat7$ $7$   &   4s 1mT 1Fo      \\
    $1$ $4$ $\sharp4$ $5$ $\flat6$ $7$   &   4s 1mT 1Fo      \\
    $1$ $4$ $\sharp4$ $5$ $\flat6$ $\flat7$   &   3s 2t 1Fo      \\
    $1$ $3$ $5$ $6$ $\flat7$ $7$   &   3s 1t 1mT 1MT      \\
    $1$ $3$ $5$ $\flat6$ $6$ $\flat7$   &   3s 1t 1mT 1MT      \\
    $1$ $3$ $\sharp4$ $6$ $\flat7$ $7$   &   3s 1t 1mT 1MT      \\
    $1$ $3$ $\sharp4$ $5$ $\flat6$ $6$   &   3s 1t 1mT 1MT      \\
    $1$ $3$ $4$ $6$ $\flat7$ $7$   &   4s 2MT      \\
    $1$ $3$ $4$ $\sharp4$ $5$ $\flat7$   &   3s 1t 1mT 1MT      \\
    $1$ $3$ $4$ $\sharp4$ $5$ $6$   &   3s 1t 1mT 1MT      \\
    $1$ $\flat3$ $\sharp4$ $6$ $\flat7$ $7$   &   3s 3mT      \\
\end{tabular}

These hexatonics contain four or more consecutive semitones and are less likely to be useful in most circumstances than the ones above:

\begin{tabular}{ll}
    $1$ $5$ $\flat6$ $6$ $\flat7$ $7$   &   5s 1Fi      \\
    $1$ $\sharp4$ $\flat6$ $6$ $\flat7$ $7$   &   4s 1t 1Tr      \\
    $1$ $\sharp4$ $5$ $\flat6$ $6$ $\flat7$   &   4s 1t 1Tr      \\
    $1$ $4$ $\flat6$ $6$ $\flat7$ $7$   &   4s 1mT 1Fo      \\
    $1$ $4$ $\sharp4$ $5$ $\flat6$ $6$   &   4s 1mT 1Fo      \\
    $1$ $3$ $\flat6$ $6$ $\flat7$ $7$   &   4s 2MT      \\
\end{tabular}

\subsection*{Heptatonics}
The following contain four or more consecutive semitones; the text does cover a few in that category, but not many:

\begin{tabular}{ll}
    $1$ $\sharp4$ $5$ $\flat6$ $6$ $\flat7$ $7$   &   6s 1Tr      \\
    $1$ $4$ $5$ $\flat6$ $6$ $\flat7$ $7$   &   5s 1t 1Fo      \\
    $1$ $4$ $\sharp4$ $\flat6$ $6$ $\flat7$ $7$   &   5s 1t 1Fo      \\
    $1$ $4$ $\sharp4$ $5$ $\flat6$ $6$ $7$   &   5s 1t 1Fo      \\
    $1$ $4$ $\sharp4$ $5$ $\flat6$ $6$ $\flat7$   &   5s 1t 1Fo      \\
    $1$ $3$ $5$ $\flat6$ $6$ $\flat7$ $7$   &   5s 1mT 1MT      \\
    $1$ $3$ $\sharp4$ $\flat6$ $6$ $\flat7$ $7$   &   4s 2t 1MT      \\
    $1$ $3$ $\sharp4$ $5$ $\flat6$ $6$ $\flat7$   &   4s 2t 1MT      \\
    $1$ $3$ $4$ $\sharp4$ $5$ $\flat6$ $7$   &   5s 1mT 1MT      \\
    $1$ $3$ $4$ $\sharp4$ $5$ $\flat6$ $\flat7$   &   4s 2t 1MT      \\
    $1$ $3$ $4$ $\sharp4$ $5$ $\flat6$ $6$   &   5s 1mT 1MT      \\
    $1$ $\flat3$ $\sharp4$ $\flat6$ $6$ $\flat7$ $7$   &   4s 1t 2mT      \\
    $1$ $\flat3$ $\sharp4$ $5$ $\flat6$ $6$ $7$   &   4s 1t 2mT      \\
    $1$ $\flat3$ $\sharp4$ $5$ $\flat6$ $6$ $\flat7$   &   4s 1t 2mT      \\
    $1$ $\flat3$ $4$ $\sharp4$ $5$ $\flat6$ $\flat7$   &   3s 3t 1mT      
\end{tabular}

\subsection*{Octatonics}
The following contain four or more consecutive semitones:

\begin{tabular}{ll}
    $1$ $4$ $\sharp4$ $5$ $\flat6$ $6$ $\flat7$ $7$   &   7s 1Fo      \\
    $1$ $3$ $\sharp4$ $5$ $\flat6$ $6$ $\flat7$ $7$   &   6s 1t 1MT      \\
    $1$ $3$ $4$ $5$ $\flat6$ $6$ $\flat7$ $7$   &   6s 1t 1MT      \\
    $1$ $3$ $4$ $\sharp4$ $\flat6$ $6$ $\flat7$ $7$   &   6s 1t 1MT      \\
    $1$ $3$ $4$ $\sharp4$ $5$ $\flat6$ $\flat7$ $7$   &   6s 1t 1MT      \\
    $1$ $3$ $4$ $\sharp4$ $5$ $\flat6$ $6$ $7$   &   6s 1t 1MT      \\
    $1$ $3$ $4$ $\sharp4$ $5$ $\flat6$ $6$ $\flat7$   &   6s 1t 1MT      \\
    $1$ $\flat3$ $\sharp4$ $5$ $\flat6$ $6$ $\flat7$ $7$   &   6s 2mT      \\
    $1$ $\flat3$ $4$ $5$ $\flat6$ $6$ $\flat7$ $7$   &   5s 2t 1mT      \\
    $1$ $\flat3$ $4$ $\sharp4$ $5$ $\flat6$ $6$ $7$   &   5s 2t 1mT      \\
    $1$ $\flat3$ $4$ $\sharp4$ $5$ $\flat6$ $6$ $\flat7$   &   5s 2t 1mT      \\
    $1$ $\flat3$ $3$ $5$ $\flat6$ $6$ $\flat7$ $7$   &   6s 2mT      \\
    $1$ $\flat3$ $3$ $\sharp4$ $\flat6$ $6$ $\flat7$ $7$   &   5s 2t 1mT      \\
    $1$ $\flat3$ $3$ $\sharp4$ $5$ $\flat6$ $6$ $\flat7$   &   5s 2t 1mT      \\
    $1$ $\flat3$ $3$ $4$ $\flat6$ $6$ $\flat7$ $7$   &   6s 2mT      \\
    $1$ $\flat3$ $3$ $4$ $\sharp4$ $5$ $6$ $7$   &   5s 2t 1mT      \\
    $1$ $\flat3$ $3$ $4$ $\sharp4$ $5$ $\flat6$ $\flat7$   &   5s 2t 1mT      \\
    $1$ $2$ $3$ $\sharp4$ $\flat6$ $6$ $\flat7$ $7$   &   4s 4t      
\end{tabular}

\subsection*{Nonatonics}

The following contain four or more consecutive semitones:

\begin{tabular}{ll}
    $1$ $3$ $4$ $\sharp4$ $5$ $\flat6$ $6$ $\flat7$ $7$   &   8s 1MT      \\
    $1$ $\flat3$ $4$ $\sharp4$ $5$ $\flat6$ $6$ $\flat7$ $7$   &   7s 1t 1mT      \\
    $1$ $\flat3$ $3$ $\sharp4$ $5$ $\flat6$ $6$ $\flat7$ $7$   &   7s 1t 1mT      \\
    $1$ $\flat3$ $3$ $4$ $5$ $\flat6$ $6$ $\flat7$ $7$   &   7s 1t 1mT      \\
    $1$ $\flat3$ $3$ $4$ $\sharp4$ $\flat6$ $6$ $\flat7$ $7$   &   7s 1t 1mT      \\
    $1$ $\flat3$ $3$ $4$ $\sharp4$ $5$ $6$ $\flat7$ $7$   &   7s 1t 1mT      \\
    $1$ $\flat3$ $3$ $4$ $\sharp4$ $5$ $\flat6$ $\flat7$ $7$   &   7s 1t 1mT      \\
    $1$ $\flat3$ $3$ $4$ $\sharp4$ $5$ $\flat6$ $6$ $7$   &   7s 1t 1mT      \\
    $1$ $\flat3$ $3$ $4$ $\sharp4$ $5$ $\flat6$ $6$ $\flat7$   &   7s 1t 1mT      \\
    $1$ $2$ $3$ $\sharp4$ $5$ $\flat6$ $6$ $\flat7$ $7$   &   6s 3t      \\
    $1$ $2$ $3$ $4$ $5$ $\flat6$ $6$ $\flat7$ $7$   &   6s 3t      \\
    $1$ $2$ $3$ $4$ $\sharp4$ $\flat6$ $6$ $\flat7$ $7$   &   6s 3t      \\
    $1$ $2$ $3$ $4$ $\sharp4$ $5$ $6$ $\flat7$ $7$   &   6s 3t      \\
    $1$ $2$ $3$ $4$ $\sharp4$ $5$ $\flat6$ $\flat7$ $7$   &   6s 3t      \\
    $1$ $2$ $3$ $4$ $\sharp4$ $5$ $\flat6$ $6$ $7$   &   6s 3t      \\
    $1$ $2$ $\flat3$ $4$ $\sharp4$ $\flat6$ $6$ $\flat7$ $7$   &   6s 3t 
\end{tabular}


\subsection*{Decatonics}

In the language of Part III, the decatonic scales are the coscales of intervals. There are therefore six modal groups, one for each of the modal groups of intervals. All decatonic scales contain at least four consecutive semitones; the only one included in this book is the symmetrical Tritone Coscale.

\begin{tabular}{ll}
    $1$ $\flat3$ $3$ $4$ $\sharp4$ $5$ $\flat6$ $6$ $\flat7$ $7$   &   9s 1mT      \\
    $1$ $2$ $3$ $4$ $\sharp4$ $5$ $\flat6$ $6$ $\flat7$ $7$   &   8s 2t      \\
    $1$ $2$ $\flat3$ $4$ $\sharp4$ $5$ $\flat6$ $6$ $\flat7$ $7$   &   8s 2t      \\
    $1$ $2$ $\flat3$ $3$ $\sharp4$ $5$ $\flat6$ $6$ $\flat7$ $7$   &   8s 2t      \\
    $1$ $2$ $\flat3$ $3$ $4$ $5$ $\flat6$ $6$ $\flat7$ $7$   &   8s 2t      
\end{tabular}


\subsection*{Hendecatonics}

The only possible hendecatonic scale group is the one containing the Chromatic Scales with one note missed out.

\begin{tabular}{ll}
    $1$ $2$ $\flat3$ $3$ $4$ $\sharp4$ $5$ $\flat6$ $6$ $\flat7$ $7$   &   10s 1t      
\end{tabular}


\subsection*{Dodecatonics}

The only dodecaphonic scale is, of course, the Chromatic Scale itself.

\begin{tabular}{ll}
    $1$ $\flat2$ $2$ $\flat3$ $3$ $4$ $\sharp4$ $5$ $\flat6$ $6$ $\flat7$ $7$   &   12s      
\end{tabular}







\begin{thebibliography}{99}
	\bibitem{bergonzi}Bergonzi, Jerry, \emph{Inside Improvisation Series Vol 7: Hexatonics}, Advance Music, Germany, 2006
	\bibitem{campbell}Campbell, Gary, \emph{Expansions}, Houston Publishing, 1998
	\bibitem{campbell2}Campbell, Gary, \emph{Triad Pairs for Jazz}, Alfred Publishing, United States, 2001
	\bibitem{coker}Coker, Jerry \emph{et al.}, \emph{Patterns for Jazz}, Studio PR, Miami, 1970
	\bibitem{forte}Forte, Allen, \emph{The Structure of Atonal Music}, Yale University Press, United States, 1977
	\bibitem{friedman}Friedman, Michael L, \emph{Ear Training for Twentieth-Century Music}, Yale University Press, London, 1991
	\bibitem{green2}Green, Andrew, \emph{Jazz Guitar Structures}
	\bibitem{greene1}Greene, Ted, \emph{Chord Chemistry}, Dale Zdenek, United States, 1971
	\bibitem{greene2}Greene, Ted, \emph{Jazz Guitar Single-Note Soloing} (two volumes), Dale Zdenek, United States, 1978
	\bibitem{holdsworth}Holdsworth, Allan, \emph{Just for the Curious}, CPP Media, United States, 1993
	\bibitem{khan}Khan, Steve, \emph{Contemporary Chord Khancepts}, Manhatten Music, New York, 1996
	\bibitem{lewin}Lewin, David, \emph{Generalized Musical Intervals and Transformations}, Oxford University Press, United States, 2007
	\bibitem{liebman}Liebman, David, \emph{A Chromatic Approach to Jazz Harmony and Melody}, Advance Music, Germany, 1991
	\bibitem{mariano}Mariano, Charlie, \emph{An Introduction to South Indian Music}, Advance Music, Germany, 2000
	\bibitem{munro}Munro, Doug, \emph{Jazz Guitar: Bebop and Beyond}, Alfred Publishing, United States, 2002
	\bibitem{russell}Russell, George, \emph{The Lydian Chromatic Concept of Tonal Organization}, Concept, United States, 2002
	\bibitem{scottlloyd}Scott-Lloyd, Marcus, \emph{Arab Music Theory in the Modern Period}, unpublished PhD thesis, University of California, United States, 1989
	\bibitem{Serafine}Serafine, Mary Louise, \emph{Music as Cognition: The Development of Thought in Sound}, Colombia Univerity Press, New York, 1988
	\bibitem{slonimsky}Slonimsky, Nicolas, \emph{Thesaurus of Scales and Melodic Patterns}, Schirmer Books, United States, 1947
	\bibitem{Tymoczko}Tymoczko, Dmitri, `The Consecutive-Semitone Constraint on Scalar Structure: A Link Between Impressionism and Jazz', in \emph{Integral}, Vol 11, pp. 135-179, 1997
	\bibitem{vai}Vai, Steve, \emph{The Frank Zappa Guitar Book}, Munchkin Music, Los Angeles, 1981
	\bibitem{weiskopf}Weiskopf, Walt, \emph{Intervallic Improvisation: The Modern Sound}, Jamey Aebersold, Indiana, 1995
	\bibitem{white}White, Andrew, various Coltrane transcriptions, Andrew's Music, 4830 South Dakota Ave, NE Washington DC, 20017 (transcriptions sold individually and not available in shops -- send return postage for instructions on acquiring the catalogue)
\end{thebibliography}|

\end{large}

\end{document}

